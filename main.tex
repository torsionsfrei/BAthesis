% \chapter{Einleitung}


% \par
% In \cite{kato_1954}
% %[\textsc{Kato}: \textit{On the semi-groups generated by Kolmogoroff's differential equations}. Journal of the Mathematical Society of Japan, 1954] 
% zeigt \textsc{T. Kato} die Existenz einer $C_0$-Halbgruppe, welche das Kolmogorov'sche Differentialgleichungssystem eines klassischen Geburts- und Todesprozesses in $\ell^1$ löst. Das Ergebnis wurde u. a. von \textsc{J. Voigt} in \cite{voigt_1989} %[\textsc{Voigt}: \textit{On resolvent positive operators and positive $C_0$-semigroups on AL-spaces}. Semigroup Forum, 1989] 
% auf die Klasse der AL-Räume erweitert, welche die Banachverbände  $\ell^1$ und $\text{L}^1(\Omega,\mu)$ umfasst. In dieser Arbeit wird das Ergebnis von \textsc{J. Banasiak} et al. in \cite{banasiak_lachowicz_2007}
% %[\textsc{Banasiak, Lachowicz}: \textit{Around the Kato generation theorem for semigroups}. Studia Mathematica, 2007] 
% vorgestellt, welches die Existenz von Lösungen auf die Klasse der Kantorovich-Banachräume verallgemeinert. Diese umfasst neben den AL-Räumen auch die reflexiven Banachverbände $\ell^p$ und $\text{L}^p(\Omega,\mu)$. 
\chapter{Einleitung}

\par
In \cite{kato_1954}
%[\textsc{Kato}: \textit{On the semi-groups generated by Kolmogoroff's differential equations}. Journal of the Mathematical Society of Japan, 1954] 
zeigt \textsc{T. Kato} die Existenz einer $C_0$-Halbgruppe, welche das Kolmogorov'sche Differentialgleichungssystem eines Geburts- und Todesprozesses in $\ell^1$ löst. Das Ergebnis wurde u. a. von \textsc{J. Voigt} in \cite{voigt_1989} %[\textsc{Voigt}: \textit{On resolvent positive operators and positive $C_0$-semigroups on AL-spaces}. Semigroup Forum, 1989] 
auf die Klasse der $AL$-Räume erweitert, welche die Banachverbände  $\ell^1$ und $\text{L}^1(\Omega,\mu)$ umfasst.
\par
In dieser Arbeit wird das Ergebnis von \textsc{J. Banasiak} et al. in \cite{banasiak_lachowicz_2007}
%[\textsc{Banasiak, Lachowicz}: \textit{Around the Kato generation theorem for semigroups}. Studia Mathematica, 2007] 
vorgestellt, welches die Existenz von Lösungen auf die Klasse der $KB$-Räume verallgemeinert. Diese umfasst neben den $AL$-Räumen auch die reflexiven Banachverbände $\ell^p$ und $\text{L}^p(\Omega,\mu)$. 
\par

\chapter{Geordnete Vektorräume}

% \newpage\section{Normierte Räume}

% \begin{defi}
% Sei $X$ Vektorraum. Eine Abbildung $p\colon X\to[0,\infty)$ heißt \index{Vektorraum!Norm}\textbf{Norm}, falls gilt:
% \begin{enumerate}
% \item $p(\lambda x)=|\lambda|p(x)$ für alle Skalare $\lambda$ und $x\in X$.
% \item $p(x+y)\leq p(x)+p(y)$ für alle $x,y\in X$ (\textbf{Dreiecksungleichung}).
% \item $p(x)=0$ impliziert $x=0$. 
% \end{enumerate}
% Ist $X$ bzgl. dieser Norm ein vollständiger Vektorraum, so bezeichne diesen als \index{Vektorraum!Banachraum}\textbf{Banachraum}.
% \end{defi}


% \begin{bsp}
% Sei $(\Omega, \Sigma,\mu)$ ein \textbf{Maßraum}, d. h. wir haben eine Menge $\Omega$ zusammen mit einer $\sigma$-Algebra $\Sigma$ von offenen Teilmengen von $\Omega$ sowie einem $\sigma$-additiven Maß $\mu$ auf $\Sigma$. 

% \par
% Für gewöhnlich ist $\Omega\subseteq \R^n$ und $\Sigma$ die $\sigma$-Algebra der \textbf{Lebesgue-messbaren} Mengen. In diesem Fall bezeichne $\mu$ als \textbf{Lebesguemaß}. Dieses ist ein \textbf{$\sigma$-endliches} Maß, d. h. wir können $\Omega$ als abzählbare Vereinigung von Mengen mit endlichem Maß darstellen.

% \par
% Eine Funktion $f\colon\Omega\to \R$ heißt \textbf{messbar}, falls $f^{-1}(B)\in\Sigma$ für Lebesgue-Mengen $B\subseteq \R$ gilt. Wir identifizieren diejenigen Funktionen miteinander, welche auf $\mu$-messbaren Mengen mit dem Maß $0$ übereinstimmen. 

% \par 
% Dann bezeichne $L^0(\Omega, \textnormal{d}\mu)=:L^0(\Omega)$ die Menge aller Äquivalenzklassen messbarer reellwertiger Funktionen auf $\Omega$. Mit den punktweisen Verknüpfungen der Addition und Skalarmultiplikation wird $L^0(\Omega)$ zu einem Vektorraum. Das \textbf{Integral} einer Menge $\Omega$ einer messbaren Funktion $f$ bzgl. $\mu$ wird mit $\int_\Omega f(x)\textnormal d\mu$ bezeichnet.

% \par
% Für $1\leq p< \infty$ bezeichne der Teilraum $\textnormal{L}^p(\Omega,\mu)$ die Menge aller $f\in L^0(\Omega)$, für die gilt:
% \begin{equation*}
% \|f\|_p:=\|f\|_{\textnormal{L}^p(\Omega,\mu)}:=\Big(\int_\Omega |f(x)|^p\textnormal dx\Big)^{1/p}<\infty
% \end{equation*}
% Dann ist $\|f\|_p$ für alle $1\leq p< \infty$ eine Norm und $\textnormal{L}^p(\Omega,\mu)$ wird mit dieser Norm zu einem Banachraum. 

% \par
% Weiter sei $L^\infty(\Omega)$ der Vektorraum aller reellwertiger messbaren Funktionen, die (bis auf $\mu$-Nullmengen) beschränkt sind. Die zugehörige Norm $\|\cdot\|_\infty$ ist gegeben durch: 
% \begin{equation*}
% \|f\|_\infty:=\|f\|_{L^\infty(\Omega)}:=\inf\{M;\mu(\{x\in\Omega; |f(x)|<M\})=0\}<\infty.
% \end{equation*}

% \end{bsp}

% \begin{bsp}
% \par
% Ist die Menge $\Omega$ abzählbar, so identifiziere diese mit $\N$ (bzw. $\N_0$). In diesem Fall wird $(\Omega,\Sigma, \mu)$ mit dem \textbf{Zählmaß} zu einem Maßraum. Für $1\leq p< \infty$ bezeichne dann $L^p(\N)=:\ell^p$ den Banachraum aller Folgen $\textbf{x}=(x_n)_{n\in\N}$, für die gilt:
% \begin{equation*}
% \|\textbf{x}\|_p:=\Big(\sum_{n=1}^\infty|x_n|^p\Big)^{1/p}< \infty.
% \end{equation*}
% Weiter bezeichne $l^\infty$ den Vektorraum aller Folgen $\textbf{x}=(x_n)_{n\in\N}$ mit $\|\textbf{x}\|_\infty:=\sup_{n\in\N}\|x_n\|<\infty$.
% \end{bsp}


% \hl{Lorem ipsum dolor sit amet, consectetur adipiscing elit. Pellentesque vulputate pellentesque nunc, ut iaculis purus ornare nec. Nunc finibus rhoncus odio, non maximus tortor elementum eu. Duis rutrum tincidunt dignissim. Donec vel urna non felis congue iaculis. Aenean eu velit sagittis, tempor magna eu, volutpat urna.}


Wir beginnen mit einer kurzen Wiederholung des Begriffs des Banachverbands, welcher grundlegend für diese Arbeit ist.  Anschließend stellen wir die speziellen Banachverbände der $KB$- und $AL$-Räume vor (vgl. \cite{aliprantis_burkinshaw_2006}, \cite{banasiak_arlotti_2006}). 
% Zuletzt wir Kriterien für die monotone bzw. dominierte Konvergenz positiver Folgen in Banachverbänden sicher, welche  angewendet auf $KB$-Räume für den Beweis des verallgemeinerten Störungstheorems (\Cref{Störungstheorem nach Kato}) von Bedeutung sind.


\section{Banachverbände}

% die Arbeit ist fast komplett, allerdings wäre es gut, wenn Du noch ein paar Ein- und Überleitungen ergänzen könntest. Hier z.B. "Wir beginnen mit einer Einführung/Wiederholung des Begriffs des Banachverbands, welcher grundlegend für diese Arbeit ist." (oder etwas ähnliches) Es lohnt sich, das ufgeschriebene erst zu motivieren, damit der Leser besser versteht, warum etwas definiert oder bewiesen wird... Am besten man verweist am Anfang des Abschnitts auch auf seine Quellen und Standardliteratur zu diesem Thema (Schaefer...)


\begin{defi}
Sei $X$ eine Menge. Eine (partielle) \index{Ordnung}\textbf{Ordnung} ist eine (binäre) Relation "$\geq$"\; auf $X$ mit den Eigenschaften:
\begin{enumerate}
\item Für alle $x\in X$ ist $x\geq x$.
\item $x\geq y$ und $y\geq x$ $\Rightarrow$ $x=y$ für alle $x,y\in X$.  
\item $x\geq y$ und $y\geq z$ $\Rightarrow$ $x\geq z$ für alle $x,y, z\in X$.
\end{enumerate}
Für $x,y\in X$ schreibe $x\leq y$, falls $y\geq x$ gilt. 
\par
Eine \index{Ordnung!Schranke}\textbf{obere Schranke} einer Teilmenge $S\subseteq X$ ist ein Element $x\in X$ mit $x\geq y$ für alle $y\in S$. Analog definiere eine \textbf{untere Schranke} einer Teilmenge von $X$.  
% Weiter heißt die Teilmenge $S$ nach oben (bzw. unten) \index{Ordnung!gerichtet}\textbf{gerichtet}, falls für alle $x,y\in S$ eine obere (bzw. untere) Schranke existiert.
Das \index{Ordnung!Supremum, Infimum}\textbf{Supremum} einer Teilmenge von $X$ bezeichne die kleinste obere Schranke, das \textbf{Infimum} die größte untere Schranke. 

\par
Nenne eine geordnete Menge $X$ einen \index{Ordnung!Verband}\textbf{Verband}, falls für alle Paare $x,y\in X$ sowohl das Supremum $x\vee y:=\sup\{x,y\}$ als auch Infimum $x\wedge y:=\inf\{x,y\}$ in $X$ existieren.
\end{defi}

\begin{defi}
Eine (reeller) Vektorraum $X$ heißt \index{Vektorraum!Ordnung}\textbf{geordnet}, falls es eine Ordnung "$\geq$"\; auf der Menge $X$ gibt mit den Eigenschaften:
\begin{enumerate}
\item $x\geq y$ $\Rightarrow$ $x+z\geq y+z$ für alle $x,y,z\in X$. 
\item $x\geq y$ $\Rightarrow$ $\alpha x\geq \alpha y$ für alle $x,y\in X$ und $\alpha\geq0$.
\end{enumerate}
Die Menge $X_+:=\{x\in X; x\geq0\}$ heißt \index{Vektorraum!Positiver Kegel}\textbf{positiver Kegel} in $X$. 
\par
Ein geordneter Vektorraum $X$ heißt \index{Vektorraum!Vektorverband}\textbf{Vektorverband}, falls die zugrunde liegende geordnete Menge $X$ ein Verband ist. 
\end{defi}

\begin{bsp}
Sei $X$ die Menge aller reellwertigen Funktionen auf einer Menge $\Omega$. Dann wird $X$  zusammen mit den punktweisen Verknüpfungen zu einem Vektorraum.
Auf $X$ ist eine kanonische Ordnung gegeben durch
\begin{equation*}
f\leq g:\iff f(x)\leq g(x),\quad\forall x\in \Omega,\forall f,g\in X.
\end{equation*}
Für alle $f,g\in X$ ist
\begin{equation*}
(f\vee g)(x)=\max\{f(x), g(x)\},\quad (f\wedge g)(x)=\min\{f(x), g(x)\},\quad\forall x\in\Omega.
\end{equation*}
Dann ist $X$ ein Vektorverband, falls $f\vee g, f\wedge g\in X$ für alle $f,g\in X$ erfüllt ist. 
%In diesem Fall bezeichne den Vektorverband $X$ auch als \index{Vektorraum!Funktionenraum}\textbf{Funktionenraum}. 
\par

\begin{bsp}
Für $1\leq p\leq\infty$ sind die Räume $\ell^p$ und $\textnormal{L}^p(\Omega,\mu)$  Vektorverbände.
\end{bsp}


% Es ist nicht unmittelbar klar, dass dies UnterVERBÄNDE sind. Man muss sich klar darüber sein, dass zwei integrierbare Funktionen ein Infimum und Supremum besitzen.

\end{bsp}

\begin{bem}
Sei $X$  ein Vektorraum. Eine Teilmenge $C\subseteq X$ heißt \textbf{konvexer Kegel} in $X$, falls gilt:
\begin{enumerate}
\item $C+C\subseteq C$.
\item $\alpha C\subseteq C$ für alle $\alpha\geq0$.
\item $C\cap(-C)=\{0\}$.
\end{enumerate}
Einen konvexer Kegel $C\subseteq X$ heißt \textbf{erzeugend}, falls $X=C-C$ gilt.
\end{bem}

\begin{bsp}%[\cite{banasiak_arlotti_2006}, 2.4.4]
Der positive Kegel $X_+=\{x\in X; x\geq 0\}$ eines Vektorverbandes $X$ ist ein konvexer Kegel. 
\end{bsp}

\begin{defi}
Sei $X$ ein Vektorverband. Für ein Element $x\in X$ sind der zugehörige  \textbf{Positivteil}, \textbf{Negativteil} und \textbf{Absolutbetrag} jeweils gegeben durch
\begin{equation*}
x_+:=x\vee 0,\quad x_-:=-x \wedge 0\quad\text{ und }\quad |x|:=x \vee -x.
\end{equation*}
\end{defi}



\begin{prop}
In einem Vektorverband $X$ gilt stets
\begin{equation*}
x = x_+ - x_-\quad \text{und}\quad |x| = x_+ + x_-,\quad \forall x\in X.
\end{equation*}
Insbesondere ist der positive Kegel $X_+$ in $X$ erzeugend.
\end{prop}

\begin{proof}
Siehe \cite{aliprantis_burkinshaw_2006}, Theorem 1.5.
\end{proof}


\begin{defi}
Sei $X$ ein Vektorverband und $\|\cdot\|$ eine Norm auf $X$. Dann bezeichne  $\|\cdot\|$ als \textbf{Verbandsnorm}, falls gilt:
\begin{equation*}\label{identity}
|x |\leq  |y |\Rightarrow  \|x \|\leq  \|y \|,\quad\forall x,y\in X.
\end{equation*}
Ist ein Vektorverband $X$ bzgl. der Verbandsnorm ein vollständiger Raum, so heißt dieser \index{Vektorraum!Banachverband}\textbf{Banachverband}.
\end{defi}

\begin{bsp}
Für $1\leq p<\infty$ sind die Vektorverbände $\ell^p$ und $\textnormal{L}^p(\Omega,\mu)$ zusammen mit den kanonischen Normen $\|\cdot \|_{\ell^p}$ bzw. $\|\cdot\|_{\textnormal{L}^p(\Omega,\mu)}$ Banachverbände.
\end{bsp}

\begin{bem}\label{Abgeschlossenheit des Kegels}
Ist $X$ ein normierter Vektorverband, so sind die Verbandsoperationen $X\ni x\mapsto x_\pm, |x|\in X$ stetig bzgl. der Norm. Insbesondere ist der positive Kegel $X_+=\{x\in X; x_-=0\}$ eine bzgl. der Norm  abgeschlossene Teilmenge in $X$.
\end{bem}


\begin{bem}\label{Norm und Betrag in Vektorverbänden}
In einem Vektorverband  $X$ gilt stets
\begin{equation*}
\|x \| =  \| |x | \|,\quad\forall x\in X.
\end{equation*}
\end{bem}



 
\newpage\section{$KB$- und $AL$-Räume}

\begin{defi} 
Ein Banachverband $X$ heißt \index{Vektorraum!$AL$-Raum}\textbf{$AL$-Raum} (\textbf{Abstrakter $L$-Raum}), falls gilt:
\begin{equation*}
\|x+y\|=\|x\|+\|y\|,\quad\forall x,y\in X_+.
\end{equation*}
\end{defi}

% \begin{defi}
% \hl{Isomorphie $\cong$ von Banachverbänden?}
% \end{defi}


% \begin{bem}
% \hl{Lorem ipsum dolor sit amet, consectetur adipiscing elit. Pellentesque vulputate pellentesque nunc, ut iaculis purus ornare nec. Nunc finibus rhoncus odio, non maximus tortor elementum eu. Duis rutrum tincidunt dignissim. Donec vel urna non felis congue iaculis. Aenean eu velit sagittis, tempor magna eu, volutpat urna.}
% \end{bem}

\begin{satz}[Kakutani–Bohnenblust–Nakan]\label{Charakterisierung von AL-Räumen}\index{Vektorraum!Charakterisierung von $AL$-Räumen}
Sei $X$ ein Banachverband. \newline Dann sind äquivalent:
\begin{enumerate}
\item $X$ ist $AL$-Raum.
\item Es gibt einen Maßraum $(\Omega,\mu)$ so, dass $X\cong \textnormal{L}^1(\Omega,\mu)$ gilt.
\end{enumerate}
\end{satz}

\begin{proof}
Siehe \cite{aliprantis_burkinshaw_2006}, Theorem 4.27.
\end{proof}

% \begin{defi}
% Sei $X$ Vektorverband und $(x_n)_{n\in\N}$ eine Folge in $X$.
% \begin{enumerate}
% \item Für $(x_n)_{n\in\N}$ monoton fallend schreibe  $x_n\downarrow x$, falls $\inf_{n\in\N} x_n=x$ gilt.
% \item Für $(x_n)_{n\in\N}$ monoton steigend schreibe $x_n\uparrow x$, falls $\sup_{n\in\N} x_n=x$ gilt.
% \end{enumerate}
% Dann heißt die Folge $(x_n)_{n\in\N}$ \textbf{ordnungskonvergent} mit Grenzwert $x\in X$, falls es monotone Folgen $(a_n)_{n\in\N}$ und $(a_n)_{n\in\N}$ in $X$ gibt  mit $b_n\downarrow x$ und $a_n\uparrow x$ so, dass $a_n\leq x_n\leq b_n$ für alle $n\in\N$ gilt.
% \end{defi}


% \begin{defi}
% Sei $X$ Menge sowie $\Delta$ eine gerichtete Menge.
% \begin{enumerate}
% \item Ein \index{Netz}\textbf{Netz} $(x_\alpha)_{\alpha\in\Delta}$  ist eine Abbildung von $\Delta$ in $X$.
% \item Sei nun $X$ ein normierter Vektorraum. Dann bezeichne $(x_\alpha)_{\alpha\in\Delta}$ als \index{Netz!normkonvergent}\textbf{(norm-)konvergent} mit Grenzwert $x\in X$, falls für alle $\epsilon >0$ ein $\alpha_0\in\Delta$ existiert so, dass $\|x_\alpha - x\|\leq \epsilon$ für alle $\alpha\geq \alpha_0$ gilt.
% \item Ein Netz  $(x_\alpha)_{\alpha\in\Delta}$ heißt \index{Netz!monoton}\textbf{monoton fallend}, schreibe $x_\alpha\downarrow$, falls für alle $\alpha_1,\alpha_2\in \Delta$ mit $\alpha_1\geq\alpha_2$ die Ungleichung $x_{\alpha_1}\leq x_{\alpha_2}$ gilt. Die Notation $x_\alpha\downarrow x$ bezeichne ein Netz $(x_\alpha)_{\alpha\in\Delta}$, für das $x_\alpha\downarrow$ und $\inf_{\alpha\in\Delta}x_\alpha = x$ gilt. Analog definiere $x_\alpha\uparrow$ und $x_\alpha\uparrow x$ für ein  \textbf{monoton steigendes} Netz $(x_\alpha)_{\alpha\in\Delta}$.
% \end{enumerate}   
% \end{defi}

% \begin{bsp}
% Für eine beliebige gerichtete Menge  $\Delta$ ist die Identität $I\colon\Delta\to\Delta$ ein Netz.
% \end{bsp}

% \begin{prop}[\cite{banasiak_arlotti_2006}, 2.73]\label{Abgeschlossenheit des positiven Kegels}
% Sei $X$ ein Vektorverband. Dann gilt
% \begin{enumerate}
% \item Der positive Kegel $X_+$ ist abgeschlossen.
% \item Ist $(x_\alpha)_{\alpha\in\Delta}$ eine monoton fallende, normkonvergente Folge in $X$ mit $x_n\to x$, so gilt $\inf_{\alpha\in\Delta} x_\alpha=x$.
% \item Ist $(x_\alpha)_{\alpha\in\Delta}$ eine monoton steigende, normkonvergente Folge in $X$ mit $x_\alpha\to x$, so gilt $\sup_{\alpha\in\Delta} x_\alpha=x$.
% \end{enumerate}
% \end{prop}

% \begin{bsp}
% Sei $X$ ein Banachverband. Dann gibt es Folgen $(x_n)_{n\in\N}$ in $X$ mit $x_n\uparrow x$, die nicht normkonvergent sind.
% \end{bsp}

% \begin{proof}
% Für $n\in\N$ setze $x_n:=(1,\dots, 1,0,\dots)$, wobei $1$ für die ersten $n$ Einträge sei und $0$ sonst. Dann ist $x_n\uparrow x$ mit $x=(1,\dots, 1,\dots)\in l^\infty$ und  $\sup_{n\in\N}x_n=x$. Jedoch ist $\|x_n - x\|_\infty = 1$ für alle $n\in\N$, also ist $(x_n)_{n\in\N}$ nicht normkonvergent.
% \end{proof}

% \begin{defi}
% Sei $X$ ein Banachverband. Dann bezeichne die zugehörige Norm $\|\cdot \|$ als \index{Vektorraum!ordnungsstetig}\textbf{ordnungsstetig}, falls für jedes Netz $(x_\alpha)_{\alpha\in\Delta}$ in $X$ gilt:
% \begin{equation*}
% x_\alpha\downarrow 0\Rightarrow \|x_\alpha\|\downarrow 0.
% \end{equation*}
% \end{defi}

% \begin{satz}[\cite{aliprantis_burkinshaw_2006}, 1.56]
% Für einen Banachverband $X$ sind äquivalent:
% \begin{enumerate}
% \item Die Norm von $X$ ist ordnungsstetig.
% \item Jede positive Folge $(x_n)_{n\in\N}$ mit $0\leq x_n\uparrow x$ in $X$ ist eine Cauchyfolge.
% \item $X$ ist \hl{$\sigma$-Ordnung vollständig} und es für jede Folge $(x_n)_{n\in\N}$ in $X$ gilt:
% \begin{equation*}
% x_n\downarrow 0\Rightarrow \|x_n\|\to 0.
% \end{equation*}
% \end{enumerate}
% \end{satz}

% \begin{bsp}
% Der Vektorraum $C([0,1])$ ist nicht vollständig bzgl. der  $\sigma$-Ordnung: 
% \end{bsp}

% \begin{bsp}
% Für $1\leq p< \infty$ ist die Norm $\|\cdot\|_p$ von  $\textnormal{L}^p(\Omega,\mu)$ ordnungsstetig.
% \end{bsp}

% \begin{proof}
% \par
% Sei $f_n\downarrow 0$. Mit dem \textbf{Satz der dominierten Konvergenz} [...] ist dann $\|f_n\|^p=\int_\Omega f_n^p\text d\mu\to 0$. Da $\textnormal{L}^p(\Omega,\mu)$ zudem $\sigma$-Ordnungs vollständig ist, folgt die Behauptung.
% \end{proof}

\begin{defi}
Ein Banachverband $X$ heißt \index{Vektorraum!$KB$-Raum}\textbf{$KB$-Raum} (\textbf{Kantorovich-Banachraum}), falls jede monoton steigende, in der Norm beschränkte und positive Folge in $X$ in der Norm konvergent ist.
\end{defi}

% \begin{bem}
% Ist $X$ ein Banachverband mit einer nicht ordnungsstetigen Norm, so kann $X$ nicht $KB$-Raum sein.
% \end{bem}

% \begin{proof}
% \par
% Sei $(x_n)_{n\in\N}$ eine Folge in $X$. Dann impliziert $x_n\uparrow x$, dass $\|x_n\|\leq \|x\|$ für alle $n\in\N$ gilt. Damit ist nach \Cref{} für jeden $KB$-Raum die Norm bzgl. der Ordnung stetig. Insbesondere sind die Räume $l^\infty$ und $L^\infty(\Omega)$ nicht KB-Räume.
% \end{proof}

\begin{satz}
Sei $X$ ein Banachverband und für $X$ gelte die schwache Folgenvollständigkeit, d. h. jede Folge in $X$, die bzgl. der schwachen Konvergenz Cauchy ist, konvergiert in  $X$.  Ist $(x_n)_{n\in\N}$ eine monoton steigende Folge und $(\|x_n\|)_{n\in\N}$ beschränkt, dann gibt es $x\in X$ mit
\begin{equation*}
    \lim_{n\to\infty}x_n=x.
\end{equation*}
Insbesondere sind die reflexiven Banachverbände
%, welche stets der schwachen Folgenvollständigkeit genügen,
stets $KB$-Räume.
\end{satz}

\begin{proof}
Siehe \cite{banasiak_arlotti_2006}, Theorem 2.82.
\end{proof}

\begin{bsp}
Für alle $1<p<\infty$ sind die reflexiven Banachverbände $\ell^p$ und $\textnormal{L}^p(\Omega,\mu)$ % mit \Cref{Charakterisierung Reflexiver Banachverbände} 
KB-Räume.
\end{bsp}


\begin{satz}[\cite{banasiak_arlotti_2006}, Theorem 2.83]\label{Jeder $AL$-Raum ist ein $KB$-Raum}
Jeder $AL$-Raum ist bereits $KB$-Raum. 
\end{satz}

\begin{proof}
Sei $(x_n)_{n\in\N}$ eine monoton steigende, in der Norm beschränkte und positive Folge in $X$. Für  $0\leq x_n\leq x_m$ mit $n\leq m$ gilt
\begin{equation*}
\|x_m\| = \|x_m - x_n\| + \|x_n\|.
\end{equation*}
Mit $x_m-x_n\geq0$ können wir dies schreiben als
\begin{equation*}
\|x_m - x_n\| = \|x_m \| - \|x_n\| = \big|\|x_m\| - \|x_n\| \big |.    
\end{equation*}
Nach Voraussetzung ist die Folge $(\|x_n \|)_{n\in\N}$ monoton steigend und beschränkt, insbesondere konvergent. Wegen der Additivität der Norm ist somit $(x_n)_{n\in\N}$ eine Cauchyfolge und die Behauptung folgt.
\end{proof}

% \begin{bsp}
% \hl{Die Menge aller KB-Räume ist echt enthalten in der Menge aller Banachverbände mit ordnungsstetiger Norm.}
% \end{bsp}

% \begin{proof}
% \par
% Betrachte den Vektorraum $c_0$ aller Nullfolgen: \hl{Die Vollständigkeit bzgl. der $\sigma$-Ordnung ist klar}. Sei $(x_n)_{n\in\N}$ eine Folge in $c_0$ mit $x_n:=(x_k^n)_{k\in\N}$ und $x_n\downarrow 0$. Wähle $\epsilon >0$. Dann gibt es  $k_0\in\N$ mit $|x_k^1|<\epsilon$ für alle $k\geq k_0$. Da die Folge $(x_n)_{n\in\N}$ fallend ist, erhalten wir ebenso $|x_k^n|< \epsilon$ für alle $k\geq k_0$ und $n\geq1$. Es gibt also $n_0$ so, dass $|x_k^n|<\epsilon$ für alle $n\geq n_0$ und $1\leq k\leq  k_0$ gilt. Zusammen ist dann $\|x_n\|<\epsilon$ für alle $n\geq n_0$, und somit $\|x_n\|\to 0$. Mit \Cref{} ist damit die Norm von $c_0$ ordnungsstetig.

% \par
% Hingegen ist die monoton steigende und normbeschränkte Folge $(x_n)_{n\in\N}$ mit  $x_n:=(1,1,\dots,1,0,0,\dots)$, wobei die ersten $n$ Einträge jeweils $1$ seien, in $c_0$ nicht konvergent. Damit ist sie auch nicht $KB$-Raum, womit die Behauptung folgt.
% \end{proof}



% \begin{satz}[Ogasawara, \cite{aliprantis_burkinshaw_2006}, Theorem 4.70]\label{Charakterisierung Reflexiver Banachverbände}
% Sei $X$ ein Banachverband. Dann ist $X$ genau dann reflexiv, wenn sowohl $X$ als auch $X'$ KB-Räume sind.
% \end{satz}



% \begin{proof}
% \par
% Sie hierzu $(x_n)_{n\in\N}$ eine monoton steigende positive Folge in $X$ mit $\sup_{n\in\N}\|x_n\|<+\infty$. Dann gibt es $x''\in X''$ mit $x_n\uparrow x''$. Da $X$ reflexiv ist, erhalten wir $x''\in X$. \hl{Mit der Stetigkeit der Norm bzgl. der Ordnung ist somit auch $(x_n)_{n\in\N}$ normkonvergent}.
% \end{proof}


% \begin{defi}
% Eine lineare Abbildung $T\colon X\to Y$ zweier normierten Banachräume heißt \textbf{Einbettung}, falls es positive Konstanten $K$ und $M$ gibt mit
% \begin{equation*}
%     K\|x\|\leq \|T(x)\|\leq M\|x\|.
% \end{equation*}
% \par
% Ein Banachraum $X$ lässt sich in einen Banachraum $Y$ \textbf{einbetten}, falls eine Einbettung $T\colon X\to Y$ existiert. Falls $T$ zudem eine Homomorphismus von Banachverbänden ist, so bezeichne die Einbettung auch als \textbf{Verbandseinbettung}.
% \end{defi}


% \begin{bsp}
% Der Raum $c_0$ ist nicht $KB$-Raum (\cite{aliprantis_burkinshaw_2006}, Theorem 4.60).
% \end{bsp}

\begin{satz}
Für einen Banachverband $X$ sind äquivalent:
\begin{enumerate}
\item $X$ ist $KB$-Raum.
\item In $X$  gilt die schwache Folgenvollständigkeit.
\item Es gibt keine (Verbands-)Einbettung von $c_0$ in $X$.
\end{enumerate}
\end{satz}
\begin{proof}

Siehe \cite{aliprantis_burkinshaw_2006}, Theorem 4.60.
\end{proof}

% \begin{folg}[\cite{aliprantis_burkinshaw_2006}, Theorem 4.61]
% Ein Banachverband $X$ ist genau dann nicht $KB$-Raum, wenn eine Verbandseinbettung $T\colon c_0\to X$ existiert.
% \end{folg}


%\newpage\section{Positive Folgen in Banachverbänden}

% \begin{bem}
% Für parametrisierte Folgen  ist nachfolgende Aussage eine analoge Version  zu den Sätzen der dominierten und monotonen Konvergenz aus der Analysis.
% \end{bem}


% \begin{bem}
% \hl{Lorem ipsum dolor sit amet, consectetur adipiscing elit. Pellentesque vulputate pellentesque nunc, ut iaculis purus ornare nec. Nunc finibus rhoncus odio, non maximus tortor elementum eu. Duis rutrum tincidunt dignissim. Donec vel urna non felis congue iaculis. Aenean eu velit sagittis, tempor magna eu, volutpat urna.}
% \end{bem}

\newpage
\begin{satz}[Monotone und dominierte Konvergenz positiver Folgen in Banachverbänden]
\label{Positive Folgen in Banachverbänden}\index{Vektorraum!Monotone und dominierte Konvergenz}
Sei $T\subseteq \R$ und für alle $t\in T$ sei $(x_n(t))_{n\in\N}$ eine positive Folge in einem Banachverband $X$. Sei weiter $t_0\in T$. Dann gilt:
\begin{enumerate}
\item Ist für alle $n\in\N$ die Abbildung $t\mapsto x_n(t)$   monoton steigend und $\lim_{t\uparrow t_0}x_n(t) =  x_n$ in der Norm konvergent, dann gilt
\begin{equation*}\label{toll}
\lim_{t\uparrow t_0}\sum_{n=0}^\infty x_n(t) = \sum_{n=0}^\infty x_n.
\end{equation*}
Ist  $\sum_{n=0}^\infty x_n$ in $X$ nicht existent, falls also 
\begin{equation*}
    \|\sum_{n=0}^\infty x_n\|:=\sup_{N\in\N}\|\sum_{n=0}^N x_n\|=+ \infty
\end{equation*}
gilt, dann wird die obige Identität so verstanden, dass die  Norm auf beiden Seiten gleich unendlich ist.
\item Ist  $\lim_{t\to t_0} x_n(t)= x_n$ für alle $n\in\N$  in der Norm konvergent und existiert eine Folge $(a_n)_{n\in\N}$ in $X$, welche $\sum_{n=0}^\infty \|a_n \|<+\infty$ sowie $x_n(t)\leq a_n$ für alle $t\in T$ und $n\in \N$ erfüllt, dann gilt ebenfalls
\begin{equation*}
\lim_{t\uparrow t_0}\sum_{n=0}^\infty x_n(t) = \sum_{n=0}^\infty x_n.    
\end{equation*}
\end{enumerate}
\end{satz}

\begin{proof}
Siehe \cite{banasiak_arlotti_2006}, Theorem 2.91.
\end{proof}

% \begin{proof}
% Siehe \cite{banasiak_arlotti_2006}, Aussage 2.91.
% \end{proof}

% \begin{proof}
% \par
% Zu (1): Betrachte zunächst den Fall "$\sum_{n=0}^\infty x_n\in X$". Dann gilt für alle $t\in T$
% \begin{equation*}
% 0\leq\sum_{n=0}^\infty x_n(t)\leq\sum_{n=0}^\infty x_n.
% \end{equation*}
% Somit ist $\sum_{n=0}^\infty x_n(t)$ für alle $t\in T$ ebenfalls konvergent. Insbesondere gibt es für jedes $\epsilon>0$ ein $N\in\N$ so, dass für alle $t\in T$ gilt: 
% \begin{equation*}
%     \Big\|\sum_{n=N+1}^\infty x_n(t)\Big\|\leq \Big\|\sum_{n=N+1}^\infty x_n\Big\|\leq \frac{\epsilon}{3}.
% \end{equation*}
% Für jedes solches $N\in\N$ fest wiederum gibt es  $t'< t_0$ so, dass für alle  $n\leq N$ und $t'< t< t_0$ gilt:
% \begin{equation*}
%     \|x_n -x_n(t)\|\leq \frac{\epsilon}{3(N+1)}.
% \end{equation*}
% Mit der \textbf{Dreiecksungleichung} folgt somit für alle $t\in T$ mit $t'< t<t_0$
% \begin{equation*}
% \Big\|\sum_{n=0}^\infty x_n(t) - \sum_{n=0}^\infty x_n\Big \|\leq \epsilon.
% \end{equation*}

% \par
% Zu "$\sum_{n=0}^\infty x_n\not\in X$". Dann ist $\|\sum_{n=0}^\infty x_n\|=+\infty$. Ohne Einschränkung sei $\sum_{n=0}^\infty x_n(t)\in X$ für alle $t\in T$. Andernfalls ist die Identität wegen der Monotonie von $t\mapsto x_n(t)$ klar. Für alle $M\in\N$ gibt es ein $N\in\N$ mit
% \begin{equation*}
%     \Big\|\sum_{n=0}^N  x_n\Big\|\geq M+1.
% \end{equation*}
% Mit der \textbf{umgekehrten Dreiecksungleichung} gilt dann
% \begin{equation*}
% \Big\|\sum_{n=0}^N x_n(t)\Big\|= \Big \|\sum_{n=0}^N (x_n(t)- x_n) + \sum_{n=0}^N x_n \Big \| \geq \Bigg |\Big \|  \sum_{n=0}^N x_n\Big \| - \Big \| \sum_{n=0}^N (x_n (t) - x_n)\Big \|\Bigg |.
% \end{equation*}
% Wegen $\lim_{t\to t_0}x_n(t)=x_n$ können wir in obiger Abschätzung den zweiten Term in der Differenz für alle $N\in\N$ mit $t$ hinreichend nah an $t_0$ kleiner als $1/(N+1)$ machen. Dies ergibt
% \begin{equation*}
% \Big \|\sum_{n=0}^\infty x_n (t)\Big\|\geq \Big\|\sum_{n=0}^N x_n(t)\Big \|\geq M.
% \end{equation*}
% Da dies für alle $M\in\N$ gilt, folgt $\lim_{t\to t_0} \|\sum_{n=0}^\infty x_n (t) \| = \infty.$

% \par
% Zu (2): Da $\lim_{t\to t_0}x_n(t)=x_n$ mit $0\leq x_n(t)\leq a_n$ ist, folgt mit der Abgeschlossenheit des positiven Kegels (\Cref{Abgeschlossenheit des Kegels}), dass $x_n\leq a_n$ für alle $n\in\N$ gilt. Dies liefert
% \begin{align*}
% \Big\|\sum_{n=0}^\infty (x_n - x_n(t))\Big\| 
% &\leq \Big\|\sum_{n=0}^N (x_n - x_n(t))\Big\| + \Big\|\sum_{n=N+1}^\infty (x_n- x_n(t))\Big\|\\
% &\leq \Big\|\sum_{n=0}^N (x_n - x_n(t))\Big \|+  2 \Big \|\sum_{n=N+1}^\infty  a_n \Big \|.
% \end{align*}
% Sei $\epsilon >0$. Mit $\sum_{n=0}^\infty\| a_n\|< +\infty$ kann der zweite Summand für alle $N\in\N$ kleiner als $\epsilon/4$ gemacht werden. Für jedes $N$ fest kann wegen der komponentenweisen Konvergenz von $x_n\to x_n(t)$ der erste Summand kleiner als $\epsilon/2$ gemacht werden, womit die Behauptung folgt.
% \end{proof}

\begin{bem}\label{Positive Folgen in $KB$-Raum}
Ist $X$  $KB$-Raum und es gilt $\lim_{t\uparrow t_0}\sum_{n=0}^\infty x_n(t)\in X$, so liefert \Cref{Positive Folgen in Banachverbänden}, Aussage (1),  bereits die Konvergenz von $\sum_{n=0}^\infty x_n$.

% \hl{Für alle $t\in T\subseteq \R$ sei  $(x_n(t))_{n\in\N}$ eine Familie positiver Folgen in einem $KB$-Raum $X$. Angenommen, es ist $\lim_{t\uparrow t_0}\sum_{n=0}^\infty x_n(t)\in X$ für ein $t_0\in T$. Dann ist $\sum_{n=0}^\infty x_n$ konvergent.}
\end{bem}

% \begin{proof}
% \hl{Betrachte $N\mapsto \sum_{n=0}^N x_n$. Mit $x_n\geq 0$ ist die Folge monoton steigend, und wegen $X$ $KB$-Raum gilt somit entweder $\sum_{n=0}^\infty\in X$ oder $\|\sum_{n=0}^\infty x_n\| = \infty$. }
% \end{proof}

\chapter{Etwas Operatorentheorie}

In diesem Kapitel wiederholen wir einige Begriffe und Aussagen (positiver) linearer Operatoren und $C_0$-Halbgruppen. Hierbei ist der Begriff der Dissipativität linearer Operatoren zusammen mit der Charakterisierung von Kontraktionshalbgruppen nach \textbf{Lumer-Phillips} sowie die Approximation von $C_0$-Halbgruppen nach \textbf{Trotter-Kato} für den Beweis der Störungstheoreme des nachfolgenden Kapitels von zentraler Bedeutung (vgl. \cite{engel_nagel_2006}, \cite{werner_2007}, \cite{aliprantis_burkinshaw_2006}, \cite{pazy_1983}).

% Hierbei verwenden wir eine stärkere Variante (siehe \Cref{Trotter-Kato Bedingung (2)}) des Störungstheorems nach \textbf{Trotter-Kato} (\Cref{Trotter-Kato}). 

% \hl{Lorem ipsum dolor sit amet, consectetur adipiscing elit. Pellentesque vulputate pellentesque nunc, ut iaculis purus ornare nec. Nunc finibus rhoncus odio, non maximus tortor elementum eu. Duis rutrum tincidunt dignissim. Donec vel urna non felis congue iaculis. Aenean eu velit sagittis, tempor magna eu, volutpat urna.}



\section{Positive Operatoren}

\begin{verein}
Wir verwenden die in \cite{engel_nagel_2006} gebräuchliche Terminologie für die Bezeichnung von linearen Operatoren.
\end{verein}


% \newpage\section{Grundlegendes zu Operatoren}


% \begin{defi}
% Es seien $X$ und $Y$ Banachräume. 
% \par
% Ist $D(A)\subseteq X$ ein linearer Teilraum, so heißt eine lineare Abbildung $A\colon D(A)\to Y$  \index{Operator}\textbf{linearer Operator} (mit Definitionsbereich) auf $D(A)$, schreibe $\big(A, D(A)\big)$.
% \par
% Die Menge aller linearen Operatoren mit $D(A)=X$ wird mit $L(X, Y)$, bzw. $L(X):= L(X,Y)$, falls $Y=X$ gilt, bezeichnet. 
% %Für $A\in L(X)$ bezeichne $A$ (oder $\big(A, D(A)\big)$) auch als linearen Operator \textbf{in} $X$. 
% \par
% %Für die Menge aller skalarwertige linearen Operatoren setze $L(X)^*:= L(X,\R)$. 
% Es bezeichne $\mathcal L(X,Y)$ die Menge aller  $A\in L(X,Y)$ mit endlicher  \index{Operator!Operatornorm}\textbf{Operatornorm}, d. h. $\|A\|:=\sup_{\|x\|\leq 1}\|Ax\|<+\infty$ und  $A$ heißt \index{Operator!beschränkt}\textbf{beschränkt}. 
% %Mit der Operatornorm wird $\mathcal L(X,Y)$ zu einem Banachraum.
% \par
% Falls $\big(A, D(A)\big)$ ein linearer Operator auf $X$ und $Y\subseteq X$ eine Teilmenge ist, so ist die \index{Operator!Einschränkung}\textbf{Einschränkung}  von $A$ auf $Y$ durch die Vorschrift $A|_Yy:= Ay$ für alle $y\in D(A|_Y):=\{x\in D(A); x\in Y\}$  gegeben. Für den Fall $Y=X_+$ schreibe $D(A|_Y)=D(A)_+$. Für zwei lineare Operatoren $A, B$ bezeichne $B$ als eine \index{Operator!Fortsetzung}\textbf{Fortsetzung} von $A$, kurz $B\supseteq A$, falls $D(A)\subseteq D(B)$ und $B|_{D(A)}=A$ gilt.

% \par
% Die Teilräume  $\text{Bild}(A):= \{y\in Y; \exists x\in D(A): Ax = y\}$ und $\text{Kern}(A):=\{x\in D(A); Ax=0\}$ werden als \index{}\textbf{Bild} bzw. \index{Operator!Kern}\textbf{Kern} von $A$ bezeichnet. Der \index{Operator!Graph}\textbf{Graph} eines linearen Operators ist die Menge $G(A):=\{(x,y)\in X\times Y; x\in D(A), Ax= y\}$. Zusammen mit der \index{Operator!Graphennorm}\textbf{Graphennorm}  $\|\cdot\|_{D(A)}:=\|x\|_X + \|Ax\|_Y$ wird $D(A)$ zu einem normierten Vektorraum.

% \par
% \textbf{}Nenne $A\in  L(X, Y)$ \index{Operator!abgeschlossen}\textbf{abgeschlossen}, falls $G(A)$ eine abgeschlossene Teilmenge in $X\times Y$ ist. $\big(A, D(A)\big)$ ist genau dann abgeschlossen, wenn für jede Folge $(x_n)_{n\in\N}$ in $D(A)$ mit $\lim_{n\to\infty} x_n = x$ in $X$ und $\lim_{n\to\infty} Ax_n = y$ in $Y$ folgt, dass  $x\in D(A)$ und $y= Ax$ gilt. Insbesondere ist $\big(A, D(A)\big)$ bzgl. der Graphennorm $\|\cdot\|_{D(A)}$ stets abgeschlossen.

% \par
% Bezeichne $A$ als \index{Operator!abschließbar}\textbf{abschließbar}, falls der Abschluss $\overline{G(A)}$ der Graph eines linearen Operators ist. 
% %Dies ist genau dann der Fall, wenn $(0,y)\in\overline{G(A)}$ stets $y=0$ impliziert. 
% Ist $A$ abschließbar, so bezeichne den linearen Operator $\overline A$ mit Graphen $\overline {G(A)}$ als \index{Operator!Abschluss}\textbf{Abschluss} von $A$.

% % $\big(A, D(A)\big)$ ist genau dann beschränkt, wenn $(D(A), \|\cdot\|_{D(A)})$ ein Banachraum ist.
% \end{defi}

% \begin{bsp}
% Sei $X=L^p([0,1])$ und $f_n(x):=(np+1)^{1/p}$. Dann gilt $\|f_n\|=1$ und weiter:
% \begin{equation*}
%     \|f_n'\|=n\Big(\frac{np+1}{np+1-p}\Big)^{1/p}.
% \end{equation*}
% Damit ist $Tf = f'$ ein nicht-beschränkter Operator auf $D(T)=\{f\in X; Tf\in X\}$, der sog. \textbf{Differentialoperator}.
% \end{bsp}

% \begin{prop}\label{satz vom abgeschlossenen Graphen}
% Seien $X, Y$ Banachräume, $\big(A, D(A)\big)$ Operator. Ist $A$ abgeschlossen mit $D(A)=X$, dann ist $A$ stetig.
% \end{prop}

% \begin{proof}
% Setze $Y:=D(A)$. Da $\text{Bild}(A)\subseteq Y$ linearer Teilraum ist, haben wir $G(A)\subseteq X\times Y$ linearer Teilraum. Da $A$ abgeschlossen,  ist $G(A)$ abgeschlossen und insgesamt ist $G(A)$ Banachraum. Setze \begin{equation}\label{eq:}
% \pi_X\colon G(A)\to X,\quad \pi_Y\colon G(A)\to Y.
% \end{equation}
% Dann sind die Projektionen $\pi_X, \pi_Y$ stetig sowie $\pi_X$ bijektiv mit $D(A)=X$. Mithilfe \index{}\textbf{Banachs Homomorphiesatz} ist die Stetigkeit von $\pi_X^{-1}\colon X\to G(A)$. Dann folgt die Stetigkeit von $A$ wegen $A=\pi_Y\circ \pi_X^{-1}$.
% \end{proof}


% \begin{defi}
% Sei $(A_n)_{n\in\N}$ eine Folge von Operatoren in  $\mathcal L(X,Y)$.
% \begin{enumerate}
% \item  $(A_n)_{n\in\N}$ heißt \index{Operator!starke konvergent}\textbf{stark konvergent}, falls es $A\in\mathcal L(X,Y)$ gibt mit $\lim_{n\to\infty} A_nx = Ax$ für alle $x\in X$.
% \item $(A_n)_{n\in \N}$ heißt \index{Operator!normkonvergent}\textbf{(norm-)konvergent}, falls es $A\in\mathcal L(X,Y)$ gibt mit $\lim_{n\to\infty}\|A_n-A\|=0$.
% \end{enumerate}
% \end{defi}

% \begin{defi}
% Sei $B\subseteq \mathcal L(X, Y)$. Dann heißt
% \begin{enumerate}
% \item $B$ heißt \index{}\textbf{gleichmäßig beschränkt}, falls Menge $\{\|A\|; A\in B\}$ beschränkt ist.
% \item $B$ heißt  \index{}\textbf{punktweise beschränkt}, falls Menge $\{\|Ax\|;A\in B\}$ für alle $x\in X$ beschränkt ist.
% \end{enumerate}
% \end{defi}




% \begin{satz}[Banach-Steinhaus, \cite{werner_2007}, Theorem IV.2.1]\index{Banach-Steinhaus}\label{Banach-Steinhaus}
% Sei $X$ ein Banachraum und $Y$ ein normierter Vektorraum. Dann sind äquivalent:
% \begin{enumerate}
% \item $B\subseteq \mathcal L(X,Y)$ ist gleichmäßig beschränkt.
% \item $B\subseteq \mathcal L(X,Y)$ ist punktweise beschränkt.
% \end{enumerate}
% \end{satz}

% \begin{folg}\label{Banach-Steinhaus Folgerung 1}
% Eine Folge von Operatoren $(A_n)_{n\in\N}$ ist genau dann stark konvergent, wenn sie gleichmäßig konvergent auf kompakten Intervallen ist.
% \end{folg}

% \begin{proof}
% Hier ist lediglich die Rückrichtung zu zeigen: Ohne Einschränkung sei $A_n\to 0$ stark konvergent. Mit \textbf{Banach-Steinhaus}  ist dann $a:=\sup_{n\in\N}\|A_n\| <+ \infty$. Ist $\Omega\subseteq X$ eine kompakte Teilmenge, dann gibt es für alle $\epsilon>0 $ eine endliche Teilmenge $N_\epsilon =\{x_1,\dots,x_n\}$ so, dass für alle $x\in \Omega$ ein $x_i\in N_\epsilon$ existiert mit $\|x-x_i\|\leq \epsilon /2a$. Da $N_\epsilon$ endlich ist, gibt es ein $n_0\in\N$ so, dass für alle $n>n_0$ und $i=1,\dots, k$ die Abschätzung $\|A_nx_i\|\leq\epsilon /2$ erfüllt ist. Folglich gilt:
% \begin{equation*}
%     \|A_nx\|=\|A_nx_i\|+a\|x-x_i\|\leq \epsilon,\quad\forall x\in\Omega.
% \end{equation*}
% \end{proof}

% \begin{folg}\label{Banach-Steinhaus Folgerung 2}
% Seien $X,Y$ Banachräume und $(A_n)_{n\in\N}$ eine Folge in $\mathcal L(X,Y)$ mit $\sup_{n\in\N}\|A_n\|=: M$ für ein $M<+\infty$. Gibt es eine dichte Teilmenge $D\subseteq X$ so, dass $(A_nx)_{n\in\N}$ eine Cauchyfolge für alle $x\in D$ ist, dann konvergiert $(A_nx)_{n\in\N}$ für alle $x\in X$ gegen ein $A\in\mathcal L(X,Y)$.
% \end{folg}

% \begin{proof}
% Sei $\epsilon >0$ und $y\in X$. Dann gibt es $x\in D$ mit $\|x-y\|< \epsilon /M$. Hierfür wiederum gibt es $n_0\in\N$ so, dass $\|A_n x-A_mx\| <\epsilon$ für alle $n,m>n_0$ gilt. Mit der \textbf{Dreiecksungleichung} erhalten wir:
% \begin{equation*}
%     \|A_n y-A_my\|\leq \|A_nx-A_mx\|+\|A_n(x-y)\|+\|A_m(x-y)\|\leq 3\epsilon.
% \end{equation*}
% Also ist $(A_ny)_{n\in\N}$ für alle $y\in X$  eine Cauchyfolge in $X$, insbesondere konvergent. Die Behauptung folgt dann mit \textbf{Banach-Steinhaus}.
% \end{proof}

\begin{defi} Für einen beschränkten linearen Operator $A$ auf einem (komplexen) Banachraum $X$ definiere:
\begin{enumerate}
\item Die \index{Resolvente!Resolventenmenge}\textbf{Resolventenmenge}  von $A$ ist die Menge
\begin{equation*}
    \varrho(A):=\{\lambda \in\mathbb C; \lambda I-A \text{ ist invertierbar mit }(\lambda I-A)^{-1}\in \mathcal L(X)\}.
\end{equation*}
\item  Für alle $\lambda\in\varrho(A)$  ist $R(\lambda, A):=(\lambda I-A)^{-1}$ die \index{Resolvente}\textbf{Resolvente}  von $A$. 
\item Das \index{Resolvente!Spektrum}\textbf{Spektrum} von $A$ ist das Komplement $\sigma(A):=\mathbb C\setminus \varrho(A)$.
\item $r_\sigma (A):=\sup_{\lambda\in \sigma(A)} |\lambda |$ heißt \index{Resolvente!Spektralradius}\textbf{Spektralradius} von $A$.
\end{enumerate}
\end{defi}

% \begin{bem}
% Das Spektrum $\sigma(A)$ eines beschränkten linearen Operators $A$ kann wie folgt unterteilt werden:
% \begin{enumerate}
% \item Das \index{Resolvente!Punktspektrum}\textbf{Punktspektrum} $\sigma_p(A)$ ist die Menge aller $\lambda\in\sigma(A)$, für die $\lambda I-A$ nicht injektiv ist.
% \item Das \index{Resolvente!Residualspektrum}\textbf{Residualspektrum} $\sigma_r(A)$ ist die Menge aller $\lambda\in\sigma(A)$, für die $\lambda I-A$ injektiv ist, aber $\text{Bild}(\lambda I-A)$ nicht dicht in $X$ liegt.
% \item Das \index{Resolvente!approximatives Spektrum}\textbf{approximative Spektrum} $\sigma_a(A)$ ist die Menge aller $\lambda\in\sigma (A)$, für die $\lambda I-A$ injektiv ist und $\text{Bild}(\lambda I-A)$ dicht in $X$ liegt, aber $\overline{\text{Bild}(\lambda I-A)}\neq X$ gilt.
% \end{enumerate}
% \end{bem}

% \begin{defi}
% Sei $A$ ein unbeschränkter linearer Operator in einem Banachraum $X$. Dann ist die  \index{Resolvente!Spektralschranke}\textbf{Spektralschranke} von $A$ gegeben durch:
% \begin{equation*}
% s(A):=\sup\{\mathfrak R\lambda; \lambda\in\sigma(A)\}.
% \end{equation*}
% \end{defi}

\begin{bem}[Resolventengleichung]\index{Resolvente!Resolventengleichung}\label{Resolventengleichung}
Für einen linearen Operator $A$ auf einem Banachraum $X$ gilt stets
\begin{equation*}
R(\lambda, A)-R(\mu, A)=(\mu-\lambda)R(\lambda, A) R(\mu, A),\quad\forall \lambda,\mu\in \varrho(A).
\end{equation*}
\end{bem}



\begin{satz}[Neumann'sche Reihe]\label{Satz von der Neumann'schen Reihe}\index{Neumann'sche Reihe}
Sei $A\in \mathcal L(X)$ mit $\|A\| < 1$. Dann ist der lineare Operator $I-A$ invertierbar und die Inverse ist gegeben durch
\begin{equation*}
    (I-A)^{-1}=\sum_{k=0}^\infty A^k.
\end{equation*}
\end{satz}

\begin{proof}
Siehe  \cite{werner_2007}, Theorem II.1.11. 
\end{proof}

% \begin{satz}[Satz vom Abgeschlossenen Graphen, \cite{werner_2007}, IV.4.5]\label{Satz vom abgeschlossenen Graphen}\index{Satz vom abgeschlossenen Graphen}
% Es seien $X, Y$ Banachräume. Dann ist ein linearer Operator $A\in L(X,Y)$ mit $D(A)=X$ genau dann beschränkt, wenn $G(A)$ abgeschlossen ist.
% \end{satz}

 
% \newpage\section{Positive Operatoren}

\begin{defi}
Seien $X, Y$ Banachverbände. Ein linearer Operator $A\colon X\to Y$ heißt \index{Operator!positiv}\textbf{positiv}, schreibe $A\geq0$, falls gilt:
\begin{equation*}
    Ax\geq0,\quad\forall x\in X_+. 
\end{equation*}
Sind $A$ und $B$ zwei positive lineare Operatoren auf $X$, so schreibe \newline$A\leq B$, falls $B-A\geq0$ gilt.
\end{defi}

% \begin{bsp}
% Sei $X=\textnormal{L}^1(\Omega,\mu)$. Dann ist für jede positive messbare Funktion $k$ auf $\Omega$ der \textbf{Integraloperator} ein positiver Operator. Dieser ist punktweise gegeben durch:
% \begin{equation*}
% (Af)(x):=\int_\Omega k(x,y)f(y)\text dy,\quad\forall f\in \textnormal{L}^1(\Omega,\mu),\forall x\in \Omega.
% \end{equation*}
% \end{bsp}


\begin{bem}\label{Charakterisierung positiver Operatoren}
Ein linearer Operator $A$ auf einem (reellen) Banachverband $X$ ist genau dann positiv, wenn $|Ax|\leq A|x|$ für alle $x\geq0$ gilt.
\end{bem}

\begin{proof}
Sei $x\in X_+$. Dann gilt $Ax= A|x| \geq |Ax|\geq0$. Für $A$ positiv wiederum erhalten wir mit $-|x|\leq x\leq |x|$, dass $-A|x|\leq Ax\leq A|x|$ gilt, also ist $|Ax|\leq A|x|$ erfüllt. 
\end{proof}

\begin{satz}[Kantorovich]\label{Fortsetzung positiver Operatoren auf dem positiven Kegel}
Sei $A\colon X_+\to Y_+$  eine additive Abbildung von Banachverbänden, d. h. es gilt $A(x+y)=Ax+Ay$ für alle $x,y\in X_+$. Dann existiert eine eindeutig bestimmte additive Abbildung $\widetilde A$ auf ganz $X$, welche Fortsetzung von $A$ ist. Diese ist punktweise gegeben durch
\begin{equation*}
\widetilde Ax = Ax_{+} - Ax_{-},\quad\forall x\in X.
\end{equation*}
\end{satz}

\begin{proof}
Siehe \cite{aliprantis_burkinshaw_2006}, Theorem 1.10.
\end{proof}

\begin{satz}[\cite{banasiak_arlotti_2006}, Theorem 2.65]\label{Beschränktheit positiver Operatoren}
Sei $X$ ein Banachverband und  $Y$ ein normierter Vektorverband. Dann ist jeder positive lineare Operator  $A\colon X\to Y$ bereits beschränkt.
\end{satz}

\begin{proof}
Angenommen, der positive lineare Operator $A$ ist unbeschränkt. Dann gibt es eine beschränkte Folge $(x_n)_{n\in\N}$ in $X$ mit $\|x_n\|=1$ für alle $n\in\N$, welche $\|Ax_n\|\geq n^3$ für alle $n\in\N$ erfüllt. Da $X$ vollständig ist, gilt 
\begin{equation*}
    x:=\sum_{n=1}^\infty n^{-2}|x_n|\in X.
\end{equation*}
Insbesondere ist $0\leq n^{-2}|x_n|\leq x$. Mit \Cref{Charakterisierung positiver Operatoren} gilt somit für alle $n\in\N$
\begin{equation*}
n \leq \|A(n^{-2}x_n)\|= \||A(n^{-2}x_n)|\| \leq \|A(n^{-2}|x_n|)\|\leq \|A|x|\|<+\infty.
\end{equation*}
Dies ist ein  \textbf{Widerspruch}.% Also ist $A$ beschränkter Operator.
\end{proof}

\begin{prop}[\cite{banasiak_arlotti_2006}, Proposition 2.67]\label{Norm positiver Operatoren}
Sei $A$ ein positiver linearer Operator auf einem Banachverband $X$. Dann ist die Operatornorm von $A$ bereits gegeben durch
\begin{equation*}
\|A\|=\sup_{x\geq 0, \| x \| \leq 1} \|Ax\|.
\end{equation*}
\end{prop}

\begin{proof}
Zu "$\geq$": Die Abschätzung ist erfüllt, da das Supremum über eine kleinere Menge genommen wird.
%, also ist $\| A\| = \sup_{\|x\|\leq 1}\|Ax\|\geq\sup_{x\geq 0,\|x\|\leq 1}\|Ax\| = \|Ax\|$.

\par 
Zu "$\leq$": Sei $x\in X$ mit $\|x\|\leq 1$. Wegen  $|x| = x_+ + x_-\geq0$ gilt $\||x|\|=\|x\|\leq 1$. Da $A$ positiv ist, wissen wir, dass $A|x|\geq |Ax|$ und damit $\|A|x|\|\geq \||Ax|\|=\|Ax\|$ gilt. Somit erhalten wir
\begin{equation*}
    \sup_{\|x\|\leq 1}\|Ax\|\leq\sup_{\|x\|\leq 1}\|A|x|\|=\sup_{x\geq 0,\|x\|\leq 1}\|Ax\|
\end{equation*}
und die Behauptung folgt.
\end{proof}

\begin{bem}
Für lineare Operatoren $A, B$ von Banachverbänden gilt stets
\begin{equation*}
0\leq A\leq B\Rightarrow \|A\|\leq \|B\|. 
\end{equation*}
\end{bem}

% \begin{proof}
% Wegen $\|B-A\|\geq0$ erhalten wir mit der \textbf{Dreiecksungleichung}, dass $\|B\| + \|A\| \geq 0$ für alle $x\geq0$ gilt. Mit \Cref{Norm positiver Operatoren} gilt dies bereits für  alle $x\in X$.
% \end{proof}

\begin{bem}
Sei $A$ ein positiver linearer Operator auf einem Banachraum $X$. Gibt es ein $K<+\infty$ mit $\|Ax\|\leq K\|x\|$ für alle $x\geq0$, so gilt dies bereits für alle $x\in X$.
\end{bem}










% \begin{defi}
% Sei $X$ ein (reellwertiger) Banachraum. Eine Funktion $x^*\colon X\to \R$ heißt \textbf{Funktional}. Bezeichne die Menge $X^*:=\mathcal L(X,\R)$ aller stetigen linearen Funktionale als  \textbf{Dualraum} zu $X$. Weiter bezeichne mit $X^{**}:=(X^*)^*$ das \textbf{Bidual} zu $X$. 
% \par
% \hl{Wir können jedes $x\in X$ mit einem Element von  $X^{**}$ mithilfe der \textbf{dualen Paarung} identifizieren:}
% \begin{equation*}
% x(x^*):=\langle x^*, x\rangle:= x*(x).
% \end{equation*}
% \par
% \hl{Damit lässt sich $X$ als Teilraum von $X^{**}$ auffassen. Für gewöhnlich gilt $X\neq X^{**}$, andernfalls bezeichne $X$ als \textbf{reflexiv}.}
% \end{defi}

% \begin{bsp}
% Sei $1<p<+\infty$. Dann können wir das Dual zu $\textnormal{L}^p(\Omega,\mu)$ mit $L_q(\Omega)$ identifizieren, falls $1/p + 1/q = 1$ gilt. Die duale Paarung ist gegeben durch:
% \begin{equation*}
% \langle f,g\rangle:=\int_\Omega f(x)g(x)\text dx,\quad\forall f\in \textnormal{L}^p(\Omega,\mu),\forall g\in L_q(\Omega).
% \end{equation*}
% Insbesondere sind die Räume $\textnormal{L}^p(\Omega,\mu)$ für alle $1<p<\infty$ reflexiv. 
% \end{bsp}


 
\section{Positive $C_0$-Halbgruppen}

\begin{defi}
Eine Familie $\big(T(t)\big)_{t\geq0}$ von beschränkten linearen Operatoren auf einem Banachraum $X$  heißt \index{$C_0$-Halbgruppe}\textbf{$C_0$-Halbgruppe}, falls gilt:
\begin{enumerate}
\item $T(0)=I$.
\item $T(t+s)=T(t)T(s)$ für alle $t,s\geq0$.
\item $\lim_{t\downarrow 0}T(t)x=x$ für alle $x\in X$.
\end{enumerate}
\end{defi}
\begin{defi}
Eine Abbildung $A\colon D(A)\to X$ heißt \index{$C_0$-Halbgruppe!Generator}\textbf{Generator} einer $C_0$-Halbgruppe $\big(T(t)\big)_{t\geq0}$, falls gilt:
\begin{equation*}
Ax=\lim_{h\downarrow 0}\frac{T(h)x - x}{h},\quad \forall x\in D(A).
\end{equation*}
Hierbei ist $D(A)$ die Menge aller $x\in X$, für die $\lim_{h\downarrow 0} h^{-1}(T(h)x-x)$ existiert. Insbesondere ist $\big(A, D(A)\big)$  ein linearer Operator. 
\par
\end{defi}

\begin{verein}
Wir bezeichnen die  von einem Generator $\big(A, D(A)\big)$ erzeugte $C_0$-Halbgruppe auch mit $\big(T_A(t)\big)_{t\geq0}$.
\end{verein}


\begin{satz}\label{Darstellung einer Halbgruppe mithilfe der Resolvente}
Sei $\big(A, D(A)\big)$ Generator einer $C_0$-Halbgruppe  $\big(T(t)\big)_{t\geq0}$. Dann gilt
\begin{equation*}\label{Darstellung der Gruppe mithilfe der Resolvente}
T(t)x = \lim_{n\to\infty}\Big(I- \frac t n A\Big)^{-n}x = \lim_{n\to\infty} \Big(\frac n t R\Big(\frac n t, A\Big)\Big)^n x,\quad\forall x\in X, t\geq0.
\end{equation*}
% Hierbei gilt  auf allen kompakten Intervallen bereits gleichmäßige Konvergenz. 

\end{satz}

\begin{proof}
Siehe \cite{pazy_1983}, Theorem I.8.3.
\end{proof}

\begin{defi}
Eine $C_0$-Halbgruppe $\big(T(t)\big)_{t\geq0}$ heißt \index{$C_0$-Halbgruppe!Kontraktionshalbgruppe}\textbf{Kontraktionshalbgruppe}, falls gilt:
\begin{equation*}
\|T(t)x\|\leq\|x\|,\quad \forall x\in X,  t\geq0.
\end{equation*}
\end{defi}


% \begin{bsp}
% Sei $X=L^p(1)$ mit $I=\R$. Dann ist die \textbf{Translationshalbgruppe} gegeben durch:
% \begin{equation*}
% (T(t)f)(s):=f(t+s),\quad \forall f\in X,\quad \forall s,t\in I.
% \end{equation*}
% Die Halbgruppeneigenschaft für $T\colon I\to X$ ist klar. Für alle $t\geq0$ gilt:
% \begin{equation*}
% \|T(t)f\|_p^p=\int_I|f(t+s)|^p\text ds\leq\int_I|f(r)|^p \text dr =\|f\|_p^p.
% \end{equation*}
% Folglich genügt $\big(T(t)\big)_{t\geq0}$ der Abschätzung $\|T(t)\|\leq 1$.

% \par
% Zur starken Stetigkeit: Hierfür verwende \Cref{}. Sei dazu $\phi\in C_0^\infty(1)$. Dann ist $\phi$ gleichmäßig stetig (mit kompaktem Träger), folglich gibt es für jedes $\epsilon>0$ eine $\delta>0$ so, dass für alle $s\in I$ und $0<t<\delta$ gilt:
% \begin{equation*}
% |\phi(t+s)-\phi(s)|<\epsilon.
% \end{equation*}
% Damit sehen wir
% \begin{equation*}
% \int_I|\phi(t+s)-\phi(s)|^p\text ds\leq M_\phi \text{exp}(p).
% \end{equation*}
% \hl{Hierbei sei $M_\phi$ das Maß} [...]. Da $\mathbb C_0^\infty(1)$ für alle $1\leq p<\infty$ dicht in $L^p(1)$ liegt und $\big(T(t)\big)_{t\geq0}$ kontraktiv ist, liefert \Cref{}, die starke Stetigkeit von $\big(T(t)\big)_{t\geq0}$.
% \end{bsp}



\begin{bem}\label{Relative Beschränktheit von Halbgruppen}
Für eine $C_0$-Halbgruppe $\big(T(t)\big)_{t\geq0}$ gibt es stets Konstanten $M\geq 1$ und $\omega\in \R$ so, dass gilt:
\begin{equation*}\label{Abschätzung}
\|T(t)\|\leq M\textnormal{exp}(\omega t),\quad t\geq0.
\end{equation*}
\end{bem}

\begin{proof}
Siehe \cite{engel_nagel_2006}, Proposition I.1.4.
\end{proof}

% \begin{verein}
% \hl{Schreibe $A\in \mathcal G(M,\omega)$, falls $\big(A, D(A)\big)$ Generator einer $C_0$-Halbgruppe $\big(T_A(t)\big)_{t\geq0}$ ist, welcher der Abschätzung} in \Cref{Relative Beschränktheit von Halbgruppen} für Konstanten $M\geq 1$ und $\omega\in \R$ genügt.
% \end{verein}




% \begin{prop}[\cite{engel_nagel_2006}, Proposition ???]Für jede$C_0$-Halbgruppe $\big(T_A(t)\big)_{t\geq0}$ gilt:
% \begin{enumerate}
% \item Für alle $x\in X$ ist $\lim_{h\to 0}\frac{1}{h}\int_t^{t+h} T_A(s)x\textnormal ds = T_A(t)x$.
% \item [...]
% \end{enumerate}
% \end{prop}

\begin{satz}[Hille-Yosida]\label{Hille-Yosida}\index{Hille-Yosida}
Sei $\big(A, D(A)\big)$ linearer Operator auf einem Banachraum $X$ und es seien $\omega\in\R$ und $M\geq1$ Konstanten. Dann ist $\big(A, D(A)\big)$ genau dann Generator einer $C_0$-Halbgruppe $\big(T(t))_{t\geq0}$ mit 
\begin{equation*}
    \|T(t)\|\leq M\textnormal{exp}(\omega t),\quad t\geq0,
\end{equation*}
 wenn folgende Aussagen erfüllt sind:
\begin{enumerate}
\item $\big(A, D(A)\big)$ ist dicht definiert (und abgeschlossen).
\item Für $M\geq 1$ und  $\omega \in\mathbb R$ gilt $(\omega, \infty)\subseteq \varrho(A)$ mit
\begin{equation*}
\|(\lambda I- A)^{-n}\|\leq\frac{M}{ (\lambda - \omega)^{-n}},\quad\forall n\in\N, \lambda > \omega.
\end{equation*}
\end{enumerate}
\end{satz}

\begin{proof}
Siehe \cite{engel_nagel_2006}, Theorem II.3.8.
\end{proof}
\begin{prop}[Integraldarstellung der Resolvente]\label{Integraldarstellung der Resolvente}
Sei $\big(A, D(A)\big)$ Generator einer $C_0$-Halbgruppe $\big(T_A(t)\big)_{t\geq0}$  mit $(\omega,\infty)\subseteq\varrho(A)$ für ein $\omega\in\R$. Dann gilt
\begin{equation*}
R(\lambda, A)x = \int_0^\infty \exp(-\lambda t)T_A(t)x\textnormal dt,\quad  \forall x\in X, \mathfrak R \lambda >\omega.
\end{equation*}
\end{prop}

\begin{proof}
Siehe  \cite{engel_nagel_2006}, Theorem II.1.10.
\end{proof}




% \begin{bsp}
% Ist $A$ Generator einer $C_0$-Halbgruppe $\big(T_A(t)\big)_{t\geq0}$, so auch  $B:=aA+b$ mit $a>0$ und $b\in\mathbb C$ mit:
% \begin{equation*}
% R(\lambda, B)=\frac{1}{a}R\Big(\frac{\lambda-b}{a}, A\Big).
% \end{equation*}
% \end{bsp}

\begin{defi}
Sei $X$ ein Banachverband. Dann bezeichne eine $C_0$-Halbgruppe $\big(T(t)\big)_{t\geq0}$ in $X$ als \index{$C_0$-Halbgruppe!positiv}\textbf{positiv}, falls gilt:
\begin{equation*}
T(t)x\geq0, \quad\forall x\in X_+, t\geq 0.
\end{equation*}
\end{defi}

\begin{defi}
Ein linearer Operator $\big(A, D(A)\big)$ auf einem Banachverband $X$ heißt \index{Resolvente!resolventenpositiv}\textbf{resolventenpositiv}, falls es $\omega\in\R$ mit $(\omega,\infty)\subseteq \varrho(A)$ gibt so, dass gilt:
\begin{equation*}
R(\lambda, A)\geq 0, \quad \lambda >  \omega.
\end{equation*}
\end{defi}

\begin{satz}[\cite{engel_nagel_2006}, Theorem VI.1.3]\label{Charakterisierung Positiver Halbgruppen}%\index{$C_0$-Halbgruppe!Charakterisierung positiver $C_0$-Halbgruppen}
Eine $C_0$-Halbgruppe $\big(T_A(t)\big)_{t\geq0}$ ist genau dann positiv, wenn  der Generator $\big(A, D(A)\big)$ resolventenpositiv ist.
\end{satz}

\begin{proof}
Sei $\big(T_A(t)\big)_{t\geq0}$ positiv. Dann liefert die \textbf{Integraldarstellung der Resolvente} (\Cref{Integraldarstellung der Resolvente}) die Positivität von $R(\lambda, A)$.

\par
Ist hingegen $\big(A, D(A)\big)$ resolventenpositiv, dann liefert \Cref{Darstellung einer Halbgruppe mithilfe der Resolvente}
\begin{equation*}
    T_A(t)x=\lim_{n\to\infty}(nt^{-1}R(nt^{-1}, A))^nx,\quad\forall x\in X, t\geq0.
\end{equation*}
Insbesondere ist mit der Positivität von $R(\lambda, A)$ für alle $\lambda$ hinreichend groß auch  $T_A(t)\geq0$ für alle $t\geq0$.
\end{proof}

% \begin{satz}[\cite{banasiak_arlotti_2006}, 3.34]
% Sei $\big(T_A(t)\big)_{t\geq0}$ eine positive $C_0$-Halbgruppe auf einem Banachverband $X$ mit Generator $\big(A, D(A)\big)$. Dann gilt für alle $\lambda\in\mathbb C$ mit $\mathfrak R\lambda > s(A)$:
% \begin{equation*}
% R(\lambda, A)x=\int_0^\infty \textnormal{exp}(-\lambda t)T_A(t)x\textnormal dt,\quad\forall x\in X.
% \end{equation*}
% Weiter gilt:
% \begin{enumerate}
% \item Es ist entweder $s(A)=-\infty$ oder $s(A)\in\sigma(A)$.
% \item Sei $\lambda \in \varrho(A)$. Dann gilt $R(\lambda, A)\geq0$ genau dann, wenn $\lambda > s(A)$.
% \item Für alle $\mathfrak R\lambda >s(A)$ und $x\in X$ gilt $|R(\lambda, A)x|\leq R(\mathfrak \lambda, A)|x|$.
% \end{enumerate}
% \end{satz}


% \begin{bsp}
% \hl{Betrachte die positive  \hl{Translationshalbgruppe} $\big(T(t)\big)_{t\geq0}$ auf $[0,1]$ aus. Für $t>1$ gilt dann $T(t)f=0$ für alle $f\in X$ und damit  ist  \hl{$\omega_1(T)=-\infty$}. Folglich ist $s(A)=-\infty$ und $\sigma(A)=\emptyset$.}
% \end{bsp}


% \newpage\section{Adjungierte Operatoren}

% \begin{defi}
% Seien $X,Y$ Banachräume sowie $A\colon X\to Y$ beschränkter Operator. Dann der zu $A$ \textbf{adjungierte Operator} $A^*\colon Y^*\to X^*$ gegeben durch:
% \begin{equation*}
% (A^*y^*)(x)=y^*(Ax),\quad\forall y^*\in Y^*,\forall x\in X.
% %\langle y^*, Ax\rangle =\langle A^*y*, x\rangle
% \end{equation*}
% \end{defi}

% \begin{bem}
% Obige Vorschrift wird auch mit $\langle y^*, Ax\rangle =  \langle A^* y^*, x\rangle$ ausgedrückt. Weiter gilt $A^*\in \mathcal L(Y^*, X^*)$ mit $\|A^*\| = \|A\|$.
% \end{bem}

\newpage 
\section{Dissipative Operatoren}


\begin{satz}[Hahn-Banach]\label{Hahn-Banach}\index{Hahn-Banach}
Sei $X$ ein normierter Vektorraum und $U\subseteq X$ ein linearer Teilraum. Dann gibt es zu jedem stetigen linearen Funktional $u'$ auf $U$ ein stetiges lineares Funktional $x'$ auf $X$ so, dass gilt:
\begin{equation*}
x'|_U = u'\quad\text{ und }\quad \|x'\|=\|u'\|.
\end{equation*}
\end{satz}

\begin{proof}
Siehe  \cite{werner_2007}, Theorem III.1.5.
\end{proof}


% \begin{bem}
% für einen banachraum $x$ bezeichne mit $x'$ den zugehörigen (stetigen) dualraum. mit $\Cref{}$
% \end{bem}

\begin{defi}
Sei $X$ ein Banachraum. Die \index{Dualitätsmenge}\textbf{Dualitätsmenge} von $x\in X$ ist gegeben durch
\begin{equation*}
\mathcal J(x):=\{x'\in X'; \langle x', x\rangle=\|x\|^2 = \|x'\|^2\}.%\subseteq X'.
\end{equation*}
\end{defi}

\begin{bem}
Mit  \textbf{Hahn-Banach} ist $\mathcal J(x)\neq \emptyset$ für alle $x\in X$.
\end{bem}

\begin{defi}
Ein linearer Operator $\big(A, D(A)\big)$ auf einem (reellen) Banachraum $X$ heißt \index{Operator!dissipativ}\textbf{dissipativ}, falls für alle $x\in D(A)$ ein $x'\in \mathcal J(x)$ existiert so, dass gilt:
\begin{equation*}
%\mathfrak R
\langle Ax, x'\rangle \leq 0.
\end{equation*}
\end{defi}

\begin{prop}\label{Charakterisierung Dissipativer Operatoren}
Ein linearer Operator $\big(A, D(A)\big)$ ist genau dann dissipativ, wenn gilt:
\begin{equation*}
    \|(\lambda I- A)x\|\geq \lambda \|x\|,\quad \forall x\in D(A), \mathfrak \lambda >0.
\end{equation*}
\end{prop} 

\begin{proof}
Siehe \cite{pazy_1983}, Theorem I.4.2.
\end{proof}

\begin{prop}\label{Eigenschaften Dissipativer Operatoren}
Sei $\big(A, D(A)\big)$ ein linearer Operator. Ist $A$ dissipativ, dann gelten die Eigenschaften:
\begin{enumerate}
\item Für alle $\lambda >0$ ist $\lambda I- A$ injektiv und es gilt:
\begin{equation*}
    \|(\lambda I- A)^{-1}x\|\leq\frac{1}{\lambda}\|x\|,\quad\forall x\in\textnormal{Bild}(\lambda I- A).
\end{equation*}
\item Es gilt $\textnormal{Bild}(\lambda I- A)=X$ für ein $\lambda >0$ genau dann, wenn $\textnormal{Bild}(\lambda I- A)=X$ für alle $\lambda >0$ gilt.
\item $A$ ist abgeschlossen genau dann, wenn $\textnormal{Bild}(\lambda I-A)$ abgeschlossen für ein (und damit für alle) $\lambda >0$ ist.
\item Ist $A$ dicht definiert, dann ist $A$ abschließbar. Der Abschluss $\overline{A}$ ist ebenfalls dissipativ und es gilt $\textnormal{Bild}(\lambda I- \overline{A})=\overline{\textnormal{Bild}(\lambda I-A)}$.
\end{enumerate}
\end{prop}

\begin{proof}
Siehe \cite{engel_nagel_2006}, Proposition III.3.14.
\end{proof}

\begin{satz}[Lumer-Phillips]\index{Lumer-Phillips}\label{Lumer-Phillips}
Sei $\big(A, D(A)\big)$ ein linearer Operator auf einem Banachraum $X$. Ist $A$ dicht definiert und dissipativ, dann sind äquivalent:
\begin{enumerate}
\item Der Abschluss $\overline{A}$ ist Generator einer Kontraktionshalbgruppe in $X$.
\item $\textnormal{Bild}(\lambda I- A)$ liegt für ein (und damit für alle) $\lambda>0$ dicht in $X$.
\end{enumerate}
% \hl{In jedem Falle ist $\mathfrak R\langle x', Ax\rangle\leq 0$ bereits für alle $x'\in\mathcal J(x)$ erfüllt.}
\end{satz}

\begin{proof}
Siehe  \cite{engel_nagel_2006}, Theorem III.3.15.
\end{proof}

% \begin{bsp}
% Der Differentialoperator $T_1$ aus \Cref{} ist dicht definiert und mit \Cref{} dissipativ. Folglich ist $T_1$ Generator einer Kontraktionshalbgruppe in $L^p([0,1])$. [...]
% \end{bsp}

% \begin{bsp}
% Sei $\big(A, D(A)\big)$ Operator mit $\overline{D(A)}=X$. Ist sowohl $A$ als auch die Adjungierte $A'$ dissipativ, dann ist $\overline A$ Generator einer Kontraktionshalbgruppe in $X$: [...]
% \end{bsp}

\newpage
\section{Approximation von $C_0$-Halbgruppen}

\begin{defi}
 Sei $X$ ein Banachraum und $\Lambda\subseteq\mathbb C$ eine Teilmenge. Eine Familie $\{J(\lambda); \lambda\in\Lambda\}$ beschränkter linearer Operatoren in $X$ heißt \index{Resolvente!Pseudoresolvente}\textbf{Pseudoresolvente} auf $\Lambda$, falls gilt:
\begin{equation*}
J(\lambda)-J(\mu)=(\mu-\lambda)J(\lambda)J(\mu),\quad\forall\lambda,\mu\in\Lambda.
\end{equation*}
\end{defi}

\begin{prop}\label{Charakterisierung Pseudoresolvente}
Sei $\{J(\lambda); \lambda\in\Lambda\}$ eine Pseudoresolvente. Dann gilt
\begin{enumerate}
\item Es sind $\textnormal{Bild}(J(\lambda))$ und $\textnormal{Kern}(J(\lambda))$ von $\lambda\in\Lambda$ unabhängig.
\item $J(\lambda)$ ist die Resolvente eines eindeutig bestimmten, abgeschlossenen und dicht definierten linearen Operator $\big(A, D(A)\big)$ in $X$ genau dann, wenn $\textnormal{Kern}(J(\lambda))=\{0\}$  gilt und $\textnormal{Bild}(J(\lambda))$ dicht in $X$ liegt. 
\end{enumerate}
\end{prop}

\begin{proof}
Siehe \cite{engel_nagel_2006}, Proposition 1.6.
\end{proof}

% \begin{proof}
% Zu (1): Für alle $\lambda, \mu\in\Lambda$ gilt
% \begin{equation*}
%     J(\lambda)=J(\mu)\big(I+(\mu-\lambda)J(\lambda)\big).
% \end{equation*}
% Damit erhalten wir die Inklusion $\text{Bild}(J(\lambda))\subseteq \text{Bild}(J(\mu))$. Da wir $\lambda,\mu$ in obiger Identität vertauschen können, gilt schon $\text{Bild}(J(\lambda))= \text{Bild}(J(\mu))$.
% %, womit die Menge $\textnormal{Bild}(J(\lambda))$ unabhängig von der Wahl  $\lambda\in\Lambda$.
% \par
% Für $\lambda, \mu\in\Lambda$ betrachte
% \begin{equation*}
%     J(\lambda)=\big(I+(\mu-\lambda)J(\lambda)\big)J(\mu).
% \end{equation*}
% Dies liefert $\text{Kern}(J(\lambda))\supseteq \text{Kern}(J(\mu))$ und mit dem Vertauschen von $\lambda,\mu$ erhalten wir $\text{Kern}(J(\lambda))= \text{Kern}(J(\mu))$. 
% % Damit sind  $\textnormal{Bild}(J(\lambda))$ und $\textnormal{Kern}(J(\lambda))$ nicht von der Wahl  $\lambda\in\Lambda$ abhängig.

% \par
% Zu (2): Hier ist lediglich die Rückrichtung zu zeigen.
% % Zur Hinrichtung: Ist $J(\lambda)$ Resolvente, so liefert die Bijektivität insbesondere $\textnormal{Kern}(J(\lambda))=\{0\}$ und $\overline{\textnormal{Bild}(J(\lambda))}=X$.
% Sei $\text{Kern}(J(\lambda))=\{0\}$ und $\overline{\text{Bild}(J(\lambda))}=X$. Dann ist $J(\lambda)$ für alle $\lambda\in\Lambda$ injektiv. Für ein $\lambda_0\in\Lambda$ erhalten wir so die wohldefinierte Abbildung $D(A)=\text{Bild}(J(\lambda_0))$ vermöge
% \begin{equation*}
%     A:=\lambda_0I - J(\lambda_0)^{-1}.
% \end{equation*}
% Nach Konstruktion ist dann $A$ ein abgeschlossener, dicht definierter linearer Operator mit $D(A)=\text{Bild}(J(\lambda_0))$.  Weiter gilt
% \begin{equation*}
%     R(\lambda_0, A) = J(\lambda_0).
% \end{equation*}
% Sei nun $\lambda\in\Lambda$. Dann gilt
% \begin{align*}
% (\lambda I- A)J(\lambda)
% &=\big((\lambda-\lambda_0)I+(\lambda_0-A)\big)J(\lambda)\\
% &=\big((\lambda-\lambda_0)I+(\lambda_0-A)\big)J(\lambda_0)\big(I+(\lambda_0-\lambda)J(\lambda)\big)\\
% &=(\lambda-\lambda_0)J(\lambda_0)\big(I+(\lambda_0-\lambda)J(\lambda)\big)+I+(\lambda_0-\lambda)J(\lambda)\\
% &=I + (\lambda-\lambda_0)\big(J(\lambda_0)- J(\lambda)- (\lambda-\lambda_0)J(\lambda)J(\lambda_0)\big)=I.
% \end{align*}
% Umgekehrt erhalten wir
% \begin{align*}
% J(\lambda)(\lambda I- A)
% &=\big(I+(\lambda_0-\lambda)J(\lambda)\big)J(\lambda_0)\big((\lambda-\lambda_0)I+(\lambda_0 I- A)\big)\\
% &= \big(I+(\lambda_0 - \lambda)J(\lambda)\big)\big((\lambda-\lambda_0)J(\lambda_0)+I\big)\\
% &=I+(\lambda_0-\lambda)\big(-J(\lambda_0)+J(\lambda)+(\lambda-\lambda_0)J(\lambda)J(\lambda_0)\big)=I.
% \end{align*}
% Also ist $J(\lambda)=R(\lambda, A)$ für alle $\lambda\in\Lambda$. Insbesondere ist $A$ unabhängig von $\lambda$ und durch $J(\lambda)$ eindeutig bestimmt. 
% \end{proof}

\begin{folg}\label{Bedingung wenn Pseudoresolvente Resolvente ist}
Sei $\Lambda\subseteq \mathbb C$ eine unbeschränkte Teilmenge und $J(\lambda)$ eine Pseudoresolvente auf $\Lambda$ in $X$. Angenommen, es gibt eine Folge $(\lambda_n)_{n\in\N}$ in $\Lambda$ mit $|\lambda_n|\to\infty$ so, dass eine der beiden Aussagen gilt:
\begin{enumerate}
\item Es gilt $\sup_{n\in\N}\|\lambda_nJ(\lambda_n)\|=: M$ für ein $M<+\infty$ und $\textnormal{Bild}(J(\lambda))$ liegt dicht in $X$.
\item Für alle $x\in X$ gilt $\lim_{n\to\infty}\lambda_n J(\lambda_n)x=x$. 
\end{enumerate}
Dann ist $J(\lambda)$ die Resolvente eines eindeutig bestimmten, dicht definierten und abgeschlossenen linearen Operators $\big(A, D(A)\big)$ auf $X$.
\end{folg}

\begin{proof}
Siehe \cite{engel_nagel_2006}, Korollar 1.7.
\end{proof}

% \begin{proof}
% Mit (1): Mit \Cref{Charakterisierung Pseudoresolvente} ist nur "$\text{Kern}(J(\lambda))=\{0\}$"\; zu zeigen. 

% \par
% Da $\|\lambda_nJ(\lambda_n)\|\leq M$ für $n\to\infty$ gilt, erhalten wir wegen  $|\lambda_n|\to\infty$ die Konvergenz
% \begin{equation*}
%     \lim_{n\to\infty}\|J(\lambda_n)\|= 0.
% \end{equation*}
% Nach Definition von $J(\lambda)$ gilt für alle $\mu\in\Lambda$
% \begin{equation*}
%     J(\lambda_n)-\mu J(\mu)J(\lambda_n)=J(\mu)-\lambda_n J(\lambda_n)J(\mu),\quad\forall n\in\N.
% \end{equation*}
% Für $\mu\in\Lambda$ fest erhalten wir damit
% \begin{equation*}
%     \lim_{n\to\infty}\|(\lambda_n J(\lambda_n)-I)J(\mu)\|=0.
% \end{equation*}
% Dies liefert für alle $x\in \text{Bild}(J(\mu))$ dann
% \begin{equation*}
%     \lim_{n\to\infty}\lambda_n J(\lambda_n)x=x.
% \end{equation*}
% Da $\text{Bild}(J(\mu))$ dicht in $X$ liegt und  $\{\lambda_n J(\lambda_n)\}_{n\in\N}$ beschränkt ist, liefert \Cref{Banach-Steinhaus Folgerung 2} die Konvergenz auf ganz $X$. Sei nun $x\in \text{Kern}(J(\mu))$. Da  $\text{Kern}(J(\lambda))$ nach \Cref{Charakterisierung Pseudoresolvente}, Aussage (1), unabhängig von $\lambda$ ist, wissen wir, dass
% \begin{equation*}
%     \lambda_n J(\lambda_n)x= 0,\quad\forall n\in\N,n\in\N.
% \end{equation*}
% Also ist $x=\lim_{n\to\infty}\lambda_n J(\lambda_n)x=0$.

% \par
% Mit (2): Wir zeigen, dass die Voraussetzungen für \Cref{Bedingung wenn Pseudoresolvente Resolvente ist}, Aussage (1), erfüllt sind.
% \par
% Da mit \Cref{Charakterisierung Pseudoresolvente}, Aussage (1), die Teilmenge $\text{Bild}(J(\mu))$ unabhängig von $\mu$ ist, gilt für in festes, beliebiges $\mu\in\Lambda$
% \begin{equation*}
%     \lambda_n J(\lambda_n)x\in \text{Bild}(J(\mu)),\quad\forall x\in X,n\in\N.
% \end{equation*}
% Nach Annahme ist $ \lim_{n\to\infty}\lambda_n J(\lambda_n)x=x$ für alle $x\in X$. Damit erhalten wir  $\overline{\text{Bild}(J(\mu))}=X$. Weiter liefert die Konvergenz die punktweise Beschränktheit von $\{\lambda_n J(\lambda_n)\}_{n\in\N}$. Mit \textbf{Banach-Steinhaus} (\Cref{Banach-Steinhaus}) gibt es damit ein $M<+\infty$ mit  \begin{equation*}
%     \|\lambda_nJ(\lambda_n)\|\leq M,\quad\forall n\in\N.
% \end{equation*}
% \end{proof}

 
\begin{satz}[Trotter-Kato]\label{Trotter-Kato}\index{Trotter-Kato}
Für alle $n\in\N$ sei $\big(A_n, D(A_n)\big)$ Generator einer $C_0$-Halbgruppe $\big(T_n(t)\big)_{t\geq0}$ auf einem Banachraum $X$  mit 
% $\big(A, D(A)\big)$ linearer Operator auf einem Banachraum $X$ und es seien $\omega\in\R$ und $M\geq1$ Konstanten. Dann ist $\big(A, D(A)\big)$ genau dann Generator einer $C_0$-Halbgruppe $\big(T(t))_{t\geq0}$ mit 
\begin{equation*}
    \|T_n(t)\|\leq M\textnormal{exp}(\omega t),\quad t\geq0,\forall n\in \N,
\end{equation*}
für Konstanten $\omega\in\R$ und $M\geq1$.
%  wenn folgende Aussagen erfüllt sind:
% Sei $A_n\in \mathcal G(M,\omega)$ eine Folge von Generatoren auf einem Banachraum $X$. 
Angenommen, es gibt  $\lambda_0\in\mathbb C$ mit $\mathfrak R\lambda_0 >\omega$ so, dass gilt:
\begin{enumerate}
\item Für alle $x\in X$ existiert der Grenzwert $\lim_{n\to\infty}R(\lambda_0, A_n)x =:R(\lambda_0)x$.
\item Es liegt $\textnormal{Bild}(R(\lambda_0))$ dicht in $X$.
\end{enumerate}
Dann existiert ein eindeutig bestimmter Generator $\big(A, D(A)\big)$ mit
\begin{equation*}
R(\lambda_0, A)x=R(\lambda_0)x,\quad\forall x\in X.
\end{equation*}
% Für $n\in\N$ seien $\big(T_n(t)\big)_{t\geq0}$ die von $A_n$ sowie 
Sei $\big(T_A(t)\big)_{t\geq0}$ die von $\big(A, D(A)\big)$ erzeugte $C_0$-Halbgruppe. Dann ist die Folge $(T_n(t)x)_{n\in\N}$ für alle $x\in X$  auf kompakten Intervallen gleichmäßig konvergent und es gilt
\begin{equation*}
\lim_{n\to\infty}T_n(t)x=T_A(t)x,\quad \forall x\in X, t\geq0.
\end{equation*}
\end{satz}



% \begin{proof}
% Wir nehmen $\omega=0$ an. Zeige zunächst, dass die Konvergenz für alle $\lambda$ mit $\text{Re}(\lambda)>0$ existiert. Sei dazu $S$ die Menge alle $\lambda$, für die $(R(\lambda, A_n)x)_{n\in\N}$ konvergiert. Wähle $\mu\in S$ und stelle $R(\lambda, A_n)$ mithilfe der \index{}\textbf{Taylor-Reihe} um $\mu$ dar, also
% \begin{equation}
% R(\lambda, A_n)=\sum_{k=0}^\infty (\mu-\lambda)^k R(\mu, A_n){k+1}.
% \end{equation}
% Wegen $A_n\in\mathcal G(M,\omega)$ ist weiter
% \begin{equation}
% \|R(\mu, A_n)^k\|\leq M\text{Re}(\mu){-k},
% \end{equation}
% also konvergiert die Reihe für alle $\lambda$ mit $|\lambda-\mu|
% \end{proof}

\begin{proof}
Siehe \cite{pazy_1983}, Theorem III.4.4.
\end{proof}

\begin{bem}
Ist die Folge $(R(\lambda_0, A_n)x)_{n\in\N}$ für ein  $\lambda_0\in\mathbb C$ mit $\mathfrak R\lambda_0 >\omega$ konvergent, so gilt dies bereits für alle $\mathfrak R\lambda >\omega$.
\end{bem}


% \begin{bem}
% \hl{Lorem ipsum dolor sit amet, consectetur adipiscing elit. Pellentesque vulputate pellentesque nunc, ut iaculis purus ornare nec. Nunc finibus rhoncus odio, non maximus tortor elementum eu. Duis rutrum tincidunt dignissim. Donec vel urna non felis congue iaculis. Aenean eu velit sagittis, tempor magna eu, volutpat urna.}
% \end{bem}
\newpage
\begin{folg}[\cite{kato_1995}, Theorem IX.2.17]\label{Trotter-Kato Bedingung (2)}
Für alle $n\in\N$ sei $\big(A_n, D(A_n)\big)$ Generator einer $C_0$-Halbgruppe $\big(T_n(t)\big)_{t\geq0}$ auf einem Banachraum $X$  mit 
% $\big(A, D(A)\big)$ linearer Operator auf einem Banachraum $X$ und es seien $\omega\in\R$ und $M\geq1$ Konstanten. Dann ist $\big(A, D(A)\big)$ genau dann Generator einer $C_0$-Halbgruppe $\big(T(t))_{t\geq0}$ mit 
\begin{equation*}
    \|T_n(t)\|\leq M\textnormal{exp}(\omega t),\quad t\geq0,\forall n\in \N,
\end{equation*}
für Konstanten $\omega\in\R$ und $M\geq1$.
%  wenn folgende Aussagen erfüllt sind:
% Sei $A_n\in \mathcal G(M,\omega)$ eine Folge von Generatoren auf einem Banachraum $X$. 
Angenommen, es gibt  $\lambda_0\in\mathbb C$ mit $\mathfrak R\lambda_0 >\omega$ so, dass gilt:
\begin{enumerate}
\item Für alle $x\in X$ gilt $\lim_{n\to\infty}R(\lambda_0, A_n)x =:R(\lambda_0)x$.
\item Für alle $x\in X$ gilt die gleichmäßige Konvergenz in $n\in\N$
\begin{equation*}
\lim_{\lambda\to\infty}\lambda R(\lambda, A_n)x=x.
\end{equation*}
\end{enumerate}
Dann sind die Voraussetzungen in \Cref{Trotter-Kato} erfüllt, d. h. $R(\lambda_0)$ ist die Resolvente eines eindeutig bestimmten, dicht definierten und abgeschlossenen Operators $\big(A, D(A)\big)$ in $X$.
\end{folg}

\begin{proof}
Mit der gleichmäßigen Konvergenz $\lim_{\lambda\to\infty}\lambda R(\lambda, A_n)x=x$ in $n$ gibt es für alle $\epsilon>0$  ein $\lambda_0$ so, dass für alle $\lambda > \lambda_0$ und $n\in\N$ gilt:
\begin{equation*}
\|\lambda R(\lambda, A_n)x-x\|\leq \epsilon.
\end{equation*}
Für alle $\lambda$ hinreichend groß gilt somit für alle $x\in X$
\begin{equation*}
    \lim_{n\to\infty}\|\lambda R(\lambda, A_n)x-x\|=\|\lim_{n\to\infty}\lambda R(\lambda, A_n)x-x\|=\|\lambda R(\lambda)x-x\|\leq \epsilon.
\end{equation*}
Damit sind die Voraussetzung von \Cref{Bedingung wenn Pseudoresolvente Resolvente ist}, Aussage (2), erfüllt. Insbesondere ist dann $\overline{\textnormal{Bild}(R(\lambda_0))}=X$ gegeben.
\end{proof}



% \begin{bem}
% % Folgende Behauptung ist ein hinreichendes Kriterium 
% % für die Gültigkeit der Voraussetzung (2) in \textbf{Trotter-Kato}:
% \end{bem}


% \begin{bsp}
% Yosida Approximationen?
% \end{bsp}

\chapter{Störungstheoreme}

% \begin{bem}
% \hl{Lorem ipsum dolor sit amet, consectetur adipiscing elit. Suspendisse et erat tincidunt, lacinia odio et, gravida purus. Aenean vitae odio eget enim fringilla dignissim rhoncus et orci.}
% \end{bem}

% \hl{Sei $\big(A, D(A)\big)$ Generator $C_0$-Halbgruppe in einem Banachraum $X$ und $\big(B, D(B)\big)$ ein weiterer Operator in $X$. Wir möchten untersuchen, unter welchen Bedingungen die Summe $K=A+B$ ebenfalls ein Generator ist, oder zumindest eine Fortsetzung $K\subseteq A+B$ existiert, welche Generator ist.

% \par
% Die Addition einer \index{Operator!Störung}\textbf{Störung} $B$ an den Generator $A$ ändert diesen meist soweit ab, dass $A+B$ nicht mehr erzeugend ist:
% \par Sei $\big(A, D(A)\big)$ ein unbeschränkter Generator. Für eine Störung $B:=-A$ ist der auf $D(A)$ dicht definierte Operator $A+B=0$ nicht  abgeschlossen, insbesondere ist $A+B$ nicht Generator einer $C_0$-Halbgruppe.
% \par}


% \begin{verein}
% In den nachfolgenden Abschnitten seien stets $\big(A, D(A)\big)$ und $\big(B, D(B)\big)$ Operatoren in einem geeigneten Banachraum $X$ mit $D(A)\subseteq D(B)$.
% \end{verein}

% \newpage\section{Kriterien über das Spektrum}

% \begin{defi}
% Es seien $\big(A, D(A)\big)$ und $\big(B, D(B)\big)$ lineare Operatoren in einem Banachraum $X$ mit $D(A)\subseteq D(B)$. Bezeichne $B$ als \index{Operator!$A$-beschränkt}\textbf{$A$-beschränkt}, falls es Konstanten $a,b\geq0$ gibt mit:
% \begin{equation*}
%     \|Bx\|\leq a\|Ax\| + b\|x\|,\quad\forall x\in D(A).
% \end{equation*}
% \end{defi}


% \begin{verein}
% In diesem Abschnitt seien stets lineare Operatoren $\big(A, D(A)\big)$ und $\big(B, D(B)\big)$ in einem Banachraum $X$ mit $D(A)\subseteq D(B)$ gegeben.
% \end{verein}

% \begin{prop}\label{Charakterisieriung von A-Beschränktheit}
% Angenommen, $\varrho(A)\neq \emptyset$. Dann sind äquivalent:
% \begin{enumerate}
% \item $B$ ist $A$-beschränkt.
% \item Es gilt $BR(\lambda, A)\in\mathcal L(X)$ für alle $\lambda\in\varrho(A)$.
% \end{enumerate}
% \end{prop}

% \begin{proof}
% Zur Hinrichtung: Sei $\lambda\in\varrho(A)$. Für $y\in X$ setze $x:=R(\lambda, A)y\in D(A)$. Mit $B$ $A$-beschränkt ist: 
% \begin{equation*}
%     \|BR(\lambda, A)y\|\leq a\|AR(\lambda, A)y\|+b\|R(\lambda, A)y\|.
% \end{equation*}
% Wegen $AR(\lambda, A)=-I + \lambda R(\lambda, A)$ gibt es somit ein $M>0$ mit:
% \begin{equation*}
%     \|BR(\lambda, A)y\|\leq a\|\lambda R(\lambda, A)y- Iy\|+b\|R(\lambda, A)y\|\leq M\|y\|.
% \end{equation*}

% \par
% Zur Rückrichtung: Sei $x\in D(A)$. Wähle $y\in X$ mit $x=R(\lambda, A)y\in D(A)$. Wegen $BR(\lambda, A)\in\mathcal L(X)$ ist $\|BR(\lambda, A)y\|\leq M\|y\|$ für ein $M>0$ für alle $y\in X$. Mit der \textbf{Dreiecksungleichung} folgt:
% \begin{equation*}
%     \|Bx\|\leq M\|(\lambda I-A)x\|\leq \lambda M\|x\| + M\|Ax\|.
% \end{equation*}
% \end{proof}

% \begin{bem}
% Für die Operatoren $A$ und $B$ gilt:
% \begin{enumerate} 
% \item Ist $B$ abgeschlossenen, so folgt $A$-Abgeschlossenheit mit dem \textbf{Satz vom abgeschlossenen Graphen}. 
% \item Ist $A$ resolventenpositiv und $\big(B, D(B)\big)$ positiver Operator mit $D(B)=D(A)$, so ist $B$ mit \Cref{Fortsetzung positiver Operatoren} $A$-abgeschlossen. 
% \end{enumerate}
% \end{bem}


% \begin{satz}[\cite{frosali_van_der_mee_mugelli_2004}, 3.2]\label{Spektrumskriterium}
% Sei $\big(K, D(K)\big)$ eine Fortsetzung von $\big(A+B, D(A)\big)$. Angenommen,  $\Lambda:=\varrho(A)\cap \varrho(K)\neq \emptyset$. Dann gilt
% \begin{enumerate}
% \item $1\not\in \sigma_p\big(BR(\lambda, A)\big)$ für alle $\lambda\in\Lambda$.
% \item $1\in \varrho\big(BR(\lambda, A)\big)$ für ein (und damit für alle) $\lambda\in\Lambda$ genau dann, wenn $D(K)=D(A)$ und $K=A+B$ gilt.
% \item $1\in\sigma_a\big(BR(\lambda, A)\big)$ für ein (und damit für alle) $\lambda\in \Lambda$ genau dann, wenn $D(A)\subsetneq D(K)$ und $K=\overline{A+B}$ gilt.
% \item $1\in\sigma_r\big(BR(\lambda, A)\big)$ für ein (und damit für alle) $\lambda\in\Lambda$ genau dann, wenn $K\supsetneq \overline{A+B}$ gilt.
% \end{enumerate}
% \end{satz}


% \begin{folg}\label{Spektrumskriterium Folgerung}
% Angenommen,  $\Lambda:=\varrho(A)\cap \varrho(K)\neq \emptyset$ und es gebe $\lambda\in\varrho(A)$ so, dass (mindestens) eine der beiden Aussagen gelte:
% \begin{enumerate}
% \item $BR(\lambda, A)$ ist kompakt.
% \item Es gilt $r_\sigma\big(BR(\lambda, A)\big)< 1$.
% \end{enumerate}
% Dann gilt $K=A+B$.
% \end{folg}

% \begin{proof}
% Mit (2): Sei $r_\sigma\big(BR(\lambda, A)\big)<1$. Dann ist $I-BR(\lambda, A)$ invertierbar und die zugehörige Inverse ist gegeben durch die \textbf{Neumann'sche Reihe}:
% \begin{equation*}
% \big(I-BR(\lambda, A)\big)^{-1}=\sum_{n=0}^\infty \big(BR(\lambda, A)\big)^n.
% \end{equation*}
% Damit ist Aussage (2) in \Cref{Spektrumskriterium} erfüllt und es gilt $K=A+B$. Darüber hinaus sehen wir:
% \begin{equation*}
% R(\lambda, A+B)=R(\lambda, A)\big(I-BR(\lambda, A)\big)^{-1} =R(\lambda, A)\sum_{n=0}^\infty \big(BR(\lambda, A)\big)^n.
% \end{equation*}

% \par
% Mit (1): Sei $BR(\lambda, A)$ kompakt. Angenommen, $I-BR(\lambda, A)$ ist nicht invertierbar. Dann ist $1$ ein Eigenwert von $BR(\lambda, A)$. Mit Aussage (1) in \Cref{Spektrumskriterium} gilt $1\not\in\sigma_p\big(BR(\lambda, A)\big)$ für alle $\lambda\in\Lambda$. Dies ist ein \textbf{Widerspruch}.
% \end{proof}


% \begin{satz}[\cite{engel_nagel_2006}, III.1.3]
% Es sei $A\in\mathcal G(M,\omega)$. Ist $B\in\mathcal L(X)$, so ist $K:=A+B$ mit $D(K):=D(A)$ Generator einer $C_0$-Halbgruppe $\big(T_K(t)\big)_{t\geq0}$ mit $K\in\mathcal G(M, \omega +M\|B\|)$. 
% \end{satz}

Wir sind an einer Lösung des folgenden Problems interessiert:
\begin{prob*} \index{Störungstheoreme}Sei $\big(A, D(A)\big)$ Generator einer $C_0$-Halbgruppe in einem Banachraum $X$ und es sei $\big(B, D(B)\big)$ ein weiterer linearer Operator auf $X$. Welche Bedingungen sind hinreichend dafür, dass $A+B$, oder zumindest eine Fortsetzung $K$ von $A+B$, ebenfalls Generator einer $C_0$-Halbgruppe in $X$ ist?
\end{prob*}
Der lineare Operator $B$ wird auch als \textbf{Störung} des Generators $A$ bezeichnet.\index{Störungstheoreme!Störung} Für gewöhnlich ist der Fall, dass $K=A+B$ erzeugend ist, nicht gegeben (betrachte etwa die Störung $B=-A$).
\par
Wir stellen zunächst ein Ergebnis vor, welches eine Lösung für $K=\overline{A+B}$ gewährleistet. Hierbei ist die Dissipativität der linearen Operatoren $A+tB$ für alle $t\in[0,1]$ von zentraler Bedeutung. Insbesondere können wir mithilfe des Resultats Generatoren von Geburts- und Todesprozessen reflexiver Räume charakterisieren (vgl. \Cref{Charakterisierung des Generators eines GTP}).

\par Anschließend zeigen wir die Verallgemeinerung des Störungstheorems nach \textsc{T. Kato}, welches die Existenz einer Fortsetzung $K\supseteq A+B$ auf einem $KB$-Raum $X$ für positive Störungen gewährleistet. Wir zeigen, dass die von $K$ erzeugte $C_0$-Halbgruppe in gewissem Sinne "minimal"\; unter allen $C_0$-Halbgruppen von Fortsetzungen ist und prüfen, unter welchen Bedingungen das Störungstheorem für  $AL$-Räume gilt.

% Dabei erfüllt die $C_0$-Halbgruppe des Generators $K$  zeigen zeigen die "Minimalität" der zugehörigen $C_0$-Halbgruppe des Generators $K$. 


% Im Allgemeinen ist die Situation, dass $K=A+B$ oder $K=\overline{A+B}$ obiges leistet, nicht gegeben. In diesem Kapitel stellen wir zunächst eine Lösung der Problems für 



\section{Störungen dissipativer Operatoren}
% \begin{verein}
% In diesem Abschnitt seien $\big(A, D(A)\big)$ und $\big(B, D(B)\big)$ lineare Operatoren in  einem Banachraum $X$  mit $D(A)\subseteq D(B)$ gegeben und für alle $t\in[0,1]$ sie der lineare Operator $\big(A+tB, D(A)\big)$ dissipativ.
% \end{verein}




% \begin{folg}[\cite{werner_2007}, ???]
% Sei $X$ ein Banachraum und $X'$ das zugehörige Dual. Dann gibt es für jedes $x\in X$ eine $x'\in X'$ mit $\langle x', x\rangle=\|x\|^2 = \|x'\|^2$.
% \end{folg}

% \begin{defi}
% Es seien  $\big(A, D(A)\big)$ und $\big(B, D(B)\big)$ lineare Operatoren. Gilt $D(A)\subseteq D(B)$, dann bezeichne $B$ als \index{Operator!$A$-beschränkt}\textbf{$A$-beschränkt}, falls es Konstanten $a,b\geq0$ gibt mit
% \begin{equation*}
%     \|Bx\|\leq a\|Ax\| + b\|x\|,\quad\forall x\in D(A).
% \end{equation*}
% \end{defi}

% \begin{bem}\label{Abschließbare Operatoren sind A-beschränkt}
% Mit \textbf{Satz des Abgeschlossenen Graphen} (\Cref{Satz vom abgeschlossenen Graphen}) ist jeder abschließbare lineare Operator $B$ auch $A$-beschränkt.
% \end{bem}


% \begin{satz}[\cite{banasiak_lachowicz_moszynski_2006}, Theorem 1.2]
% Es seien lineare Operatoren $\big(A, D(A)\big)$ und $\big(B, D(B)\big)$ in einem Banachraum $X$  mit $D(A)\subseteq D(B)$ gegeben. Angenommen, es gilt:
% \begin{enumerate}
% \item $A$ ist Generator einer Kontraktionshalbgruppe  $\big(T_A(t)\big)_{t\geq0}$.
% \item Für alle $t\in[0, 1]$ ist  $A+tB$ dissipativ.
% \item Es gibt $0\leq\alpha\leq 1 $ und $\beta\geq0$ so, dass für alle $x\in D(A)$ gilt:
% \begin{equation*}
% \|Bx\|\leq \alpha\|Ax\| + \beta\|x\|.
% \end{equation*}
% \end{enumerate}
% Dann folgt:
% \begin{enumerate}
% \item Für $\alpha < 1$ ist $A+B$ Generator einer Kontraktionshalbgruppe.
% \item Ist zudem $B'$  in $X'$ dicht definiert, dann ist für $\alpha =1$ der Abschluss $\overline{A+B}$ Generator einer Kontraktionshalbgruppe.
% \end{enumerate}
% \end{satz}


\begin{satz}[\cite{banasiak_arlotti_2006}, Theorem 4.11]\index{Störungstheoreme!Störungen dissipativer Operatoren}\label{Störungstheorem dissipativer Operatoren 1}
Es seien lineare Operatoren $\big(A, D(A)\big)$ und $\big(B, D(B)\big)$ auf einem Banachraum $X$  mit $D(A)\subseteq D(B)$ gegeben. Angenommen, es gilt:
\begin{enumerate}
\item Für alle $t\in[0,1]$ ist $\big(A+tB, D(A)\big)$ dissipativ.
\item Für ein $t_0\in[0,1]$ ist $\big(A+t_0B, D(A)\big)$ Generator einer Kontraktionshalbgruppe.
\item Für alle $x\in D(A)$ gilt $\|Bx\|\leq a\|Ax\| + b\|x\|$ für $0\leq a< 1$ und $b\geq0$.
\end{enumerate}
Dann ist $K_t=A+tB$ für alle $t\in[0,1]$ Generator einer Kontraktionshalbgruppe in $X$.
\end{satz}

\begin{proof}
%Nach Voraussetzung ist $\big(A, D(A)\big)$ dissipativ und dicht definiert.
Wegen \Cref{Eigenschaften Dissipativer Operatoren}, Aussage (1), ist $I-(A+t_0B)$ injektiv und mit \textbf{Lumer-Phillips} (\Cref{Lumer-Phillips}) invertierbar. Setze
\begin{equation*}
R(t_0):=\big(I-(A+t_0B)\big)^{-1}.
\end{equation*}
Wegen \textbf{Hille-Yosida} (\Cref{Hille-Yosida}) ist $\|R(t_0)\|\leq 1$. Mit $0\leq t_0\leq 1$ gilt dann
\begin{align*}
\|Bx\|
&\leq a\|Ax\|+ b\|x\|\\
&\leq a\|(A+t_0 B)x\| + at_0\|Bx\|+b\|x\|\\
&\leq a\|(A+t_0B)x\| + a\|Bx\| + b\|x\|,\quad \forall x\in D(A). 
\end{align*}
Dies lässt sich umformulieren zu
\begin{equation*}
    \|Bx\|\leq \frac{a}{1-a}\|(A+t_0B)x\| + \frac{b}{1-a}\|x\|,\quad\forall x\in D(A).
\end{equation*}
Betrachte nun
\begin{equation*}
    (A+t_0B)R(t_0)=\big(I-\big(I-(A+t_0B)\big)\big)R(t_0)=R(t_0) - I.
\end{equation*}
Wegen $R(t_0)\colon X\to D(A)$ gilt dann für alle $y\in X$ mit obiger Identität 
\begin{equation*}
    \|BR(t_0)y\|\leq \frac{a}{1-a}\|(R(t_0)-I)y\|+\frac{b}{1-a}\|R(t_0)y\|\leq\frac{2a+b}{1-a}\|y\|.
\end{equation*}
Folglich ist der lineare Operator $BR(t_0)$ beschränkt. Sei nun $t\in[0,1]$. Dann gilt
\begin{align*}
    I-(A+tB)
    &=I-(A+t_0 B)+ (t_0-t)B\\
    &=(I-(A+t_0B)+(t_0-t)B)R(t_0)\big(I-(A+t_0B)\big)\\
    &= \big(I+(t_0-t)BR(t_0)\big)\big(I-(A+t_0B)\big).
\end{align*}
Somit ist $I-(A+tB)$ genau dann invertierbar, wenn $I+(t_0-t)BR(t_0)$ invertierbar ist. Dies ist mit \Cref{Satz von der Neumann'schen Reihe} erfüllt, falls 
\begin{equation*}
    |t-t_0|\|BR(t_0)\|<1.
\end{equation*}
Mit $\|BR(t_0)y\|\leq (1-a)^{-1}(2a+b)\|y\|$ können wir dies sicherstellen, wenn
\begin{equation*}
    |t-t_0|<\frac{1-a}{2a+b}\leq\|BR(t_0)\|^{-1}.
\end{equation*}

% Hence, if for some t0 the cone W is R(λ,t0) invariant, then it is also R(λ,t) invariant for 0 ≤ t − t0 <δ λ. Thus, starting from t0 = 0 we can reach the result for t = 1 in finite number of steps, since δλ > 0 and it does not depend on t0.


Sei $0<\delta$ die Länge des Intervalls, für welches obige Ungleichung erfüllt ist. Dann ist $\delta$ von $t_0$ unabhängig, und somit können wir durch wiederholtes Anwenden des bisher Bewiesenen in endlich vielen Schritten das gesamte Intervall $[0,1]$ abdecken, auf dem dann $I-(A+tB)$ invertierbar ist. Insbesondere ist damit $\text{Bild}(I-(A+tB))=X$. Mit der Dissipativität von $A+tB$ liefert \Cref{Eigenschaften Dissipativer Operatoren}, Aussagen (1) und (2), die Invertierbarkeit von $\lambda I-(A+tB)$ für alle $\lambda>0$.  Da $A+tB$ dicht definiert und abgeschlossen ist, liefert \textbf{Hille-Yosida} (\Cref{Hille-Yosida}) die Behauptung.
\end{proof}

\begin{satz}[\cite{banasiak_lachowicz_2007}, Theorem 4.12]\label{Störungstheorem dissipativer Operatoren 2}
Es seien lineare Operatoren $\big(A, D(A)\big)$ und $\big(B, D(B)\big)$ auf einem Banachraum $X$  mit $D(A)\subseteq D(B)$ gegeben. Angenommen, es gilt:
\begin{enumerate}
\item Für alle $t\in[0,1]$ ist $\big(A+tB, D(A)\big)$ dissipativ.
\item $\big(A, D(A)\big)$ ist Generator einer Kontraktionshalbgruppe.
\item Für alle $x\in D(A)$ gilt $\|Bx\|\leq a\|Ax\| + b\|x\|$ für  $a=1$ und $b\geq0$.
\item $B'$ ist dicht definiert.
\end{enumerate}
Dann ist $K=\overline{A+B}$ Generator einer Kontraktionshalbgruppe in $X$.
\end{satz}






\begin{proof}
Der lineare Operator $\big(A+B, D(A)\big)$ ist dissipativ und dicht definiert. Nach \Cref{Eigenschaften Dissipativer Operatoren} ist  $\big(A+B, D(A)\big)$ abschließbar und $\overline{A+B}$ dissipativ. Wir müssen somit lediglich $\text{Bild}(I-\overline{A+B})=X$ zeigen. Da $\text{Bild}(I-\overline{A+B})$  zudem abgeschlossen ist (siehe \Cref{Eigenschaften Dissipativer Operatoren}), gilt dies, falls $\text{Bild}(I-\overline{A+B})$ dicht in $X$ liegt.

\par
Sei zunächst $f\in X$ beliebig gewählt. Für alle $0\leq r <1$ erfüllt der lineare Operator $\big(rB, D(B)\big)$ mit $a=r<1$ die Voraussetzungen in \Cref{Störungstheorem dissipativer Operatoren 1}. Folglich ist $\big(A+rB, D(A)\big)$ für alle $0\leq r <1$ Generator einer Kontraktionshalbgruppe. Mit \textbf{Lumer-Phillips} (\Cref{Lumer-Phillips}) sowie \Cref{Eigenschaften Dissipativer Operatoren} ist dann $I-(A+rB)$ invertierbar. Demnach gibt es für alle $0\leq r <1$ ein $u_r\in D(A)$ so, dass gilt:
\begin{equation*}
    u_r-(A+rB)u_r=f.
\end{equation*}
Hierbei erfüllt $u_r$ die Abschätzung $\|u_r\|\leq \|f\|$. Damit gilt
\begin{align*}
\|Bu_r\|
&\leq \|Au_r\|+b\|u_r\|\\
&\leq \|(A+rB)u_r\| + r\|B u_r\|+ b\|u_r\|\\
&=\|f-u_r\| + r\|Bu_r\|+ b\|u_r\|\\
&\leq (2+b)\|f\|+r\|Bu_r\|.
\end{align*}
Dies lässt sich umformulieren zu
\begin{equation*}
    (1-r)\|Bu_r\|\leq (2+b)\|f\|.
\end{equation*}
Für alle $v'\in D(B')$ gilt dann mit $r\to 1$ 
\begin{align*}
|\langle v', (1-r)Bu_r\rangle |
&= (1-r)|\langle B'v', u_r\rangle |\\
&\leq (1-r)\|B' v'\|\|u_r\|\\
&\leq (1-r)\|B'v'\|\|f\|\to 0.
\end{align*}
Da die Familie $\{(1-r)\|Bu_r\|\}_{0\leq r<1}$ gleichmäßig beschränkt in $r$ ist und $D(B')$ dicht in $X'$ liegt, impliziert 
\begin{equation*}
    |\langle v', (1-r)Bu_r\rangle |= (1-r)|\langle B'v', u_r\rangle |\to 0,\quad \forall v'\in D(B')
\end{equation*} die schwache Konvergenz von
\begin{equation*}
  \langle v', (1-r)Bu_r\rangle \to  0,\quad\forall v'\in X'.
\end{equation*}
\par
Sei nun $0\neq y'\in X'$ mit 
\begin{equation*}
    \langle y', z\rangle =0,\quad\forall z\in\text{Bild}(I-\overline{A+B}).
\end{equation*}
Insbesondere gilt dann 
\begin{equation*}
    \langle y', u_r-Au_r-Bu_r\rangle =0,\quad\forall u\in D(A).
\end{equation*}
Mit \textbf{Hahn-Banach} (\Cref{Hahn-Banach}) gibt es stets ein $f\in X$ mit $0\neq\langle y', f\rangle$. Mit obiger Überlegung angewendet auf ein solches $f$ fest gewählt erhalten wir die schwache Konvergenz 
\begin{align*}
\langle y', f\rangle 
&=\langle y', u_r-(A+rB)u_r\rangle\\
&=\langle y', (1-r)Bu_r\rangle + \langle y', u_r - Au_r -Bu_r\rangle\\
&=\langle y', (1-r)Bu_r\rangle \to 0.
\end{align*}
Dies ist ein \textbf{Widerspruch}. %Also liegt $\text{Bild}(I-\overline{A+B})$ dicht in $X$. 
\end{proof}

\begin{bem}\label{Kato Dual Satz}\label{Kato Dual abgeschlossen und dicht definiert}
Sei $\big(A, D(A)\big)$ ein linearer Operator auf einem reflexiven Banachraum $X$. Ist $A$ dicht definiert und abschließbar, dann ist $A'$ abgeschlossen und ebenfalls dicht definiert (vgl. \cite{kato_1995}, Theorem III.5.29). Insbesondere ist die Adjungierte $B'$ in \Cref{Störungstheorem dissipativer Operatoren 2} dicht definiert, falls der lineare Operator $\big(B, D(B)\big)$ abschließbar und $X$ ein reflexiver Banachraum ist.
\end{bem}

% \begin{satz}\label{Störungstheoreme!Störungen dissipativer Operatoren}
% Sei $X$ Banachraum sowie  $\big(A, D(A)\big)$ und $\big(B, D(B)\big)$ Operatoren in $X$ mit $D(A)\subseteq D(B)$. Angenommen, es gilt:
% \begin{enumerate}
% \item $A$ ist Generator einer Kontraktionshalbgruppe.
% \item Für alle $t\in[0, 1]$ ist  $A+tB$ dissipativ.
% \item Es gibt $0\leq\alpha\leq 1 $ und $\beta\geq0$ so, dass für alle $x\in D(A)$ gilt
% \begin{equation*}
% \|Bx\|\leq \alpha\|Ax\| + \beta\|x\|.
% \end{equation*}
% \end{enumerate}
% Dann folgt:
% \begin{enumerate}
% \item Für $\alpha < 1$ ist $A+B$ Generator einer Kontraktionshalbgruppe.
% \item Ist zudem $B'$  in dicht definiert, dann ist für $\alpha =1$ der Abschluss $\overline{A+B}$ Generator einer Kontraktionshalbgruppe.
% \end{enumerate}
% \end{satz}





\section{Positive Störungen auf $KB$-Räumen}

% \begin{verein}
% Im folgenden Abschnitt seien die linearen Operatoren $\big(A, D(A)\big)$ und $\big(B, D(B)\big)$ mit $D(A)\subseteq D(B)$ auf einem $KB$-Raum $X$ gegeben.
% %Falls $X$ ein $AL$-Raum ist, so können wir mit \Cref{Charakterisierung von AL-Räumen} ohne Einschränkung $X=L_1(\Omega)$ setzen.
% \end{verein}



\begin{lem}[\cite{banasiak_lachowicz_2007}, Lemma 3.1] \label{Hahn-Banch Lemma}
Sei $X$ ein normierter Vektorverband. Dann gibt es für jedes $0\neq x\in X_+$  ein $x'\in X'_+$ mit
\begin{equation*}
\|x'\|= 1\quad\text{und}\quad\langle x', x\rangle = \|x\|.
\end{equation*}
\end{lem}

\begin{proof}
Sei $x\in X_+$. Mit \textbf{Hahn-Banach} (\Cref{Hahn-Banach}) existiert ein stetiges Funktional  $x'\in X'$ mit $\|x'\| =1$ so, dass gilt:
\begin{equation*}
    \|x\| = \sup_{\|y'\|\leq 1,\substack{y'\in X'}}\langle y', x\rangle = \langle x', x\rangle.
\end{equation*}
Sei $x'\not\in X_+'$. Dann gilt
\begin{equation*}
0 < \|x\|=\langle x', x\rangle = \langle x_+', x\rangle - \langle x_-', x\rangle\leq \langle x_+', x\rangle.
\end{equation*}
Weiter liefert $x_+'\leq |x'|$, dass    $\|x_+'\|\leq \|x'\|=1$, also $\|x\|\geq  \langle x_+', x\rangle$. Somit genügt $x_+'$ der Bedingung
\begin{equation*}
\langle x_+', x\rangle = \langle x', x \rangle = \|x\|.
\end{equation*}
Zeige noch $\|x'_+\|=1$: Angenommen, $\|x'_+\| < 1$. Dann ist $\|\widetilde{x}'\|=1$ für $\widetilde{x}':= \|x_+'\|^{-1} x_+'$  und $\langle \widetilde{x}', x\rangle> \langle x', x\rangle$. Dies ist ein \textbf{Widerspruch}.
\end{proof}



\begin{fsatz}[Kato, \cite{banasiak_lachowicz_2007}, Theorem 3.2]\label{Störungstheorem nach Kato}\index{Störungstheoreme!Kato}
%Sei $X$ ein $KB$-Raum. 
Es seien lineare Operatoren $\big(A, D(A)\big)$ und $\big(B, D(B)\big)$ auf einem $KB$-Raum $X$  mit $D(A)\subseteq D(B)$ gegeben. Angenommen, es gilt:
\begin{enumerate}
\item $\big(A, D(A)\big)$ ist Generator einer positiven Kontraktionshalbgruppe.
\item Es  sei $BR(\lambda, A)$ beschränkt mit $r_\sigma(BR(\lambda, A))\leq1$ für ein $\lambda >0$.%(=s(A)).
\item Es gilt $Bx\geq0$ für alle $x\in D(A)_+$.
\item Für jedes $x\in D(A)_+$ gibt es ein $x'\geq0$ mit $\langle x', x\rangle = \|x\|$ und
\begin{equation*}
   \langle x', (A+B)x\rangle\leq 0.
\end{equation*}

%für alle i $x\in D(A)_+$. Dann gibt es $x'\geq0$ mit   $\langle x', x\rangle = \|x\|$. $\langle x', (A+B)x\rangle\leq 0$ für alle $x\in D(A)_+$,  $\langle x', x\rangle = \|x\|$ und $x'\geq0$ gilt 

%\hl{$\langle x', (A+B)x\rangle\leq 0$ für alle $x\in D(A)_+$, wobei  $x'\in X'_+$ mit   $\langle x', x\rangle = \|x\|$.}
% \begin{align*}
%     \langle x', x\rangle = \|x\|\quad\text{und}\quad \langle x', (A+B)x\rangle\leq 0.
% \end{align*}
\end{enumerate}
Dann existiert eine Fortsetzung $\big(K, D(K)\big)$ von $\big(A+B, D(A)\big)$, welche Generator einer positiven Kontraktionshalbgruppe in $X$ ist. 

\par
Für alle $\lambda >0$ ist $\lambda\in\varrho(K)$ und $R(\lambda, K)$ ist gegeben durch
\begin{align*}\label{Darstellung der Resolvente von K}
R(\lambda, K)x
%&= \lim_{n\to\infty} R(\lambda, A)\sum_{k=0}^n \big(BR(\lambda, A)\big)^k x\\
&=\sum_{k=0}^\infty R(\lambda, A)\big(BR(\lambda, A)\big)^kx,\quad \forall x\in X.
\end{align*}
\end{fsatz}



\begin{bem}\label{Störungstheorem nach Kato Annahme -A positiv}
Ist der lineare Operator $-A$ positiv, so kann die Annahme (2)  ersetzt werden durch
\begin{equation*}
    \|Bx\|\leq \|Ax\|,\quad \forall x\in D(A)_+.
\end{equation*}
\end{bem}

\begin{proof}
\par
Für $\lambda >0$ gilt
\begin{equation*}
0\leq -AR(\lambda, A)=I-\lambda R(\lambda, A)\leq I.
\end{equation*}
Also ist $\|AR(\lambda, A)\|\leq1$. Sei $y\in X_+$. Für $x:=R(\lambda, A)y\in D(A)_+$ gilt
\begin{equation*}
\|Bx\|\leq \|Ax\|\leq \|(\lambda I-A)x\|.    
\end{equation*}
Da $BR(\lambda, A)\geq0$ ist, gilt somit $\|BR(\lambda, A)y\|\leq \|y\|$ für alle $y\in X$.
\end{proof}

\begin{bem}
Ist die Annahme (2) für ein $\lambda_0>0$ erfüllt, so gilt $r_\sigma(BR(\lambda, A))\leq1$ bereits für alle $\lambda > \lambda_0$.
\end{bem}


\begin{proof}
\par
Sei $\lambda > \lambda_0$. Mit der \index{}\textbf{Resolventengleichung} (\Cref{Resolventengleichung}) erhalten wir
\begin{equation*}
BR(\lambda, A)-BR(\lambda_0, A)=(\lambda_0 - \lambda)BR(\lambda_0, A)R(\lambda, A).
\end{equation*}
Da sowohl $BR(\lambda_0, A)$ als auch $R(\lambda,A)$ positiv sind, folgt die Behauptung mit 
\begin{equation*}
    BR(\lambda_0, A)\geq BR(\lambda,A),\quad\forall \lambda > \lambda_0.
\end{equation*}
\end{proof}

\newpage
\begin{proof}[Beweis von \Cref{Störungstheorem nach Kato}]
\par
Für $0\leq r < 1$ definiere den linearen Operator $\big(K_r, D(K_r)\big)$ mit Definitionsbereich $D(K_r):=D(A)$ durch
\begin{equation*}
K_r:= A+rB.
\end{equation*}

Zeige mithilfe von \textbf{Hille-Yosida} (\Cref{Hille-Yosida}), dass $\big(K_r, D(K_r)\big)$ Generator einer positiven Kontraktionshalbgruppe ist.
\par 
Nach Annahme (1) ist $A$ Generator, also ist $K_r$ nach Konstruktion dicht definiert. 

\par 
Zur Existenz und Positivität von $R(\lambda, K_r)$: Für $\lambda>0$ ist
\begin{equation*}
\lambda I - K_r = \lambda I- (A+rB)=\big(I-rBR(\lambda, A)\big)(\lambda I-A).
\end{equation*}
Da $r_\sigma(rBR(\lambda, A))\leq r< 1$ gilt, ist $I-rBR(\lambda, A)$ für alle $0\leq r < 1$ invertierbar (\Cref{Satz von der Neumann'schen Reihe}) und wir erhalten die  Inverse mit der \textbf{Neumann'sche Reihe} vermöge
\begin{equation*}
\big(I-rBR(\lambda, A)\big)^{-1}=\sum_{k=0}^\infty r^k \big(BR(\lambda, A)\big)^k.
\end{equation*}
Folglich ist $\lambda\in\varrho(K_r)$ für alle $\lambda>0$ und die Resolvente ist gegeben durch
\begin{align*}
R(\lambda, K_r)
&=\Big(\big(I-rBR(\lambda, A)\big)(\lambda I-A)\Big)^{-1}\\
&=R(\lambda, A)\big(I-rBR(\lambda, A)\big)^{-1}\\
&=R(\lambda, A)\sum_{k=0}^\infty r^k \big(BR(\lambda, A)\big)^k.
\end{align*}
\par 
Sei $x\in X_+$. Da $\big(A, D(A)\big)$ Generator einer positiven $C_0$-Halbgruppe ist, gilt $0\leq R(\lambda, A)x\in D(A)$ für alle $\lambda$ hinreichend groß (\Cref{Charakterisierung Positiver Halbgruppen}). Damit folgt $BR(\lambda,A)x\geq0$. Also ist  $R(\lambda, K_r)$ ein positiver linearer Operator auf $X_+$. Mit \Cref{Fortsetzung positiver Operatoren auf dem positiven Kegel} ist $R(\lambda, K_r)$ bereits auf ganz $X$ definiert.

\par 
Zur Beschränktheit von $R(\lambda,K_r)$: Sei $x\in D(A)_+$. % Mit \mbox{\Cref{Lemma nach Hahn-Banach}} gibt es $x'\geq0$ mit $\langle x', x\rangle = \|x\|$. 
Dann gibt es $x'\in X_+'$ mit $\langle x', (A+B)x\rangle\leq 0$. Wegen $Bx\geq0$ ist $\langle x', Bx\rangle \geq0$. Damit folgt
\begin{equation*}
\langle x', (A+rB)x\rangle = \langle x', (A+B)x\rangle + (r-1)\langle x', Bx\rangle \leq  0.
\end{equation*}
Nach \Cref{Charakterisierung Dissipativer Operatoren} ist somit $K_r=A+rB$ dissipativ, es gilt also
\begin{equation*}
\|(\lambda I - K_r)x\|\geq \lambda \|x\|.
\end{equation*}
Sei nun $y\in X_+$. Setze $x:=R(\lambda, K_r)y\in D(A)_+$. Mit der  Dissipativität von $K_r$ sowie \Cref{Fortsetzung positiver Operatoren auf dem positiven Kegel} erhalten wir
\begin{equation*}\label{Resolvente von K_r ist relativ beschränkt}
\|R(\lambda, K_r)y\|\leq \lambda^{-1}\|y\|,\quad\forall y\in X.
\end{equation*}
Damit ist  $\big(K_r, D(K_r)\big)$ nach \textbf{Hille-Yosida} (\Cref{Hille-Yosida}) sowie \Cref{Charakterisierung Positiver Halbgruppen} Generator einer positiven Kontraktionshalbgruppe $\big(T_r(t)\big)_{t\geq0}$ in $X$.
% \par
% Zur Abgeschlossenheit von $K_r$: Sei $x_n\in D(K_r)$ mit $x_n\to x$ und $K_rx_n\to y$. Da $R(\lambda, K_r)$ beschränkt ist, ist
% \begin{align*}
% x 
% &= \lim_{n\to\infty} x_n\\
% &=\lim_{n\to\infty}R(\lambda, K_r)(\lambda x_n - K_rx_n)\\
% &=R(\lambda, K_r)(\lambda x-y),
% \end{align*}
% womit $x\in D(K_r)$ und $y=K_r x$.



\par
Zeige mithilfe von \textbf{Trotter-Kato} (\Cref{Trotter-Kato}), dass der starke Limes von $K_r$ in $r\uparrow 1$ Generator einer Kontraktionshalbgruppe in $X$ ist.

\par
Sei hierzu $x\in X_+$. Für $\lambda >0$ betrachte
\begin{equation*}
R(\lambda, K_r)x=R(\lambda, A)\sum_{k=0}^\infty r^k \big(BR(\lambda, A)\big)^kx.
\end{equation*}
Dann ist $R(\lambda, K_r)x$ für $r\uparrow 1$ monoton steigend. Weiter zeigt die Abschätzung $\|R(\lambda, K_r)x\|\leq \lambda^{-1}\|x\|$, dass die Familie $\{\|R(\lambda, K_r)x\|\}_{0\leq r<1}$ gleichmäßig beschränkt in $r$ ist. Da $X$ ein $KB$-Raum ist, gibt es für alle $x\in X_+$ und $\lambda >0$ ein Element $y_{\lambda, x}\in X_+$ so, dass gilt:
\begin{equation*}\label{Grenzwert der Resolventen}
\lim_{r\uparrow 1}R(\lambda, K_r)x = y_{\lambda, x}.
\end{equation*}
Mit der Monotonie und Linearität des Limes erhalten wir somit für alle $\lambda >0$ einen positiven linearen Operator auf $X$ vermöge 
\begin{equation*}
    R(\lambda)\colon x\mapsto R(\lambda)x:=y_{\lambda, x}.
\end{equation*}
%Sei $R(\lambda)\colon x\mapsto y_{\lambda,x}$ die hierdurch punktweise definierte Abbildung auf $X_+$. 
Damit ist die Voraussetzung (1) in \textbf{Trotter-Kato} (\Cref{Trotter-Kato}) erfüllt.

\par
Nach \Cref{Trotter-Kato Bedingung (2)} ist noch zu zeigen, dass $\lambda R(\lambda, K_r)x\to x$ mit $\lambda\to\infty$ für alle $x\in X$  gleichmäßig in $r$ konvergiert. Betrachte hierzu
\begin{equation*}
K_r R(\lambda, K_r)=\lambda R(\lambda, K_r)-I.
\end{equation*}
Sei $x\in D(A)$. Für alle $\lambda>0$ ist
\begin{align*}
\|\lambda R(\lambda, K_r)x -x\|
&=\|K_r R(\lambda, K_r)x\|\\
&=\|R(\lambda, K_r)K_rx\|\\
&\leq \lambda^{-1}\|(A+rB)x\|\\
&\leq \lambda^{-1}(\|Ax\| + \|Bx\|).
\end{align*}
Dies liefert für alle $x\in D(A)$ die gleichmäßige Konvergenz in $r$ von
\begin{equation*}
    \lim_{\lambda\to\infty}\lambda R(\lambda, K_r)x=x.
\end{equation*}
Sei nun $y\in X$ beliebig gewählt. Für alle $\epsilon >0$ gibt es dann ein $x\in D(A)$ mit
\begin{equation*}
    \|y - x\|< \epsilon.
\end{equation*}
Wir erhalten damit
\begin{align*}
\|\lambda R(\lambda, K_r)y - y\|
&\leq \lambda \|R(\lambda, K_r)(y-x)\| + \|y-x\| + \|\lambda R(\lambda, K_r)x -x\|\\
&\leq 2\epsilon + \lambda^{-1}(\|Ax\| + \|Bx\|).
\end{align*}
Mit \textbf{Trotter-Kato} (\Cref{Trotter-Kato}) existiert ein eindeutig bestimmter linearer Operator $\big(K, D(K)\big)$, welcher Generator einer Kontraktionshalbgruppe $\big(T_K(t)\big)_{t\geq0}$ in $X$ ist. Für alle $\lambda>0$ ist $\lambda\in\varrho(K)$ und es gilt
\begin{equation*}
    R(\lambda, K)=R(\lambda).
\end{equation*}
Darüber hinaus gilt für alle $x\in X$ die Identität
\begin{equation*}
\lim_{r\uparrow 1} T_r(t)x = T_K(t)x,\quad\forall t\geq0.
\end{equation*}
Da $\big(T_r(t)\big)_{t\geq0}$ für alle $0\leq r <1$ positiv ist, liefert die  Monotonie des Limes die Positivität von  $\big(T_K(t)\big)_{t\geq0}$.

\par
Zur Darstellung von $R(\lambda, K)$: Mit \textbf{Trotter-Kato} (\Cref{Trotter-Kato}) gilt
\begin{align*}
R(\lambda, K)=\lim_{r\uparrow1}R(\lambda, K_r)=\lim_{r\uparrow 1}\sum_{k=0}^\infty r^k R(\lambda, A)\big(BR(\lambda, A)\big)^k.
\end{align*}
Sei $x\in X_+$. Dann wissen wir
\begin{equation*}
    \lim_{r\uparrow 1}\sum_{k=0}^\infty r^k R(\lambda, A)\big(BR(\lambda, A)\big)^k x\in X.
\end{equation*}
Mit der Monotonie und Positivität von $r\mapsto R(\lambda, K_r)x$ liefert \Cref{Positive Folgen in Banachverbänden}, Aussage (1) und \Cref{Positive Folgen in $KB$-Raum}, dass gilt:
\begin{align*}
R(\lambda, K)x=\sum_{k=0}^\infty R(\lambda, A)\big(BR(\lambda, A)\big)^kx\in X.
\end{align*}
Mit der Positivität von $R(\lambda, K)$ gilt dies bereits für alle $x\in X$.

\par
Zeige noch, dass $K$ eine Fortsetzung von $A+B$ ist: Für $\lambda>0$ setze
\begin{equation*}
R^{(n)}(\lambda):= \sum_{k=0}^n R(\lambda, A)\big(BR(\lambda, A)\big)^k,\quad\forall n\in\N.
\end{equation*}
Dies lässt sich umformulieren zu
\begin{align*}
R^{(n)}(\lambda) 
&= R(\lambda, A) + \Big(\sum_{k=0}^{n-1}R(\lambda, A) \big(BR(\lambda, A)\big)^k\Big) BR(\lambda,  A)\\
&= R(\lambda, A) + R^{(n-1)}(\lambda) B R(\lambda, A).
\end{align*}
Sei $x\in D(A)$. Dann gilt
\begin{equation*}
R^{(n)}(\lambda)(\lambda I -A)x = x + R^{(n-1)}(\lambda)Bx.
\end{equation*}
Für $n\to \infty$ erhalten wir $R(\lambda, K)(\lambda I- A)x=x+R(\lambda, K)Bx$ und damit
\begin{equation*}
R(\lambda, K)(\lambda I -A-B)x=x.
\end{equation*}
Folglich ist $x\in D(K)$ mit $(\lambda I -K)x = (\lambda I -A-B)x$.
\end{proof}


% \begin{bem}
% \hl{Lorem ipsum dolor sit amet, consectetur adipiscing elit. Pellentesque vulputate pellentesque nunc, ut iaculis purus ornare nec. Nunc finibus rhoncus odio, non maximus tortor elementum eu. Duis rutrum tincidunt dignissim. Donec vel urna non felis congue iaculis. Aenean eu velit sagittis, tempor magna eu, volutpat urna.}
% \end{bem}



% \begin{bem}
% Es folgt das Beispiel eines Cauchy-Problems, für das Ergebnis von \Cref{Störungstheorem dissipativer Operatoren 2} auf nicht 
% \end{bem}

% \begin{bsp}[\cite{banasiak_lachowicz_2007}, Beispiel 3.8]\label{Bespiel für Störungstheorem nach Kato}
% Gegeben sei der (reelle) $KB$-Raum $X=\textnormal{L}^2([0,1])$. Betrachte das Cauchyproblem
% \begin{align*}
% \frac{\partial x}{\partial t}(t,s) &= -\nu(s) x(t,s)+\mu(s)\int_0^1\nu(r) x(t,r)\text dr,\quad 0\leq s\leq 1, t> 0,\\
% x(0,s) &= x_0(s),
% \end{align*}
% wobei $0\leq\nu\in \textnormal{L}^1([0,1])\setminus \textnormal{L}^2([0,1])$ mit $\int_0^1\nu(s)\text ds=1$ und $0\leq \mu\leq \nu$ mit $\int_0^1\mu^2(s)\text ds=1$ gilt. 
% % Setze $Ax:=-\nu x$ für $x\in D(A):=\{x\in X; \nu x\in X\}$ sowie $Bx:=\mu \int_0^1 \nu(s)x(s)\text ds$ für alle $x\in D(A)$. 
% \par
% Setze $Ax:=-\nu x$ für $x\in D(A):=\{x\in X; \nu x\in X\}$. Dann ist der lineare Operator $\big(A, D(A)\big)$ Generator einer positiven Kontraktionshalbgruppe. Mit der \textbf{Cauchy-Schwarz-Ungleichung} gilt
% \begin{equation*}
% \int_0^1 \nu(s) x(s)\text ds\leq\sqrt{\int_0 ^1\nu^2(s)x^2(s)\text ds}.
% \end{equation*}
% Somit ist $Bx:=\mu \int_0^1 \nu(r)x(r)\text dr$ für alle $x\in X$ ein wohldefinierter linearer Operator mit $D(B)= D(A)$ und für alle $x\in D(A)_+$  gilt $Bx\geq0$. 
% \par
% Dann existiert eine Fortsetzung $\big(K, D(K)\big)$ von $\big(A+B, D(A)\big)$, welche Generator einer positiven Kontraktionshalbgruppe in $X$ ist. 
% \end{bsp}

% \begin{bem}
% Der lineare Operator $\big(B, D(B)\big)$ ist von endlichem Rang und somit nicht abschließbar (siehe \cite{kato_1995}, Aussage III.5.18). Somit ist etwa  \Cref{Kato Dual abgeschlossen und dicht definiert} nicht anwendbar. 
% \par
% Darüber hinaus gilt mit $\nu\not\in X$, dass $D(B')=\{0\}$. Also ist die Annahme (4) in \Cref{Störungstheorem dissipativer Operatoren 2} nicht erfüllt und das Ergebnis des Satzes ist nicht auf die linearen Operatoren $\big(A, D(A)\big)$, $\big(B, D(B)\big)$ aus \Cref{Bespiel für Störungstheorem nach Kato} anwendbar. Die Behauptung kann jedoch mit \Cref{Störungstheorem nach Kato} gezeigt werden.
% \end{bem}


% \begin{proof}[Beweis des Beispiels]
% % Nach \Cref{} ist $B$ nicht abschließbar. Weiter gilt wegen $\nu\not\in X$, dass $D(B')=\{0\}$ ist. 
% Wir müssen nur noch die Annahmen (2) und (4) in \textbf{Kato} (\Cref{Störungstheorem nach Kato}) nachweisen.
% \par
% Zu (2): Für alle $x\in D(A)$ gilt
% \begin{align*}
%     \|Bx\|^2 
%     &=\int_0^1 \mu^2 (s)\Big(\int_0^1\nu(r)x(r)\text dr\Big)^2 \text ds\\
%     &=\Big(\int_0^1 \nu(r) x(r)\text dr\Big)^2\\
%     &\leq \int_0^1\nu^2(r)x^2(r)\text dr = \|Ax\|^2.
% \end{align*}
% Mit \Cref{Störungstheorem nach Kato Annahme -A positiv} ist somit $r_\sigma(BR(\lambda, A))\leq1$ für alle $\lambda >0$. 
% \par
% Zu (4): Es bezeichne $(\cdot,\cdot)$ das Standardskalarprodukt auf $\textnormal{L}^2([0,1])$. Für $x\geq0$ gilt dann 
% %\hl{ können wir ohne Einschränkung $x$ anstelle von $x/\|x\|$ schreiben}. 
% \begin{align*}
% (x, Ax+Bx)
% &=-\int_0^1 \nu(s)x^2(s) \text ds + \Big(\int_0^1 \mu(s)x(s)\text ds\Big)\Big(\int_0^1\nu(s)x(s)\text ds\Big)\\
% &\leq -\int_0^1 \nu(s) x^2(s)\text ds + \Big(\int_0^1 \nu(s)x(s)\text ds\Big)^2\\
% &\leq -\int_0^1\nu(s)x^2(s)\text ds + \Big(\int_0^1\nu(s)\text ds\Big)\Big(\int_0^1 \nu(s)x^2(s)\text ds\Big)\\
% &=\int_0^1 \nu(s)x^2(s)\text ds\Big(-1+\int_0^1\nu(s)\text ds\Big)=0.
% \end{align*}
% \par
% % Mit \textbf{Kato} (\Cref{Störungstheorem nach Kato}) existiert folglich eine Fortsetzung $\big(K, D(K)\big)$ von $\big(A+B, D(A)\big)$, welche Generator einer positiven Kontraktionshalbgruppe in $\textnormal{L}^2([0,1])$ ist.
% \end{proof}

% \begin{bem}
% Die Bedingungen an $\nu, \mu$ in \Cref{Bespiel für Störungstheorem nach Kato} sind etwa mit $\nu(s):=2s^{-1/2}$ auf $[0,1]$ sowie $\mu=\nu$ auf $[\text e^{-4}, 1]$ und $\mu=0$ sonst erfüllt.
% \end{bem}

% \begin{bem}
% \hl{Lorem ipsum dolor sit amet, consectetur adipiscing elit. Pellentesque vulputate pellentesque nunc, ut iaculis purus ornare nec. Nunc finibus rhoncus odio, non maximus tortor elementum eu. Duis rutrum tincidunt dignissim. Donec vel urna non felis congue iaculis. Aenean eu velit sagittis, tempor magna eu, volutpat urna.}
% \end{bem}

\begin{prop}[\cite{banasiak_lachowicz_2007}, Proposition 3.10]\label{Störungstheorem nach Kato, Minimalität der Fortsetzung}
Es seien  die Voraussetzungen in \textbf{Kato} (\Cref{Störungstheorem nach Kato}) gegeben und es bezeichne $\big(T_K(t)\big)_{t\geq 0}$ die von der Fortsetzung $K\supseteq A+B$ erzeugte $C_0$-Halbgruppe auf $X$. Sei nun $D$ Gen von $\big(A, D(A)\big)$, insbesondere gilt $\overline{D}=D(A)$. Ist $\big(\widetilde{K}, D(\widetilde{K})\big)$ eine weitere Fortsetzung von $\big(A+B, D\big)$, die Generator einer  positiven Kontraktionshalbgruppe $\big(T(t)\big)_{t\geq0}$ in $X$ ist, dann gilt $T(t)\geq T_K(t)$ für alle $t\geq0$.
\end{prop}

\begin{proof}
\par 
Zeige, dass $\big(\widetilde{K}, D(\widetilde{K})\big)$ bereits eine Fortsetzung von $\big(A+B, D(A)\big)$ ist: Sei  $x\in D(A)$. Dann gibt es eine Folge $(x_n)_{n\in\N}$ in $D$ mit $x_n \to x$ und $Ax_n\to Ax$. Mit $r_\sigma(BR(\lambda, A))\leq 1$ ist der lineare Operator $\big(B, D(B)\big)$ auf $D(A)$ beschränkt, insbesondere abgeschlossen. Damit gilt $Bx_n\to Bx$ und wir schließen
\begin{equation*}
    (A+B)x_n\to (A+B)x,\quad\forall x\in D(A).
\end{equation*}
Da $\widetilde{K}$ als Generator abgeschlossen ist, gilt
\begin{equation*}
\widetilde{K}x_n  = (A+B)x_n,\quad\forall n\in\N.
\end{equation*}
Folglich ist $\widetilde{K}x=(A+B)x$ und damit $x\in D(\widetilde{K})$.

\par
Zu "$T(t)\geq T_K(t)$": Für $\lambda >0$ ist $\lambda\in \varrho(\widetilde{K})$ und  $R(\lambda, \widetilde{K})$ für $\lambda$ hinreichend groß positiv. Für alle $0\leq r <1$ definiere wie im Beweis von \textbf{Kato} (\Cref{Störungstheorem nach Kato}) den linearen Operator $K_r:=A+rB$. Dann wissen wir, dass $\lambda\in\varrho(K_r)$ für alle $\lambda>0$ mit $R(\lambda, K_r)\geq0$ gilt. Damit erhalten wir
\begin{align*}
R(\lambda, \widetilde{K}) - R(\lambda, K_r)
&= \big(R(\lambda, \widetilde{K})(\lambda I- K_r)-I\big)R(\lambda, K_r)\\
&= R(\lambda, \widetilde{K})(\lambda I- K_r - \lambda I+  \widetilde{K})R(\lambda, K_r)\\
&= R(\lambda, \widetilde{K})(\widetilde{K} - K_r)R(\lambda, K_r)\\
&= R(\lambda, \widetilde{K})(A+B - A - rB)R(\lambda, K_r)\\
&= (1-r)R(\lambda, \widetilde{K})BR(\lambda,  K_r).
\end{align*}
Da alle beteiligten Terme positiv sind, folgt
\begin{equation*}
R(\lambda, \widetilde{K})\geq R(\lambda, K_r).
\end{equation*}
Mit der Monotonie des Limes wird dies zu
\begin{equation*}
R(\lambda, \widetilde{K})\geq R(\lambda, K) = \lim_{r\uparrow 1}R(\lambda, K_r).
\end{equation*}
Nach \Cref{Darstellung der Gruppe mithilfe der Resolvente} erhalten wir $T(t)\geq T_K(t)$ für alle $t\geq0$.
\end{proof}

% \begin{lem}\cite{kato_1954}\label{Eigenschaften von T_K(t)}
% Sei $x\in X_+$. Dann gilt
% \begin{enumerate}
% \item $\|T_K(t)x\|$ ist  fallende Funktion in $t$.
% \item Es gilt $\lim_{t\to\infty} \|T_K(t)x\| = \|x\| - \lim_{\lambda\downarrow 0} \lim_{n\to\infty}\|\big(BR(\lambda, A)\big)^n x\|$.
% \item Für $0\neq x$ ist $\lambda^{-1}(\|x\| - \lim_{n\to\infty}\|\big(BR(\lambda, A)\big)^n x\|)$ eine strikt positive, fallende Funktion in $\lambda$ mit
% \begin{equation*}
% \int_0^\infty \|T_K(t)x\|\textnormal dt = \lim_{\lambda\downarrow 0} \lambda^{-1}(\|x\| - \lim_{n\to\infty}\|\big(BR(\lambda, A)\big)^n x\|)
% \end{equation*}
% \end{enumerate}
% \end{lem}

% \begin{satz}[\cite{}, Theroem ???]\label{Hauptaussage im Komplexen}
% Sei $X_C$ die Komplexifizierung eines $KB$-Raumes $X$. Sind die Voraussetzungen in  \Cref{Hauptaussage} gegeben, dann existiert eine Komplexifizierung $K_C$ von $K$, welche Generator einer positiven Kontraktionshalbgruppe $(T_{K_C}(t))_{t\geq0}$ in  $X_C$ ist.
% \end{satz}


% \begin{bem}
% Sei $1\leq p<\infty$. Dann gilt \Cref{Hauptaussage} nach \Cref{} insbesondere für die Räume $\ell^p$ und  $L^p(\Omega, \mu)$.
% \end{bem}

%\newpage\section{Positive Störungen in AL-Räumen}

% \begin{verein}
% Im folgenden Abschnitt seien die Operatoren $\big(A, D(A)\big)$ und $\big(B, D(B)\big)$ mit $D(A)\subseteq D(B)$ auf einem $AL$-Raum $X$ gegeben. 
% \end{verein}



% \begin{satz}[\cite{voigt_1989}, 1.1]\label{Satz von Voigt}
% Es seien lineare Operatoren $\big(A, D(A)\big)$ und $\big(B, D(B)\big)$ auf einem  Banachverband $X$  mit $D(A)\subseteq D(B)$ gegeben. Angenommen, es gilt:
% \begin{enumerate}
% \item $\big(A, D(A)\big)$ ist resolventenpositiv.
% \item $Bx\geq0$ für alle $x\in D(A)$.
% \end{enumerate} 
% Dann ist für $\lambda > s(A)$ äquivalent:
% \begin{enumerate}
% \item $r_\sigma(BR(\lambda, A))<1$.
% \item $\lambda\in\varrho(A+B)$ und $\big(\lambda I-(A+B)\big)^{-1}\geq0$.
% \end{enumerate}
% Ist eine der Bedingungen erfüllt (und somit beide), dann ist $\big(A+B, D(A)\big)$ ebenfalls resolventenpositiv \hl{mit $\lambda >s(A+B)$ und es gilt}:
% \begin{equation*}
% \big(\lambda I-(A+B)\big)^{-1}=(\lambda I-A)^{-1}\sum_{n=0}^\infty \big(BR(\lambda, A+B)\big)^n\geq(\lambda I-A)^{-1}.
% \end{equation*}
% \end{satz}

% % \begin{defi}

% % \end{defi}

% \begin{lem}%[\cite{banasiak_arlotti_2006}, 5.12]
% \label{Lemma für Desch}
% Sei $X=\textnormal{L}^1(\Omega,\mu)$ sowie $\big(A, D(A)\big)$ Generator einer positiven $C_0$-Halbgruppe in $X$.  Ist $B\in\mathcal L(D(A), X)$ positiv und es gibt $\lambda > s(A)$ mit $\|B(\lambda I-A)^{-1}\|< 1$, dann ist $\big(A+B, D(A)\big)$ Generator einer positiven $C_0$-Halbgruppe in $X$.
% \end{lem}

% \begin{proof}

% \end{proof}


% \begin{satz}[Desch]\index{Störungstheoreme!Desch}
% %Sei $X$ ein $AL$-Raum. Setze ohne Einschränkung $X=\textnormal{L}^1(\Omega,\mu)$. 
% Es seien lineare Operatoren $\big(A, D(A)\big)$ und $\big(B, D(B)\big)$ auf einem $AL$-Raum $X=\textnormal{L}^1(\Omega,\mu)$  mit $D(A)\subseteq D(B)$ gegeben. Angenommen, es gilt:
% \begin{enumerate}
% \item $\big(A, D(A)\big)$ ist Generator  einer positiven $C_0$-Halbgruppe.
% \item Es ist $B\in\mathcal L(D(A), X)$ positiv.
% \item Es gibt $\lambda>s(A)$ so, dass der lineare Operator $\lambda I-(A+B)$ resolventenpositiv ist.
% \end{enumerate}
% Dann ist $K=A+B$ Generator einer positiven $C_0$-Halbgruppe in $X$.
% \end{satz}


% \begin{proof}[Beweis des Satzes]
% Für $\lambda>s(A)$ mit $\lambda I-(A+B)$  resolventenpositiv können wir wegen  \Cref{Satz von Voigt} schließen, dass $r_\sigma(BR(\lambda, A))< 1$ gilt.  Für $\lambda >s(A+B)$ ist dann $\lambda\in\varrho (A+B)$ und $R(\lambda, A+B)$ ist gegeben durch:
% \begin{equation*}
% \big(\lambda I-(A+B)\big)^{-1}=(\lambda I-A)^{-1}\sum_{n=0}^\infty \big(BR(\lambda, A+B)\big)^n\geq(\lambda I-A)^{-1}.
% \end{equation*}
% Für $s\in[0,1]$ ersetze $B$ in obigem Ausdruck mit $sB$. Dies ergibt:
% \begin{equation*}
% (\lambda I-A)^{-1}\leq (\lambda I- A-sB)^{-1}\leq \big(\lambda I-(A+B)\big)^{-1}.
% \end{equation*}
% Da $B$ positiv ist und das $\text{Bild}(\big(\lambda I-(A+B)\big)^{-1})=D(A)$ gilt, ist der lineare Operator $B\big(\lambda I-(A+B)\big)^{-1}$ wegen $B\in\mathcal L(D(A), X)$ ebenfalls beschränkt. Damit gibt es ein $n\in\N$ so, dass $\|B(\lambda I- A-B)^{-1}\|< n$ gilt. Damit können wir für alle $s\in[0,1]$ schreiben:
% \begin{equation*}
% \|n^{-1}B\big(\lambda I-(A+sB)\big)^{-1}\|<1.
% \end{equation*}
% Insbesondere gilt dann:
% \begin{equation*}
% \|n^{-1}B\Big(\lambda I-\Big(A+\frac{j}{n}B\Big)\Big)^{-1}\|<1,\quad\forall  j=0,1,\dots,n-1.
% \end{equation*}
% Somit können wir mit \Cref{Lemma für Desch} die positiven Störungen $n^{-1}B\in\mathcal L(D(A), X)$ sukzessive auf die linearen Operatoren $A, A+n^{-1}B, \dots, A+n^{-1}(n-1)B$ anwenden. Im letzten Schritt erhalten wir somit den Generator $A+B$.
% \end{proof}

% \begin{bem}
% \hl{Lorem ipsum dolor sit amet, consectetur adipiscing elit. Pellentesque vulputate pellentesque nunc, ut iaculis purus ornare nec. Nunc finibus rhoncus odio, non maximus tortor elementum eu. Duis rutrum tincidunt dignissim. Donec vel urna non felis congue iaculis. Aenean eu velit sagittis, tempor magna eu, volutpat urna.}
% \end{bem}

\begin{folg}[\cite{banasiak_lachowicz_2007}, Korollar 3.13]\label{Störungstheorem nach Kato Spezialfall}\index{Störungstheoreme!Kato, Spezialfall}
%Setze ohne Einschränkung $X=\textnormal{L}^1(\Omega,\mu)$ für einen $AL$-Raum $X$. 
Es seien lineare Operatoren\newline $\big(A, D(A)\big)$ und $\big(B, D(B)\big)$ auf $X=\textnormal{L}^1(\Omega,\mu)$  mit $D(A)\subseteq D(B)$ gegeben. Angenommen, es gilt:
\begin{enumerate}
\item $\big(A, D(A)\big)$ ist Generator einer positiven Kontraktionshalbgruppe. %$\big(T_A(t)\big)_{t\geq0}$.
\item Es gilt $Bx\geq0$ für alle $x\in D(A)_+$.
\item Für alle $x\in D(A)_+$ ist $\int_\Omega (Ax+Bx)\textnormal d\mu\leq 0$.
\end{enumerate}
Dann sind die Voraussetzungen von \Cref{Störungstheorem nach Kato} erfüllt, es existiert also eine Fortsetzung $\big(K, D(K)\big)$ von $\big(A+B, D(A)\big)$, welche Generator einer positiven Kontraktionshalbgruppe in $X$ ist.
\end{folg}
\newpage
\begin{proof}
\par
Zeige, dass  die Bedingungen (3) und (4) von \textbf{Kato} (\Cref{Störungstheorem nach Kato}) erfüllt sind.

\par


% Sei $u\in D(A)_+$. Dann ist $(A+B)u\geq0$ und damit $\|A+B\|=\int_\Omega (A+B)\text d\mu$. 



% $\int_\Omega u \text d\mu = \|u\|_1$ und \hl{Dann gibt es $u'\geq0$ mit $\langle u', u\rangle=\|u\|$. Damit erhalten wir}
% \begin{equation*}
% \langle u', (A+B)u\rangle=\int_\Omega (Au+Bu)\textnormal d\mu\leq 0.
% \end{equation*}

\par
Zu (3): Sei $y\in X_+$. Setze $x:=R(\lambda, A)y\in D(A)_+$. Für $\lambda>0$ betrachte
\begin{equation*}
(A+B)x=(A+B)R(\lambda, A)y=-Iy+BR(\lambda, A)y+\lambda R(\lambda, A)y.
\end{equation*}
Mit $\int_\Omega (Ax+Bx)\textnormal d\mu\leq 0$ für alle $x\in D(A)_+$ erhalten wir dann
\begin{equation*}
-\int_\Omega Iy\textnormal d\mu + \int_\Omega BR(\lambda, A)y\textnormal d\mu + \lambda \int_\Omega R(\lambda, A)y\textnormal d\mu\leq 0,\quad\forall y\in X_+,\lambda>0.
\end{equation*}
Da $X$ $AL$-Raum ist, lässt sich dies mit der kanonischen Norm $\|\cdot\|_1$ auf $X$ umformulieren zu
\begin{equation*}
\lambda \|R(\lambda, A)y\| + \|BR(\lambda, A)y\| - \|y\|\leq 0,\quad\forall y\in X_+,\lambda >0.
\end{equation*}
Wegen der Positivität aller Terme gilt dies bereits für alle $y\in X$. Dies liefert dann  $\|BR(\lambda, A)\|\leq 1$.
\par
Zu (4): Für $x\in D(A)_+$ wähle die konstante Linearform $0\leq x'\equiv 1$. Damit ist dann $\langle x',x\rangle = \|x\|$ und
\begin{equation*}
    \langle x', Ax+Bx\rangle =\int_\Omega (Ax+Bx)\textnormal d\mu\leq0.
\end{equation*}
% $X$ $AL$-Raum ist, gilt die Abschätzung bereits für alle 
% %Da $\|R(\lambda, A)u\|\leq \lambda^{-1}\|u\|$ gilt, wird dies zu
% \begin{align*}
% \|BR(\lambda, A)x\|\leq\|x\| - \lambda\|R(\lambda, A)x\|\leq \|x\|.
% \end{align*}
% Mit \Cref{Norm positiver Operatoren} ist die Abschätzung bereits für alle $x\in X$ erfüllt.
\end{proof}

% \begin{satz}
% Für die lineare Operatoren $\big(A, D(A)\big)$ und $\big(B, D(B)\big)$ auf einem $AL$-Raum $X=\textnormal{L}^1(\Omega,\mu)$  mit $D(A)\subseteq D(B)$ seien die Voraussetzungen von \Cref{Störungstheorem nach Kato Spezialfall} gegeben und es sei $\big(T_K(t)\big)_{t\geq0}$ die von der Fortsetzung $\big(K, D(K)\big)$ erzeugte positive Kontraktionshalbgruppe in $X$. Dann gilt
% \begin{equation*}
%     T_K(t)x=\sum_{n=0}^\infty S_n(t)x,\quad\forall x\in X, t\geq0.
% \end{equation*}
% Die Summanden $S_n(t)$ sind für alle $n\in\N_0$ rekursiv gegeben mit:
% \begin{align*}
%     S_0(t)x&=T_A(t)x,\\
%     S_n(t)x&=\int_0^t S_{n-1}(t-s)BT_A(s)x\text ds,\quad\forall x\in D(A), t\geq0.
% \end{align*}
% \end{satz}



\chapter{Anwendung auf Geburts- und Todesprozesse}

% \begin{figure}[h]
%     \centering
%     \includegraphics[scale = 0.45]{images/purity.png}
%     \caption{\textsc{Purity} (\textit{https://xkcd.com/435/})}
%     \label{fig:my_label}
% \end{figure}
\begin{konstr}[\cite{banasiak_2004}, Abschnit 1.1] Ein (klassischer) \index{Geburts- und Todesprozess}\textbf{Geburts- und Todesprozess}, kurz \textbf{GTP}, beschreibt die Evolution einer Population, deren Anzahl $k$ sich zu jedem Zeitpunkt $t$ entweder zu $k+1$ Individuen (vermöge der "Geburt"\; eines Individuums) oder zu $k-1$ Individuen (vermöge des "Todes"\; eines Individuums) abändert. 
\par
Es bezeichne $b_k$ die Wahrscheinlichkeit, mit welcher die Zustandsänderung $k\mapsto k+1$ realisiert wird  sowie $d_k$  die Wahrscheinlichkeit für die Realisierung der Zustandsänderung $k\mapsto k-1$. Weiter setze $a_k:=d_k+b_k$. Sei $x_k(t)$ die Wahrscheinlichkeit, mit der die Anzahl der Population zu einem Zeitpunkt $t$ gleich $k$ ist. \par
Dann ist das zugehörige  \textbf{Kolmogorov'sche Differentialgleichungssystem} gegeben durch
\begin{align*}
x_0' &= -a_0x_0 + d_1 x_1\\
&\;\;\vdots\\
x_n' &= -a_nx_n + d_{n+1}x_{n+1} + b_{n-1}x_{n-1}  \\
&\;\;\vdots
\end{align*}

\end{konstr}

% 

% \par

% 


% Gegeben sei ein klassischer Geburts- und Todesprozess
% \begin{align*}
% x_0' &= -a_0x_0 + d_1 x_1\\
% &\;\;\vdots\\
% x_n' &= -a_nx_n + d_{n+1}x_{n+1} + b_{n-1}x_{n-1}  \\
% &\;\;\vdots
% \end{align*}
% Wir bezeichnen mit fett gedruckten Lettern die zugehörigen Folgen an Koeffizienten, etwa $\textbf{x}=(x_0, x_1,\dots, x_n,\dots)$; ohne Einschränkung können wir $b_{-1}=d_0=0$ setzen (siehe \cite{}). Weiter seien die Folgen $\textbf{d}, \textbf{b}$ und $\textbf{a}$ alle positiv. Wir das System für alle Elemente im Vektorraum $X=\ell^p$ aller $p$-fach  summierbaren Folgen betrachten.

% \begin{konstr}
% Setzte $\mathcal K$ für die Koeffizientenmatrix aller Einträge der rechten Seite obigen Systems. Für alle Folgen $\textbf{x}\in \ell^p$ ist 
% \begin{equation*}
% \mathcal K\textnormal{\textbf{x}}:=\{b_{n-1}x_{n-1} - a_nx_n + d_{n+1}x_{n+1}\}_{n\in\N_0},
% \end{equation*}
% und analog definiere $\mathcal A$, $\mathcal B$ mit
% \begin{equation*}
%  \mathcal A\textnormal{\textbf{x}}:= \{-a_nx_n\}_{n\in\N_0},\quad 
%  \mathcal B\textnormal{\textbf{x}}:= \{b_{n-1}x_{n-1} + d_{n+1}x_{n+1}\}_{n\in\N_0}.
% \end{equation*} 
% \end{konstr}



% \par
% Damit kann der Geburts- und Todesprozess verstanden werden als
% \begin{equation}
% \textbf{x'}=\mathcal A\textbf{x} + \mathcal B\textbf{x}.
% \end{equation}

% Wir möchten die Existenz einer Familie von Operatoren zeigen, welche das Kolmogorov'sche Differentialsystem "löst". [...]


\section{Grundlegende Eigenschaften}


\begin{verein}
Wir bezeichnen die (reellen) Folgen stets mit fett gedruckten Buchstaben, etwa $\textbf{x}=(x_0,x_1,\dots)$. Weiter nehmen wir an, dass die durch den GTP gegebenen Folgen $\textbf{d},\textbf{b}$ und $\textbf{a}$ stets positiv sind und setzen $b_{-1}:=d_0:=0$.
\end{verein}


\begin{konstr}
Der formale Operator  $\mathbb  K$ eines GTP ist für eine beliebige Folge $\textbf{x}$ gegeben durch
\begin{equation*}
(\mathbb K\textnormal{\textbf{x}})_n:=(-a_nx_n + d_{n+1}x_{n+1} + b_{n-1}x_{n-1})_n.
\end{equation*}
Analog definiere die formalen Operatoren $\mathbb A$ und $\mathbb B$ eines GTP durch
\begin{equation*}
(\mathbb A\textbf{x})_n:=(-a_nx_n)_n\quad\text{ und }\quad (\mathbb B\textbf{x})_n:=(d_{n+1}x_{n+1} + b_{n-1}x_{n-1})_n.
\end{equation*}
\end{konstr}

\begin{defi}Für alle $1\leq p<\infty$ bezeichne  der lineare Operator $\big(\mathcal K_p, D(\mathcal K_p)\big)$ die \textbf{maximale Realisierung} des formalen Operators $\mathbb K$ auf $\ell^p$. Dieser ist gegeben durch
\begin{equation*}
\mathcal K_p\textnormal{\textbf{x}}:= \mathbb K\textnormal{\textbf{x}},\quad \textbf{x}\in D(\mathcal K_p):=\{\textnormal{\textbf{x}}\in \ell^p; \mathbb K\textnormal{\textbf{x}} \in \ell^p\}.
\end{equation*}
% Analog definiere die \textbf{maximale Realisierungen} $(\mathcal A_p, D(\mathcal A_p)\big)$, $\big(\mathcal B_p, D(\mathcal B_p)\big)$ der formalen Operatoren $\mathbb A$ bzw. $\mathbb B$ in $\ell^p$.
\end{defi}

\begin{lem}[\cite{banasiak_lachowicz_2007}, Lemma 4.1]\label{Abgeschlossenheit des maximalen Operators K_p}
Für alle $1\leq p<\infty$ ist die maximale Realisierung $\big(\mathcal K_p, D(\mathcal K_p)\big)$ ein abgeschlossener linearer Operator.
\end{lem}

\begin{proof}
Sei $(\textnormal{\textbf{x}}^{(n)})_{n\in\N}$ eine konvergente Folge in $\ell^p$ mit $\textnormal{\textbf{x}}^{(n)}\to \textnormal{\textbf{x}}$ und $\mathcal  K_p\textnormal{\textbf{x}}^{(n)}\to\textnormal{\index{}\textbf y}$. Dann ist $(\textnormal{\textbf{x}}^{(n)})_{n\in\N}$ insbesondere punktweise konvergent, d. h. für alle $k\in\N_0$ gilt $x_k^{(n)}\to x_k$. Somit gilt

\begin{equation*}
    y_k = b_{k-1}x_{k-1} -a_k x_k + d_{k+1} x_{k+1},\quad\forall k\in\N_0.
\end{equation*}
Damit ist $\mathcal  K_p\textnormal{\textbf{x}} = \textnormal{\index{}\textbf y}$.
\end{proof}

\begin{defi}
Für alle $1\leq p<\infty$ bezeichne der lineare Operator $\big(A_p, D(A_p)\big)$  %die \textbf{Einschränkung}
die \textbf{maximale Realisierung}  des formalen Operators $\mathbb A$ auf $\ell^p$.
%$(\mathcal A_p, D(\mathcal A_p)\big)$ auf $D(A_p)$. 
Diese ist gegeben durch
\begin{equation*}
A_p\textbf{x}:= \mathbb A\textbf{x}, \quad \textbf{x}\in D(A_p):= \{\textnormal{\textbf{x}}\in \ell^p; \mathbb A\textnormal{\textbf{x}}\in \ell^p\}=\Big\{\textnormal{\textbf{x}}\in \ell^p; \sum_{n=0}^\infty a_n^p |x_n|^p < \infty \Big\}.
\end{equation*}
Weiter bezeichne $\big(B_p, D(B_p)\big)$ die Einschränkung des formalen Operators $\mathbb B$ auf den Definitionsbereich $D(A_p)$.
%Realisierung $\mathbb B$ $\big(\mathcal B_p, D(\mathcal B_p)\big)$ auf $D(A_p)$.
\end{defi}

\begin{lem}[\cite{banasiak_lachowicz_2007}, Lemma 4.2]\label{A_p Generator einer Kontraktionshalbgruppe}
Für alle $1\leq p<\infty$ ist $\big(A_p, D(A_p)\big)$ Generator einer positiven Kontraktionshalbgruppe in $\ell^p$.
\end{lem}

\begin{proof}
\par 
Nach Konstruktion ist $\big(A_p, D(A_p)\big)$ dicht definiert. 

\par
Sei $\lambda>0$. Dann ist $\lambda\in \varrho(A_p)$ und $R(\lambda, A_p)$ ist für alle $\textbf{y}\in \ell^p$ gegeben durch
\begin{equation*}
(R(\lambda, A_p)\textnormal{\index{}\textbf y})_n = \Big(\frac{y_n}{\lambda + a_n}\Big)_n.
\end{equation*}
Insbesondere ist $\big(A_p, D(A_p)\big)$  resolventenpositiv. Schließlich gilt
% Die Beschränktheit erhalten wir mit:
% \begin{equation*}
% \|A_pR(\lambda, A_p)\textnormal{\index{}\textbf y}\|_p^p = \sum_{n=0}^\infty \frac{a_n^p}{(\lambda +  a_n)^p}|y_n|^p\leq\sum_{n=0}^\infty |y_n|^p=\|\textbf{y}\|_p^p.
% \end{equation*}
\begin{equation*}
\|R(\lambda, A_p)\textnormal{\index{}\textbf y}\|_p^p = \sum_{n=0}^\infty \frac{1}{(\lambda + a_n)^p}|y_n|^p \leq \frac{1}{\lambda^p}\|\textnormal{\index{}\textbf y}\|_p^p,\quad\forall \textbf{y}\in \ell^p.
\end{equation*}
Mit \textbf{Hille-Yosida} (\Cref{Hille-Yosida}) ist somit $\big(A_p, D(A_p)\big)$ Generator einer positiven Kontraktionshalbgruppe  in $\ell^p$.
\end{proof}



\begin{lem}\label{Lemma zur Dissipativität}
Für $1<p<\infty$ seien die linearen Operatoren $\big(A_p, D(A_p)\big)$ und  $\big(B_p, D(B_p)\big)$ eines GTP auf $\ell^p$ gegeben. Angenommen, es gilt:
\begin{enumerate}
\item Die Folgen $\textnormal{\textbf{d}}, \textnormal{\textbf{b}}\in \ell^p$ sind monoton steigend.
\item Für alle $n\in\N_0$ gilt $0\leq b_n+d_{n+1}\leq a_n$.
\end{enumerate}
Dann ist der lineare Operator $\big(A_p+B_p, D(A_p)\big)$ dissipativ. Weiter sind für alle $t\in[0,1]$ die linearen Operatoren $A_p+tB_p$ ebenfalls dissipativ.
\end{lem}

\begin{proof}
Siehe \cite{banasiak_miroslaw_lachowicz_2006}, Lemma 2.2.
\end{proof}

\section{Existenz einer Fortsetzung}


% \begin{folg}
% Es gilt stets $K_1\subseteq \mathcal K_1$.
% \end{folg}

% \begin{proof}

% \end{proof}

% \begin{satz}
% Es gilt $K_1=\overline{A+B}$ genau dann, wenn 
% \begin{equation}
% \sum_{n=0}^\infty \frac{1}{b_n}\Bigg(\sum_{i=0}^\infty\prod_{j=1}^i\frac{d_{n+j}}{b_{n+j}}\Bigg) = +\infty.
% \end{equation}
% \end{satz}

% \begin{proof}

% \end{proof}



\begin{fsatz}[\cite{banasiak_lachowicz_2007}, Theorem 4.3]\label{Störungstheorem für GTP 1}
Für $1<p<\infty$ seien die linearen Operatoren $\big(A_p, D(A_p)\big)$ und  $\big(B_p, D(B_p)\big)$ eines GTP auf $\ell^p$ gegeben. Angenommen, es gilt:
\begin{enumerate}
\item Die Folgen $\textnormal{\textbf{d}}, \textnormal{\textbf{b}}\in \ell^p$ sind monoton steigend.
\item Es gibt $\alpha\in[0,1]$ so, dass gilt:
\begin{equation*}\label{Annahme für Abschätzung}
0\leq b_n\leq \alpha a_n,\quad 0\leq d_{n+1}\leq (1-\alpha)a_n, \quad \forall n\in \N_0.
\end{equation*}
\end{enumerate} 
%und es gibt ein $\alpha\in[0,1]$ so, dass gilt:
% \begin{enumerate}
% \item $0\leq b_n\leq \alpha a_n$ für alle $n\in\N_0$.
% \item $0\leq d_{n+1}\leq (1-\alpha)a_n$ für alle $n\in\N_0$.
% \end{enumerate}
% \begin{equation*}\label{Annahme für Abschätzung}
% 0\leq b_n\leq \alpha a_n,\quad 0\leq d_{n+1}\leq (1-\alpha)a_n, \quad \forall n\in \N_0.
% \end{equation*}
Dann existiert eine Fortsetzung $\big(K_p, D(K_p)\big)$ von  $\big(A_p+B_p, D(A_p)\big)$, welche Generator einer positiven Kontraktionshalbgruppe in $\ell^p$ ist.
\end{fsatz}



\begin{proof}
\par
Zeige, dass die Voraussetzungen in \textbf{Kato} (\Cref{Störungstheorem nach Kato}) erfüllt sind. 
\par
Zu (1): Folgt mit \Cref{A_p Generator einer Kontraktionshalbgruppe}.

\par
Zu (2), (3): Da $-A_p$ nach Konstruktion positiv ist, reicht es wegen \Cref{Störungstheorem nach Kato Annahme -A positiv}, $\|B_p\textbf{x}\|_p\leq\|A_p\textbf{x}\|_p$ für alle  $\textbf{x}\in D(A_p)_+$ zu zeigen. \par Es gilt
\begin{align*}
\|B_p\textnormal{\textbf{x}}\|_p
&= \Big(\sum_{n=0}^\infty |b_{n-1}x_{n-1} + d_{n+1}x_{n+1}|^p\Big)^{1/p}\\
&\leq \Big(\sum_{n=0}^\infty b_{n-1}^p |x_{n-1}|^p\Big)^{1/p} + \Big(\sum_{n=0}^\infty d_{n+1}^p |x_{n+1}|^p\Big)^{1/p}\\
&\leq \Big(\sum_{n=0}^\infty b_n^p|x_n|^p\Big)^{1/p} + \Big(\sum_{n=0}^\infty d_n^p |x_n|^p\Big)^{1/p},\quad \forall \textbf{x}\in D(A_p)_+.
\end{align*}
Mit der Monotonie von $\textbf{d}$ ist insbesondere $d_{n}\leq d_{n+1}\leq (1-\alpha)a_n$ für alle $n\in\N_0$. Damit gilt weiter
\begin{align*}
\|B_p\textnormal{\textbf{x}}\|_p
% &\leq \Big(\sum_{n=0}^\infty \alpha^p a_n^p|x_n|^p\Big)^{1/p} + \Big(\sum_{n=0}^\infty (a_n^p-\alpha^p a_n^p) |x_n|^p\Big)^{1/p}\\
%- \Big(\sum_{n=0}^\infty \alpha a_n^p |x_n|^p\Big)^{1/p}\\
&\leq \Big(\sum_{n=0}^\infty a_n^p |x_n|^p\Big)^{1/p}= \|A_p \textnormal{\textbf{x}}\|_p,\quad \forall \textbf{x}\in D(A_p)_+.
\end{align*}
Wegen $(B_p\textbf{x})_n=b_{n-1}x_{n-1}+d_{n+1}x_{n+1}$ für $\textbf{b},\textbf{d}\geq0$ ist $\big(B_p, D(B_p)\big)$ nach Konstruktion positiv. Insbesondere ist obige Abschätzung somit für alle $\textbf{x}\in D(A_p)$ erfüllt.

Zu (4): 
Sei $\textnormal{\textbf{x}}\in D(A_p)_+$. Definiere $\tilde{\textnormal{\textbf{x}}} = (\tilde x_n)_{n\in\N}$ durch
\begin{equation*}
\tilde x_n:= 
\begin{cases}
0\quad&\text{falls }x_n=0,\\
x_n^{p-1}&\text{sonst.}
\end{cases}
\end{equation*}
Dann ist $0\leq \tilde{\textnormal{\textbf{x}}}\in \ell^q$ für $1<q<\infty$ mit $1/p + 1/q=1$ und wir erhalten
$\langle \textbf{x},\|\textbf{x}\|_p^{1-p}\tilde{\textnormal{\textbf{x}}}\rangle =\|\textbf{x}\|_p$. 

\newpage
\par 
Um die Abschätzung der Annahme (4) sicherzustellen, können wir den  Faktor $\|\textbf{x}\|_p^{1-p}$ vernachlässigen. 
% Für den Nachweis der Dissipativität können wir $\tilde{\textbf{x}}$ anstelle von $\tilde{\textbf{x}}/\|\textbf{x}\|_p^{1-p}$ schreiben. 
Sei hierzu ohne Einschränkung $x_n\neq 0$ für alle $n\in\N_0$. Dann gilt
% Weiter können wir  $\tilde{\textbf{x}}$ ohne die Multiplikation des Faktors $\|\textbf{x}\|_p^{1-p}$ für den Beweis der Abschätzung $\langle(A_p+B_p)\textbf{x},\tilde{\textbf{x}} \rangle\leq0$ annehmen. 
\begin{align*}
\langle (A_p+B_p)\textnormal{\textbf{x}}, \tilde{\textnormal{\textbf{x}}}\rangle
&= \sum_{n=0}^\infty (A_p\textnormal{\textbf{x}}+B_p\textnormal{\textbf{x}})_n x_n^{p-1}\\
&= -\sum_{n=0}^\infty a_n x_n^{p} + \sum_{n=0}^\infty b_{n-1} x_{n-1} x_n^{p-1} + \sum_{n=0}^\infty d_{n+1} x_{n+1}x_n^{p-1}\\
&\leq -\sum_{n=0}^\infty b_n x_n^p - \sum_{n=0}^\infty d_{n+1}x_n^p \\
&\quad\quad +\sum_{n=0}^\infty b_{n-1} x_{n-1} x_n^{p-1} + \sum_{n=0}^\infty d_{n+1}x_{n+1} x_n^{p-1}.
\end{align*}
Mithilfe der \index{}\textbf{Hölder-Ungleichung} lässt sich dies umformulieren zu
\begin{align*}
\langle (A_p+B_p)\textnormal{\textbf{x}}, \tilde{\textbf{x}}\rangle 
\leq &\Big(\sum_{n=0}^\infty b_n x_n^p\Big)^{1/p}\Big(\sum_{n=0}^\infty b_n x_{n+1}^p\Big)^{1/q} - \sum_{n=0}^\infty b_n x_n^p\\
& + \Big(\sum_{n=0}^\infty d_n x_n^p \Big)^{1/p}\Big(\sum_{n=0}^\infty d_{n+1}x_n^p\Big)^{1/q} - \sum_{n=0}^\infty d_{n+1} x_n^p.
\end{align*}
Da $b_n\leq b_{n+1}$ und  $d_n\leq d_{n+1}$  für alle $n\in\N_0$ gilt, liefert dies
\begin{equation*}
    \langle (A_p+B_p)\textnormal{\textbf{x}}, \tilde{\textnormal{\textbf{x}}}\rangle \leq 0.
\end{equation*}

Mit \Cref{Störungstheorem nach Kato} existiert somit eine Fortsetzung $K_p$ von $A_p + B_p$, welche Generator einer positiven Kontraktionshalbgruppe in $\ell^p$ ist.
\end{proof}
% \par
% Zu (4): Folgt mit \Cref{Lemma zur Dissipativität}.




\begin{folg}[\cite{banasiak_lachowicz_2007}, Korollar 4.4]\label{Charakterisierung des Generators eines GTP}
Sei $1<p<\infty$. Für den in  \Cref{Störungstheorem für GTP 1} gegebenen Fortsetzung $K_p\supseteq A_p+B_p$, welche Generator einer positiven Kontraktionshalbgruppe in $\ell^p$ ist, gilt 
\begin{equation*}
     K_p=\overline{A_p + B_p}.
\end{equation*}
\end{folg}



\begin{proof}
Wir zeigen, dass die Voraussetzungen für \Cref{Störungstheorem dissipativer Operatoren 2} erfüllt sind: 
\par
Wie in \Cref{Abgeschlossenheit des maximalen Operators K_p} können wir zeigen, dass $\mathbb B$ abgeschlossen ist. Damit ist die Einschränkung $\big(B_p, D(B_p)\big)$ insbesondere abschließbar. Nach Konstruktion ist $B_p$ dicht definiert. Wegen der Reflexivität von $\ell^p$ ist dann mit \Cref{Kato Dual abgeschlossen und dicht definiert} die Adjungierte $B_p'$ ebenfalls dicht definiert. Schließlich ist die Dissipativität von $A_p+tB_p$ für alle $t\in[0,1]$ mit \Cref{Lemma zur Dissipativität} sichergestellt. 
\par
Mit \Cref{Störungstheorem dissipativer Operatoren 2} ist somit $\overline{A_p+B_p}$ Generator einer positiven Kontraktionshalbgruppe in $\ell^p$. Insbesondere gilt $K_p=\overline{A_p+B_p}$.
\end{proof}

\newpage


% \begin{satz}
% \hl{Für den Fall $p=1$ sehen wir mit \mbox{\Cite{banasiak_2004}}, dass $K_1\neq\overline{A_1+B_1}$ gilt.}
% \end{satz}


\begin{folg}[\cite{banasiak_lachowicz_2007}, Korollar 4.5]\label{Störungstheorem für GTP 2}
Es seien die linearen Operatoren  $\big(A_1, D(A_1)\big)$ und  $\big(B_1, D(B_1)\big)$ eines GTP auf $\ell^1$ gegeben. Angenommen, es gilt:% für die Folgen  $\textbf{\textnormal{b}},\textbf{\textnormal{d}}\in \ell^1$ gilt: 
\begin{equation*}
a_n\geq (b_n + d_n),\quad \forall n\in\N_0.
\end{equation*}
Dann existiert eine Fortsetzung $\big(K_1, D(K_1)\big)$ von $\big(A_1+B_1, D(A_1)\big)$, welche Generator einer positiven Kontraktionshalbgruppe in $\ell^1$ ist.
\end{folg}

\begin{proof}
Wir prüfen die Bedingung (2) und (3) in \Cref{Störungstheorem nach Kato Spezialfall}.
\par
Zu (2): Mit $a_n\geq (b_n+d_n)$  gilt insbesondere $0\leq b_n\leq a_n$ und $0\leq d_n\leq a_n$ für alle $n\in\N_0$ und wir erhalten die Wohldefiniertheit von $\big(B_1, D(B_1)\big)$ auf $D(A_1)$. Nach Konstruktion ist $B_1$ auf $D(A_1)_+$ positiv.
\par
Zu (3): Sei  $\textbf{x}\in D(A_1)_+$. Dann gilt 
\begin{align*}
\sum_{n=0}^\infty ((A_1+B_1)\textbf{x})_n
&= -\sum_{n=0}^\infty a_n x_n +\sum_{n=0}^\infty b_{n-1}x_{n-1}+\sum_{n=0}^\infty d_{n+1}x_{n+1}\\
&=-\sum_{n=0}^\infty a_n x_n + \sum_{n=0}^\infty b_n x_n+\sum_{n=0}^\infty d_n x_n\leq 0.
\end{align*}
\par
Mit \Cref{Störungstheorem nach Kato Spezialfall} existiert somit eine Fortsetzung $K_1$ von $A_1+B_1$, welche Generator einer positiven Kontraktionshalbgruppe in $\ell^1$ ist.
\end{proof}

% \begin{bem}
% \hl{Lorem ipsum dolor sit amet, consectetur adipiscing elit. Suspendisse et erat tincidunt, lacinia odio et, gravida purus. Aenean vitae odio eget enim fringilla dignissim rhoncus et orci.}
% \end{bem}

% \begin{bem}
% Im Allgemeinen ist $K_1\neq\overline{A_1+B_1}$, siehe etwa \cite{banasiak_2004}, Theorem 5. 
% \end{bem}

\begin{bem}
Im Allgemeinen ist $K_1\neq\overline{A_1+B_1}$. Genauer: 
\par Es sei $K_1\supseteq A_1+B_1$ der in \Cref{Störungstheorem für GTP 2} gegebene Generator einer positiven Kontraktionshalbgruppe in $\ell^1$. Dann gilt genau dann $K_1=\overline{A_1+B_1}$, wenn
\begin{equation*}
\sum_{n=0}^\infty \frac{1}{b_n}\Bigg(\sum_{i=0}^\infty\prod_{j=1}^i\frac{d_{n+j}}{b_{n+j}}\Bigg) = +\infty.
\end{equation*}
% \textbf{Kato} (\Cref{Störungstheorem nach Kato}) gegebenen Generator $K_p\supseteq A_p+B_p$ einer positiven Kontraktionshalbgruppe in $\ell^p$. 
\end{bem}

\begin{proof}
Siehe \cite{banasiak_2004}, Theorem 5.
\end{proof}



% \begin{bem}
% \hl{Es seien die Voraussetzungen in \mbox{\Cref{toller satz}} gegeben. Ist $X$ reflexiv und $B$ abschließbar, dann liegt $D(B')$ dicht in $X'$.}
% \end{bem}


% \newpage\section{Grundlegende Eigenschaften der Fortsetzung}

% \begin{satz}
% Für $1\leq p<\infty$ sei  $\big(\mathcal K_p, D(\mathcal K_p)\big)$ die maximale Realisierung eines GTP in $\ell^p$. Ist  $\big(K_p, D(K_p)\big)$ die in \Cref{Störungstheorem für GTP 1} bzw. \Cref{Störungstheorem für GTP 2} gegebene Fortsetzung von $\big(A_p+B_p,D(A_p)\big)$, Dann gilt
% \begin{equation*}
%     K_p\subseteq \mathcal K_p.
% \end{equation*}
% \end{satz}


% \begin{proof}
% Für $\textbf{x}^r\to\textbf{x}$ mit $r\to 1$ in $\ell^p$ gilt für alle $n\in\N$:
% \begin{align*}
% \lim_{r\to1}((I-\mathcal K_p)\textbf{x}^r)_n
% &=\lim_{r\to 1}(x_n^r + a_nx_n^r - b_{n-1}x_{n-1}^r - d_{n+1}x_{n+1}^r)\\
% &=x_n+a_nx_n - b_{n-1}x_{n-1} - d_{n+1}x_{n+1}\\
% &=((I-\mathcal K_p)\textbf{x})_n.
% \end{align*}
% Sei nun $\textbf{y}\in \ell^p$. Setze  $\textbf{x}^r:=R(1, A_p+rB_p)\textbf{y}$. Dann gilt $\textbf{x}^r\to R(1, K_p)\textbf{y}$ für $r\uparrow 1$. \hl{Da $R(1, A_p+rB_p)$ die Resolvente von $\big(A_p+rB_p, D(A_p)\big)$ ist, welche eine Einschränkung der maximalen Realisierung von $\mathcal -\mathcal A+r\mathcal B$ ist, erhalten wir} für alle $n\in\N$:
% \begin{align*}
% ((I-\mathcal K_p)\textbf{x}^r)_n
% &= x_n^r + a_nx_n^r - rb_{n-1}x_{n-1}^r - rd_{n+1}x_{n+1}^r\\
% &\quad -(1-r)(b_{n-1}x_{n-1}^r + d_{n+1}x_{n+1}^r)\\
% &=y_n - (1-r)(b_{n-1}^rx_{n-1}^r + d_{n+1}x_{n+1}^r).
% \end{align*}
% Da $n$ fest gewählt ist, geht der letzte Term gegen Null und mit \Cref{} erhalten wir $((I-\mathcal K_p)\textbf{x})_n=y_n$, also $
% (I-\mathcal K_p)R(1, K_p)\textbf{y}=\textbf{y}$.
% \end{proof}



