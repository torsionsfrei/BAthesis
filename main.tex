% \chapter{Motivation}

% \section{Geburts- und Todesprozesse (GTP)}

% \par
% Wir wollen mithilfe eines \textbf{Geburts- und Todesprozess} (GTP), der Spezialfall eines \textbf{Kolmogorov'schen Differentialsystem}, die Entwicklung einer Population in Abhängigkeit der Zeit beschreiben. Dieses abzählbare System an Elementen ist gegeben durch
% \begin{align*}
% x_0' &= -a_0x_0 + d_1 x_1\\
% &\;\;\vdots\\
% x_n' &= -a_nx_n + d_{n+1}x_{n+1} + b_{n-1}x_{n-1}  \\
% &\;\;\vdots
% \end{align*}

% \par 
% In diesem Fall ist bezeichnet $x_n$ die \textbf{Wahrscheinlichkeit}, mit der die Population zu einem gewissen Zeitpunkt $t$ aus $n$ Individuen besteht. Mithilfe eines (nicht spezifizierten) Mechanismus kann sich der Zustand des Systems zu $k+1$, vermöge der 'Geburt' eines Individuums, oder analog zu $k-1$, vermöge des 'Todes' eines Individuums, ändern; bezeichnen die Wahrscheinlichkeit, mit der eine Zustandsänderung $k\mapsto k+1$ auftritt mit  $b_k$ und analog $d_k$ für $k\mapsto k-1$; mit $a_n=d_n+b_n$ bezeichnet man obiges System auch als klassischen GTP.

% In \cite{kato_1954} zeigt Kato die Existenz einer $C_0$-Halbgruppe, welche obiges Kolmogorov'sche Differentialsystem für alle absolut summierbaren Folgen $\textbf{x}\in l_1$ löst. Das Ergebnis wurde  in \cite{} auf AL-Räume verallgemeinert, welche die Räume $l_1$ und $l_1$ umfassen. In diesem Text stellen wir das Ergebnis von Banasiak et al.  \cite{banasiak_lachowicz_2007} vor, welches die Existenz für Operatoren in KB-Räumen verallgemeinert. Diese umfassen neben den AL-Räumen auch reflexive Banachräume. 


% Jedoch, falls wir die Existenz einer Halbgruppe beweisen, welche \Cref{} "löst", dann ist das, was wir wirklich erhalten, ist eine Lösung zu einer bestimmten Reformulierung des ursprünglichen Problems in welchem auf der rechten Sete steht der Genrator $K$ einer Halbgruppe. Dieser Generator kann sehr verschieden von $\mathcal A+\mathcal b$ sein und nur eine detailierte Charakterisierung von seinem Definitionsbereich  kann enthüllen ob  die konstruierte Halbgruppe gibt das volle Bild der dynamik beschreiben bei Gleichung \Cref{}. Wie wir zeigen, der Generator $K$ ist zwischen dem minimalen Operator $K_{min}=A+B$ definiert auf $D(A)$ und dem maximalen Operator $K_{max}=\mathcal A+\mathcal b$ definiert auf
% \begin{equation*}
% D_{max}=\{\textbf{u}\in X;\mathcal A\textbf{u}+ \mathcal B\textbf{u}\in X\};
% \end{equation*}
% das heißt also $K_{min}\subseteq K\subseteq K_{max}$. Wo $K$ ist situiert auf einer Skala bestimmt die wohl-gestelltheit des Problems von \Cref{}. Die folgenden Situation sind möglich
% \begin{enumerate}
% \item
% \item
% \end{enumerate}
% und jede dieser hat seine eigene spezifische Interpretation in dem Model.

% \par
% In jedem Fall wo $K\subsetneq K_{max}$ ist keine Eindeutigkeit; das heißt, es gibt differenzierbare $X$-wertige Lösungen zu \Cref{} eminierend von Null und daher sind diese nicht beschdrieben bei dem konstruierten dynamischen System: "there is more to life, than meets the semigroup"\cite{} Um Eindeutigkeit hier zu erreichen, müssen zusätzliche Einschränkungen an de Lösungen gestellt werden.

% \par
% Wenn $\overline{K_{min}}\subsetneq K$, dann obwohl der Fakt dass das mOdell ein formale Konservatives ist, die Lösung ist nicht; the beschreibene Quantität leaks out von dem System und der Mechanismus von seinem leakage wird nicht in dem Model repräsentiert.

% \par
% Abschließend, da $b_n,d_n$ sind die Raten von Austausch von Zuständen in der POpulation, für jede Lösung $\textbf{u}(t)$, die Quantität




% \begin{prop}
% Sei $\mathbb I$ endlich. Dann ist die Matrix $Q=\lim_{h\downarrow 0}\frac{1}{h}\big(P(h)-I\big)$ einer zeitlich-homogenen Markovkette mit Zustandsraum $\mathbb I$  eine Kolmogorov Matrix.
% \end{prop}

% \begin{proof}
% Für $i\neq j$ ist $q_{ij}=\lim_{t\downarrow 0}\frac{1}{t}p_{ij}(t)\geq0$, d.h. $Q$ ist positiv abseits der Diagonalen. Da $\mathbb I$ endlich ist, können wir  $\sum_{j\in \mathbb I}p_{ij}(t)=1$ differenzieren und erhalten $\sum_{j\in I}p_{ij}'(t)=0$ f+r alle $i\in \mathbb I$. Mit $t=0$ ist jede Zeilensumme gleich $0$. 
% \end{proof}

% \begin{lem}\label{Konstruktion von A}
% Es sei der Operator $Q_0$ gegeben. Setze $A$ für die Matrix mit den Diagonaleinträgen der Matrix $(q_{jk})$. Dann definiert $A$ einen Operator mit Definitionsbereich, in Zeichen $(A, D(A))$. Weiter gilt:
% \begin{enumerate}  
% \item $A$ ist abgeschlossen mit $-Ax\geq0$.
% \item $D(A)$ liegt dicht in $X$.
% \item $A$ ist die kleinste Fortsetzung der Einschränkung $Q_0$. 
% \end{enumerate} 
% Weiter existiert für alle $\lambda >0$  die Resolvente $R(\lambda, A)$ mit $R(\lambda, A)\geq0$ sowie $\|R(\lambda, A)\|\leq \lambda^{-1}$ 
% \end{lem}

% \begin{lem}\cite{kato_1954}\label{Konstruktion von B}
% Es sei der Operator $Q_0$ gegeben. Setze $B$ für die Matrix mit den Einträgen der Matrix $(q_{jk})$ abseits der Diagonalen. Dann definiert $B$ einen Operator mit Definitionsbereich, in Zeichen $(B, D(B))$, mit $D(B)= D(A)$. $B$ ist abgeschlossen und es gilt $Bx\geq0$ für alle $x\in D(A)_+$.
% \end{lem}

% % \begin{proof}
% % % Für alle $(\xi_k)=x\in D(A)$ können wir $Bx:=(\eta_k)$ mit  $\eta_k = \sum_{j\neq k}\xi_j q_{jk}$ setzen.
% % \end{proof}

% \begin{lem}
% \cite{kato_1954}
% \label{Eigenschaften von der Komposition von B und der Resolvente von A}
% Für die Operatoren $(A, D(A))$ und $(B, D(B))$ giltn:
% \begin{enumerate}
% \item $\|Bx\| = \|Ax\|$, falls $x\in D(A)_+$ und $\|Bx\| \leq \|Ax\|$ sonst.
% \item Ist $Q$ Fortsetzung von $Q_0$, so gilt $Q_0\subseteq A+B\subseteq Q$.
% \end{enumerate}
% Weite ist für alle  $\lambda >0$ $BR(\lambda, A)$ positive Kontraktion und mit $\lambda\leq \mu$ gilt
% \begin{equation}
% 0\leq BR(\mu, A)\leq BR(\lambda, A).
% \end{equation}
% \end{lem}

% \begin{proof}
% \par
% % Zu (1): Sei $x\in D(A)$. Dann ist
% % \begin{align}
% % \sum_k \sum_{j\neq k}|\xi_j q_{jk}|\leq\sum_j|\xi_j|\sum_{k\neq j} q_{jk}=\sum_j q_j|\xi_j| = \|Hx\|,
% % \end{align}
% % wobei für alle $x\geq0$ Gleichheit gilt.

% % \par
% % Zu (2):

% \par 
% Zu (1): die Komposition $BR(\lambda, A)$ ist als Produkt beschränkter, positiver Operatoren mit \Cref{} wieder positiver Operator.

% \par
% Zu (2): es ist $\|BR(\lambda, A)x\|\leq \|AR(\lambda, A)\|$ nach \Cref{} und mit
% $AR(\lambda, A)=I-\lambda R(\lambda, A)\leq I$ ist $\|BR(\lambda, A)\|\leq \|AR(\lambda, A)\|\leq1$. 
% \par
% Zu (3): Sei $\mu\geq\lambda$. Dann zeigt die \index{}\textbf{Resolventengleichung} \Cref{}
% \begin{equation*}
% BR(\lambda, A)-BR(\mu, A)=(\mu-\lambda)BR(\lambda, A) BR(\mu, A)
% \end{equation*}
% wegen $BR(\lambda, A)\geq0$, dass $BR(\lambda, A)-BR(\mu, A)\geq0$, also $BR(\mu, A)\leq BR(\lambda, A)$.
% \end{proof}

% \section{Störungstheorem nach Kato}


% \begin{prop}\cite{kato_1954}\label{Eigenschaften von Übergangswahrscheinlichkeiten einer ZHMK}
% Sei $P(t)=(p_{jk}(t))_{t\geq 0}$ eine Familie von Matrizen von Übergangswahrscheinlichkeiten einer zeitlich homogenen Markovkette mit abzählbarem Zustandsraum. Dann gilt:
% \begin{enumerate}
% \item Für alle Einträge ist  $p_{jk}(t)\geq 0$ mit $t\geq 0$. 
% \item $\sum_{k=1}^\infty p_{jk}(t) = 1$ mit $t\geq 0$.
% \item Für alle Einträge ist $p_{ik}(s+t) = \sum_{j=1}^\infty p_{ij}(s)p_{jk}(t)$ mit $s,t\geq 0$.
% \end{enumerate}
% \end{prop}

% \begin{proof}
% Siehe \cite{}.
% \end{proof}

% \begin{satz}\cite{kato_1954}\label{Differentialgleichungen ZHMK}
% Sei $(p_{jk}(t))_{t\geq 0}$ eine Matrix von Übergangswahrscheinlichkeiten einer zeitlich homogenen Markovkette mit abzählbarem Zustandsraum. Dann gibt es Konstanten $q_{jk}$ so, dass gilt:
% \begin{enumerate}
% \item Für $j\neq k$ ist $q_{jk}\geq 0$ und $q_{jj}=:q_j\geq 0 $ sonst.
% \item Es gilt $\sum_{k=1}^\infty q_{jk}=0$, insbesondere $\sum_{k\neq j} q_{jk}=q_j$.
% \end{enumerate}
% Die Einträge $p_{jk}(t)$ erfüllen dann die sog. \index{Kolmogorov'sche Diff'Gleichung}\textbf{Kolmogorov'schen Differentialgleichungen} vermöge
% \begin{align*}
% \frac{\textnormal d }{\textnormal dt}p_{jk}(t) &= \sum_{j=1}^\infty p_{ij}(t) q_{jk},\\
% \frac{\textnormal d }{\textnormal dt}p_{ik}(t) &= \sum_{j=1}^\infty q_{ij}p_{jk}(t),
% \end{align*}
% zusammen  mit der Anfangsbedingung
% \begin{align*}
% \lim_{t\downarrow0} p_{jk}(t)=p_{jk}(0) = \delta_{jk}:= 
% \begin{cases}1\quad \text{falls } j=k,\\ 0\quad\text{sonst.}
% \end{cases}
% \end{align*}
% \end{satz}

% \begin{proof}
% Siehe \cite{}.
% \end{proof}

% \begin{defi}
% Eine Matrix $Q=(q_{ij})_{ij}$ heißt  \index{Kolmogorov-Matrix}\textbf{Kolmogorov-Matrix}, falls gilt:
% \begin{enumerate}
% \item Für alle $i\in \N$ ist $0\leq -q_{ii}< \infty$.
% \item Für alle $i\neq j$ ist $q_{ij}\geq0$.
% \item Für alle $i\in\N$ ist $\sum_{j=1}^\infty q_{ij}=0$.
% \end{enumerate}
% \end{defi}

% \begin{konstr}
% Sei $Q=(q_{i,j})_{i,j\in\mathbb I}$ Kolmogorov Matrix und $e_n=(\delta_{n,i})_{i\in\mathbb I}$ Folge in $l^1$. Dann erhalten wir einen linearen Operator $(Q_0, D(Q_0))$ mit Definitionsbereich durch $$D(Q_0)=\text{Lin}(e_n;n\in\mathbb N),\quad Q_0e_i:=(q_{i,j})_{j\in\mathbb I},\quad \forall i\in\mathbb N.$$
% \end{konstr}



% \begin{satz}[Störungstheorem]
% Sei $(Q_0, D(Q_0))$ Operator einer Kolmogorov Matrix $Q$. Dann gibt es eine Fortsetzung von $(Q_0, D(Q_0))$, welche Generator einer stark stetigen Kontraktionshalbgruppe ist.
% \end{satz}



% Da $\mathcal L(X, Y)$ mithilfe der Operatornorm zu einem normierten Raum gemacht werden kann, können wir eine Folge von Operatoren $(A_n)_{n\in\N}$ in $\mathcal L(X,Y)$ als \index{Operator!gleichmäßig  konvergent}\textbf{gleichmäßig konvergent} bezeichnen, falls diese bezüglich der kanonischen Norm konvergent ist, falls also ein Operator $A$ existiert so, dass für alle $\epsilon> 0$ ein $N\in\N$ gibt mit $\|A_n-A\|< \epsilon $ für alle $n\geq N$. Weiter nenne eine Folge $(A_n)_{n\in\N}$  \index{Operator!stark konvergent}\textbf{stark konvergent}, wenn für alle $x\in X$ die Folge $(A_nx)_{n\in\N}$ konvergiert. 



% Eine wichtige Folgerung des Satzes von \index{Banach-Steinhaus}\textbf{Banach-Steinhaus} ist, dass der starke Grenzwert einer Folge beschränkter Operatoren wieder ein beschränkter linearer Operator ist. Gibt es also für jedes $x\in X$ ein $y_x\in Y$, sodass $\lim_{n\to\infty} A_n x=y_x$, dann existiert $A\in\mathcal L(X,Y)$ mit $Ax= y_x$. Darüber hinaus liefert \textbf{Banach-Steinhaus} eine Charakterisierung der starken Konvergenz von Operatorenfolgen. 

% \begin{satz}
% Eine Folge  $(A_n)_{n\in\N}$ in $\mathcal L(X, Y)$ von Operatoren ist genau dann stark konvergent, wenn sie auf allen Kompakta $K\subseteq X$ gleichmäßig konvergiert.
% \end{satz}

% \begin{proof}
% Siehe \cite{banasiak_arlotti_2006}, Aussage 2.12.
% \end{proof}

% \begin{satz}
% Seien $A$ abgeschlossener Operator und $B$ abschließbar. Ist $D(A)\subseteq D(B)$, dann ist $B$ bereits 
% \end{satz}

\chapter{Positive Vektorräume}

\section{Grundlegende Räume: $l_p$ und $L_p(\Omega)$}

\par
% Wir führen die in diesem Text gebräuchliche Notation ein und wiederholen einige Begriffe und Aussagen der Funktionalanalysis (vgl. \cite{}) sowie Operatorentheorie (vgl. \cite{engel_nagel_2006})


Sei $(\Omega, \Sigma,\mu)$ ein \textbf{Maßraum}, d. h. wir haben eine Menge $\Omega$ zusammen mit einer $\sigma$-Algebra $\Sigma$ von offenen Teilmengen von $\Omega$ sowie einem $\sigma$-additiven Maß $\mu$ auf $\Sigma$. 

\par
Für gewöhnlich ist $\Omega\subseteq \R^n$ und $\Sigma$ die $\sigma$-Algebra der \textbf{Lebesgue-messbaren} Mengen. In diesem Fall bezeichne $\mu$ auch als \textbf{Lebesguemaß}. Dieses ist ein \textbf{$\sigma$-endliches} Maß, d. h. wir können $\Omega$ als abzählbare Vereinigung von Mengen mit endlichem Maß darstellen.

\par
Eine Funktion $f\colon\Omega\to \R$ heißt \textbf{messbar}, falls $f^{-1}(B)\in\Sigma$ für Lebesgue-Mengen $B\subseteq \R$ gilt. Wir identifizieren diejenigen Funktionen miteinander, welche auf $\mu$-messbaren Mengen mit dem Maß $0$ übereinstimmen. Dann bezeichne $L_0(\Omega, \textnormal{d}\mu)=:L_0(\Omega)$ die Menge aller Äquivalenzklassen messbarer reellwertiger Funktionen auf $\Omega$. Mit den punktweisen Verknüpfungen der Addition und Skalarmultiplikation wird $L_0(\Omega)$ zu einem Vektorraum. Das \textbf{Integral} einer Menge $\Omega$ einer messbaren Funktion $f$ bzgl. $\mu$ wird mit $\int_\Omega f(x)\textnormal d\mu$ bezeichnet.

\par
Für $1\leq p< \infty$ bezeichne der Teilraum $L_p(\Omega)$ die Menge aller $f\in L_0(\Omega)$, für die gilt:
\begin{equation*}
\|f\|_p:=\|f\|_{L_p(\Omega)}:=\Big(\int_\Omega |f(x)|^p\textnormal dx\Big)^{1/p}<\infty
\end{equation*}
Dann ist $\|f\|_p$ für alle $1\leq p< \infty$ eine Norm und $L_p(\Omega)$ wird mit dieser Norm zu einem Banachraum. 

\par
Weiter sei $L_\infty(\Omega)$ der Vektorraum aller reellwertiger messbaren Funktionen, die (bis auf $\mu$-Nullmengen) beschränkt sind. Die zugehörige Norm $\|\cdot\|_\infty$ ist gegeben durch: 
\begin{equation*}
\|f\|_\infty:=\|f\|_{L_\infty(\Omega)}:=\inf\{M;\mu(\{x\in\Omega; |f(x)|<M\})=0\}<\infty.
\end{equation*}

\par
Ist die Menge $\Omega$ abzählbar, so identifiziere diese mit $\N$ (bzw. $\N_0$). In diesem Fall wird $(\Omega,\Sigma, \mu)$ mit dem \textbf{Zählmaß} zu einem Maßraum. Für $1\leq p< \infty$ bezeichne dann $l_p(\N)=:l_p$ den Banachraum aller Folgen $\textbf{x}=(x_n)_{n\in\N}$, für die gilt:
\begin{equation*}
\|\textbf{x}\|_p:=\Big(\sum_{n=1}^\infty|x_n|^p\Big)^{1/p}< \infty.
\end{equation*}
Weiter bezeichne $l_\infty$ den Raum aller Folgen $\textbf{x}=(x_n)_{n\in\N}$ mit $\|\textbf{x}\|_\infty :=\sup_{n\in\N}\|x_n\|<\infty$.

\par
[...]


\section{Vektorverbände}

\begin{defi}
Sei $X$ eine beliebige Menge. Eine (partielle) \textbf{Ordnung} auf $X$ ist eine (binäre) Relation ''$\geq$'' mit den Eigenschaften:
\begin{enumerate}
\item Für alle $x\in X$ ist $x\geq x$.
\item Ist $x\geq y$ und $y\geq x$, so folgt $x=y$ für alle $x,y\in X$.  
\item Ist $x\geq y$ und $y\geq z$, so folgt $x\geq z$ für alle $x,y\in X$.
\end{enumerate}
Schranke?
[...]
Das \textbf{Supremum} einer Teilmenge von $X$ ist deren kleinste obere Schranke, das \textbf{Infimum} deren größte obere Schranke. Nenne eine (partiell) geordnete Menge $X$ einen \textbf{Verband}, falls für alle Paare $x,y\in X$ deren Supremum $x\vee y:=\sup\{x,y\}$ als auch Infimum $x\wedge y:=\inf\{x,y\}$ in $X$ existieren.
\end{defi}

\begin{defi}
Eine (reeller) Vektorraum $X$ heißt \textbf{geordnet}, falls es eine Ordnung ''$\geq$'' auf der Menge $X$ gibt mit den Eigenschaften:
\begin{enumerate}
\item $x\geq y\Rightarrow x+z\geq y+z$ für alle $x,y,z\in X$. 
\item $x\geq y\Rightarrow \alpha x\geq \alpha y$ für alle $x,y\in X$ und $\alpha\geq0$.
\end{enumerate}
Ein geordneter Vektorraum $X$ heißt \textbf{Vektorverband}, falls die Menge $X$ Verband ist. [...]
\end{defi}

\begin{bsp}
Die Menge $X$ aller reellwertiger Funktionen auf einer Menge $\Omega$ ist ein Vektorraum. Dann wird $X$ mithilfe der punktweisen Ordnung zu einem geordneten Vektorraum: 
\begin{equation}
f\leq g:\iff f(x)\leq g(x),\quad\forall x\in \Omega.
\end{equation}
Weiter setze:
\begin{equation*}
(f\vee g)(x):=\max\{f(x), g(x)\},\quad (f\wedge g)(x):=\min\{f(x), g(x)\},\quad\forall x\in\Omega.
\end{equation*}
Dann ist $X$ ein Vektorverband, falls $f\vee g, f\wedge g\in X$ für alle $f,g\in X$ gilt. Insbesondere sind die Räume $l_p$ und $L_p(\Omega)$ für alle $1\leq p\leq\infty$ Vektorverbände.
\end{bsp}

\begin{defi}
Sei $X$ ein Vektorverband. Für ein Element $x\in X$ sind der  \textbf{Positivteil}, \textbf{Negativteil} und \textbf{Absolutbetrag} von $x$ gegeben durch:
\begin{equation*}
x_+ :=x\vee 0,\quad x_- :=-x \vee 0\quad \text{und} \quad |x| :=x \vee -x.
\end{equation*}
\end{defi}

\begin{defi}
Sei $X$ Vektorraum. Eine Teilmenge $C\subseteq X$ heißt \textbf{konvexer Kegel} in $X$, falls gilt:
\begin{enumerate}
\item $C+C\subseteq C$.
\item $\alpha C\subseteq C$ für alle $\alpha\geq0$.
\item $C\cap(-C)=\{0\}$.
\end{enumerate}
Einen konvexen Kegel $C\subseteq X$ heißt \textbf{erzeugend}, falls $X=C-C$ gilt.
\end{defi}

\begin{bsp}
Der \textbf{positive Kegel} $X_+:=\{x\in X; x\geq 0\}$ eines Vektorverbandes $X$ ist ein konvexer Kegel: [...]
\end{bsp}

\begin{prop}
In einem Vektorverband $X$ gilt für alle $x\in X$:
\begin{equation*}
x = x_+ - x_-\quad \text{und}\quad |x| = x_+ + x_-.
\end{equation*}
Insbesondere ist der positive Kegel $X_+$ in $X$ erzeugend.
\end{prop}

\begin{proof}
Siehe \cite{banasiak_arlotti_2006}, Proposition 2.46.
\end{proof}

\begin{defi}
Sei $X$ Vektorverband und $\|\cdot\|$ eine Norm auf $X$. Dann bezeichne  $\|\cdot\|$ als \textbf{Verbandsnorm}, falls für alle $x,y\in X$ gilt:
\begin{equation*}\label{identity}
|x |\leq  |y |\Rightarrow  \|x \|\leq  \|y \|.
\end{equation*}
Ist $X$  bzgl. der Verbandsnorm  ein vollständiger Vektorraum, so heißt $X$ \textbf{Banachverband}.
\end{defi}

\begin{bem}
Für $x\in X$ Vektorverband gilt stets $\|x \| =  \| |x | \|$: [...]
\end{bem}

\begin{bsp}
\hl{Beispiel Banachverband} [...]
\end{bsp}

\section{Banachverbände: AM- und KB-Räume}

\begin{defi} 
Ein Banachverband $X$ heißt \index{AL-Raum}\textbf{AL-Raum}, falls gilt:
\begin{equation*}
\|x+y\|=\|x\|+\|y\|,\quad\forall x,y\in X_+.
\end{equation*}
\end{defi}

\begin{fsatz}
Sei $X$ Banachverband. Dann sind äquivalent:
\begin{enumerate}
\item $X$ ist AL-Raum.
\item $X$ ist \hl{isometrisch isomorph} zu einem $L_1(\Omega)$ .
\end{enumerate}
\end{fsatz}

\begin{proof}
Siehe \cite{aliprantis_burkinshaw_2006}, Theorem 4.27.
\end{proof}

\begin{defi}
Sei $X$ Vektorverband und $(x_n)_{n\in\N}$ eine Folge in $X$.
\begin{enumerate}
\item Für $(x_n)_{n\in\N}$ nicht-steigend schreibe  $x_n\downarrow x$, falls $\inf_{n\in\N} x_n=x$ gilt.
\item Für $(x_n)_{n\in\N}$ nicht-fallend schreibe $x_n\uparrow x$, falls $\sup_{n\in\N} x_n=x$ gilt.
\end{enumerate}
Dann heißt $(x_n)_{n\in\N}$ \textbf{konvergent bzgl. der Ordnung} auf $X$ bzw. \textbf{ordnungskonvergent} mit Grenzwert $x\in X$, \hl{falls es monotone Folgen $b_n\downarrow x$ und $a_n\uparrow x$ gibt so, dass $a_n\leq x_n\leq b_n$ für alle $n\in\N$ gilt.}
\end{defi}

\begin{defi}
Abgeschlossenheit?
\end{defi}

\begin{prop}
Sei $X$ ein Vektorverband. Dann gilt:
\begin{enumerate}
\item Der positive Kegel $X_+$ ist eine abgeschlossene Teilmenge.
\item Ist $(x_n)_{n\in\N}$ eine nicht-steigende, normkonvergente Folge in $X$ mit $x_n\to x$, dann  gilt $\inf_{n\in\N} x_n=x$.
\item Ist $(x_n)_{n\in\N}$ eine nicht-fallende, normkonvergente Folge in $X$ mit $x_n\to x$, dann  gilt $\sup_{n\in\N} x_n=x$.
\end{enumerate}
\end{prop}

\begin{proof}
Siehe \cite{} [...]
\end{proof}

\begin{bsp}
Die Rückrichtung in \Cref{} gilt im Allgemeinen nicht, d. h. es gibt Folgen $x_n\uparrow x$ in $X$ Vektorverband, für die $(x_n)_{n\in\N}$ nicht normkonvergent ist: 
\par
Setze $x_n:=(1,\dots, 1,0,\dots)$, wobei $1$ die ersten $n$ Einträge der Folge belege und $0$ sonst. Dann ist $x_n\uparrow x$ mit $\sup_{n\in\N}x_n=x:=(1,1,\dots, 1,1,\dots)\in l_\infty$. Hingegen ist $\|x_n - x\|_\infty = 1$ für alle $n\in\N$, $(x_n)_{n\in\N}$ ist also nicht normkonvergent.
\end{bsp}

\begin{defi}
Netze?
\end{defi}

\begin{defi}
Sei $X$ ein Banachverband. Dann bezeichne die zugehörige Norm als \textbf{stetig bzgl. der Ordnung} oder \textbf{ordnungsstetig}, falls für jedes Netz $(x_\alpha)_{\alpha\in\Delta}$ in $X$  gilt:
\begin{equation*}
x_\alpha\downarrow 0\Rightarrow \|x_\alpha\|\downarrow 0.
\end{equation*}
\end{defi}

\begin{defi}
\hl{$\sigma$-Ordnungs vollständig? } [...]
\end{defi}

\begin{bsp}
\hl{Beispiel?} [...]
\end{bsp}

\begin{prop}
Für einen Banachverband $X$ sind äquivalent:
\begin{enumerate}
\item Die Norm von $X$ ist ordnungsstetig.
\item Jede positive Folge $(x_n)_{n\in\N}$ mit $0\leq x_n\uparrow x$ in $X$ ist eine Cauchyfolge.
\item $X$ ist \hl{$\sigma$-Ordnung vollständig} und es für jede Folge $(x_n)_{n\in\N}$ in $X$ gilt:
\begin{equation*}
x_n\downarrow 0\Rightarrow \|x_n\|\to 0.
\end{equation*}
\end{enumerate}
\end{prop}

\begin{proof}
Siehe \cite{} [...]
\end{proof}
% \begin{bsp}
% Der Raum $C([0,1])$ ist nicht vollständig bzgl. der  $\sigma$-Ordnung: 
% \end{bsp}

\begin{bsp}
Für $1\leq p< \infty$ ist die Norm $\|\cdot\|_p$ von  $L_p(\Omega)$ ordnungsstetig:

\par
Sei $f_n\downarrow 0$. Mit dem \textbf{Satz von der dominierten Konvergenz}[...] ist dann $\|f_n\|^p=\int_\Omega f_n^p\text d\mu\to 0$. Da $L_p(\Omega)$ zudem $\sigma$-Ordnungs vollständig ist, folgt die Behauptung.
\end{bsp}

\begin{defi}
Ein Banachverband $X$ heißt \index{KB-Raum}\textbf{KB-Raum}, falls jede nicht-fallende, normbeschränkte positive Folge in $X$ normkonvergent ist.
\end{defi}

\begin{bsp}
Ist $X$ ein Banachverband mit einer nicht ordnungsstetigen Norm, so kann $X$ nicht KB-Raum sein: 

\par
Sei $(x_n)_{n\in\N}$ eine Folge in $X$. Dann impliziert $x_n\uparrow x$, dass $\|x_n\|\leq \|x\|$ für alle $n\in\N$ gilt. Damit ist nach \Cref{} für jeden KB-Raum die Norm bzgl. der Ordnung stetig. Insbesondere sind die Räume $l_\infty$ und $L_\infty(\Omega)$ nicht KB-Räume.
\end{bsp}

\begin{bsp}
Die Menge aller KB-Räume ist echt enthalten in der Menge aller Banachverbände mit ordnungsstetiger Norm:  

\par
Betrachte den Raum $c_0$ aller Nullfolgen: \hl{Die Vollständigkeit bzgl. der $\sigma$-Ordnung ist klar}. Sei $(x_n)_{n\in\N}$ eine Folge in $c_0$ mit $x_n:=(x_k^n)_{k\in\N}$ und $x_n\downarrow 0$. Wähle $\epsilon >0$. Dann gibt es  $k_0\in\N$ mit $|x_k^1|<\epsilon$ für alle $k\geq k_0$. Da die Folge $(x_n)_{n\in\N}$ fallend ist, erhalten wir ebenso $|x_k^n|< \epsilon$ für alle $k\geq k_0$ und $n\geq1$. Es gibt also $n_0$ derart, dass $|x_k^n|<\epsilon$ für alle $n\geq n_0$ und $1\leq k\leq  k_0$ gilt. Zusammen ist dann $\|x_n\|<\epsilon$ für alle $n\geq n_0$, und somit $\|x_n\|\to 0$. Mit \Cref{} ist damit die Norm von $c_0$ ordnungsstetig.

\par
Hingegen ist die nicht-fallende und normbeschränkte Folge $(x_n)_{n\in\N}$ mit  $x_n:=(1,1,\dots,1,0,0,\dots)$, wobei die ersten $n$ Einträge jeweils $1$ seien, in $c_0$ nicht konvergent. Damit ist sie auch nicht KB-Raum, womit die Behauptung folgt.
\end{bsp}


\begin{bsp}
Jeder reflexive Banachverband $X$ ist ein KB-Raum: 

\par
Sie hierzu $(x_n)_{n\in\N}$ eine nicht-fallende positive Folge in $X$ mit $\sup_{n\in\N}\|x_n\|<\infty$. Dann gibt es $x''\in X''$ mit $x_n\uparrow x''$. Da $X$ reflexiv ist, erhalten wir $x''\in X$. \hl{Mit der Stetigkeit der Norm bzgl. der Ordnung ist somit auch $(x_n)_{n\in\N}$ normkonvergent}.
\end{bsp}


\begin{prop}
Jeder $AL$-Raum ist ein $KB$-Raum. 
\end{prop}

\begin{proof}
\hl{Sei $(x_n)_{n\in\N}$ eine nicht-fallende, normbeschränkte Folge in $X$ AL-Raum. Für $0\leq x_n\leq x_m$ mit $n\geq m$ gilt wegen $x_m - x_n\geq0$:}
\begin{equation*}
\|x_m\| = \|x_m - x_n\| + \|x_n\|.
\end{equation*}
Damit ist 
\begin{equation*}
\|x_m - x_n\| = \|x_m \| - \|x_n\| = \big|\|x_m\| - \|x_n\| \big |.    
\end{equation*}
Die Folge $(\|x_n \|)_{n\in\N}$ in $\R$ ist monoton und beschränkt,  insbesondere konvergent. Damit ist $(x_n)_{n\in\N}$ eine Cauchyfolge in $X$.
\end{proof}


\section{Positive Folgen in Banachverbänden}

\par
Für parametrisierte Folgen  ist nachfolgende Aussage eine vergleichbare Version für Banachverbände zu den Sätzen der dominierten und monotonen Konvergenz aus der Analysis:

\begin{satz}\label{Majorisierte Konvergenz in Banachverbänden}
Für alle $t\in T\subseteq \R$ sei $(x_n(t))_{n\in\N}$ eine nicht-negative Folgen in einem Banachverband $X$ und es sei $t_0\in \overline T$.
\begin{enumerate}
\item Ist für alle $n\in\N$ ist die Abbildung $t\mapsto x_n(t)$ nicht-fallend und $\lim_{t\uparrow t_0}x_n(t) =  x_n$ normkonvergent, dann gilt:
\begin{equation*}\label{toll}
\lim_{t\uparrow t_0}\sum_{n=0}^\infty x_n(t) = \sum_{n=0}^\infty x_n.
\end{equation*}
Hierbei wird im Fall von $\sum_{n=0}^\infty x_n\not\in X$ die Gleichheit so verstanden, dass die  Norm auf beiden Seiten gleich unendlich ist, d. h. 
\begin{equation*}
\|\sum_{n=0}^\infty x_n\|:=\sup\{\|\sum_{n=0}^N x_n\|; N\in\N\}= \infty.
\end{equation*}
\item Ist  $\lim_{t\to t_0} x_n(t)= x_n$ für alle $n\in\N$ normkonvergent und es existiert eine Folge $(a_n)_{n\in\N}$ in $X$ mit $\sum_{n=0}^\infty \|a_n \|<\infty$ und $x_n(t)\leq a_n$ für alle $t\in T$ und $n\in \N$, so gilt ebenfalls:
\begin{equation*}
\lim_{t\uparrow t_0}\sum_{n=0}^\infty x_n(t) = \sum_{n=0}^\infty x_n.    
\end{equation*}
\end{enumerate}
\end{satz}

% \begin{proof}
% Siehe \cite{banasiak_arlotti_2006}, Aussage 2.91.
% \end{proof}

\begin{proof}
\par
Zu (1): Angenommen, $\sum_{n=0}^\infty x_n\in X$.  Dann ist $0\leq\sum_{n=0}^\infty x_n(t)\leq\sum_{n=0}^\infty x_n$ für alle $t\in T$, also ist $\sum_{n=0}^\infty x_n(t)$ konvergent. Damit gibt es für jedes $\epsilon>0$ ein $N\in\N$ mit $\|\sum_{n=N+1}^\infty x_n(t)\|\leq \|\sum_{n=N+1}^\infty x_n\|\leq \epsilon/3$ für alle $t\in T$. Damit können wir also für $N$ endlich und fest gewählt ein $t'< t_0$ wählen, sodass für alle  $n\leq N$ und $t'< t< t_0$ die Abschätzung $\|x_n -x_n(t)\|\leq \epsilon /3(N+1)$ gilt. Damit ist 
\begin{equation*}
\Big\|\sum_{n=0}^\infty x_n(t) - \sum_{n=0}^\infty x_n\Big \|\leq \epsilon
\end{equation*}
für alle $t'< t< t_0$. 

\par
Andernfalls gilt $\|\sum_{n=0}^\infty x_n\|=\infty$. Ohne Einschränkung sei dann $\sum_{n=0}^\infty x_n(t)\in X$ für alle $t\in T$. Dann gibt es für alle $M$ ein $N$ mit $\|\sum_{n=0}^N  x_n\|\geq M+1$. Betrachte
\begin{equation*}
\Big\|\sum_{n=0}^N x_n(t)\Big\|= \Big \|\sum_{n=0}^N (x_n(t)- x_n) + \sum_{n=0}^N x_n \Big \| \geq \Bigg |\Big \|  \sum_{n=0}^N x_n\Big \| - \Big \| \sum_{n=0}^N (x_n (t) - x_n)\Big \|\Bigg |.
\end{equation*}
Da $N$ endlich ist, kann der zweite Term kleiner als $1/(N+1)$ mit $t$ hinreichend nah an $t_0$ gemacht werden. Damit ist
\begin{equation*}
\Big \|\sum_{n=0}^\infty x_n (t)\Big\|\geq \Big\|\sum_{n=0}^N x_n(t)\Big \|\geq M,
\end{equation*}
und da $M$ beliebig gewählt war, ist
\begin{equation*}
\lim_{t\to t_0}\Big \|\sum_{n=0}^\infty x_n (t)\Big \| = \infty.
\end{equation*}

\par
Zu (2): Sei $x_n(t)$ konvergent mit $x_n(t)\to x_n$ für $t\to t_0$, wobei $0\leq x_n(t)\leq a_n$ und $\sum_{n=0}^\infty a_n$ konvergent seien. Wegen der Abgeschlossenheit des positiven Kegels nach \Cref{} ist $x_n\leq a_n$. Damit ist
\begin{align*}
\Big\|\sum_{n=0}^\infty (x_n - x_n(t))\Big\| 
&\leq \Big\|\sum_{n=0}^N (x_n - x_n(t))\Big\| + \Big\|\sum_{n=N+1}^\infty (x_n- x_n(t))\Big\|\\
&\leq \Big\|\sum_{n=0}^N (x_n - x_n(t))\Big \|+  2 \Big \|\sum_{n=N+1}^\infty  a_n \Big \|.
\end{align*}
Hierbei kann der zweite Term kleiner als $\epsilon$ mit der gegebene Konvergenz der Summe gemacht werden; der erste Term wird mit $N$ fest wegen der komponentenweisen Konvergenz kleiner als $\epsilon$.
\end{proof}

\begin{folg}\label{Monotone Konvergenz in KB-Raum}
\hl{Ist $(x_n(t))_{n\in\N}$ eine Familie nicht-negativer parametrisierter Folgen in einem KB-Raum $X$, dann liefert $\lim_{t\uparrow t_0}\sum_{n=0}^\infty x_n(t)\in X$ bereits die Konvergenz}
\end{folg}

\begin{proof}
\hl{Betrachte $N\mapsto \sum_{n=0}^N x_n$. Mit $x_n\geq 0$ ist die Folge nicht-fallend, und wegen $X$ KB-Raum gilt somit entweder $\sum_{n=0}^\infty\in X$ oder $\|\sum_{n=0}^\infty x_n\| = \infty$. }
\end{proof}

% \begin{folg}
% Ist $X$ ein $KB$-Raum, dann liefert $\lim_{t\uparrow t_0}\sum_{n=0}^\infty x_n(t)\in X$ bereits Konvergenz von $\sum_{n=0}^\infty x_n$.
% \end{folg}

\chapter{Grundlegendes zur Operatorentheorie}

\section{Positive Operatoren}

Es seien $X$ und $Y$ Banachräume. Ist $D(A)\subseteq X$ ein linearer Teilraum, so heißt eine lineare Abbildung $A\colon D(A)\to Y$  \index{Operator}\textbf{Operator} (mit Definitionsbereich) auf $X$, kurz $(A, D(A))$.
\par
Die Menge aller Operatoren mit $D(A)=X$ wird mit $L(X, Y)$, bzw. $L(X):= L(X,Y)$ falls $Y=X$, bezeichnet. Ist $A\in L(X)$ Operator, so bezeichne $A$ (oder $(A, D(A))$) auch als Operator \textbf{in} $X$. 
\par
Für die Menge aller skalarwertige Operatoren setze $L(X)^*:= L(X,\R)$. Es bezeichne $\mathcal L(X,Y)$ die Menge aller  Operatoren $A\in L(X,Y)$ mit endlicher  \index{Operator!Operatornorm}\textbf{Operatornorm}, d. h. $\|A\|:=\sup_{\|x\|\leq 1}\|Ax\|<\infty$. In diesem Fall bezeichne $A$ auch als  \index{Operator!beschränkt}\textbf{beschränkten} Operator. Mit der Operatornorm wird $\mathcal L(X,Y)$ zu einem Banachraum.
\par
Falls $(A, D(A))$ ein Operator in $X$ und $Y\subseteq X$ eine Teilmenge ist, so ist der \textbf{Teiloperator} von $A$ auf $Y$ durch die Vorschrift $A_Yy := Ay$ sowie dem Definitionsbereich $D(A_Y):=\{x\in D(A)\cap Y; Ax\in Y\}$ gegeben. Eine \textbf{Einschränkung} von $(A, D(A))$ auf eine Teilmenge $D\subseteq D(A)$ wird mit $A|_D$ bezeichnet. Für Operatoren zwei $A, B\in L(X,Y)$ bezeichne $B$ als eine \textbf{Fortsetzung} von $A$, kurz $A\subseteq B$, falls $D(A)\subseteq D(B)$ und $B|_{D(A)}=A$ gilt.


\par
Die Mengen  $\text{Bild}(A):= \{y\in Y; \exists x\in D(A): Ax = y\}$ heißt und $\text{Kern}(A):=\{x\in D(A); Ax=0\}$ werden als \index{}\textbf{Bild} bzw. \index{Operator!Kern}\textbf{Kern} von $A$ bezeichnet. Der \index{Operator!Graph}\textbf{Graph} eines Operators ist die Menge $G(A):=\{(x,y)\in X\times Y; x\in D(A), Ax= y\}$.  
\par
Nenne $A\in  L(X, Y)$ \index{Operator!abgeschlossen}\textbf{abgeschlossen}, falls $G(A)\subseteq X\times Y$ eine abgeschlossene Teilmenge in $X\times Y$ ist. Bezeichne $A$ als \label{Operator!abschließbar}\textbf{abschließbar}, falls $\overline{G(A)}$ der Graph eines Operators ist. Dies ist genau dann der Fall, wenn $(0,y)\in\overline{G(A)}$ stets $y=0$ impliziert. Ist $A$ abschließbar, so bezeichne den Operator $B$ mit Graphen $\overline {G(A)}$ als \textbf{Abschluss} von $A$ und setze $\overline A:=B$. Zusammen mit der \textbf{Graphennorm}  $\|\cdot\|_{D(A)}:=\|x\|_X + \|Ax\|_Y$ wird $D(A)$ zu einem normierten Raum.
\par
\hl{Gen}

\par
[...]

\begin{bem} Für einen Operator  $A\colon D(A)\to Y$ gilt:
\begin{enumerate}
\item $A$ ist abgeschlossen genau dann, wenn für jede Folge $(x_n)_{n\in\N}$ in $D(A)$ mit $\lim_{n\to\infty} x_n = x$ in $X$ und $\lim_{n\to\infty} Ax_n = y$ in $Y$ folgt, dass  $x\in D(A)$ und $y= Ax$ gilt. 
\item $A$ ist bzgl. der Graphennorm $\|\cdot\|_{D(A)}$ stets abgeschlossen.
\item $A$ ist genau dann beschränkt, wenn $(D(A), \|\cdot\|_{D(A)})$ ein Banachraum ist.
\end{enumerate}
\end{bem}

\begin{proof}
Siehe \cite{}
\end{proof}


\begin{bsp}
\hl{Ein einfaches Beispiel eines unbeschränkten, jedoch abgeschlossenen Operators ist der \textbf{Differentialoperator}}. [...]
\end{bsp}

\begin{fsatz}[Banach-Steinhaus]\index{Banach-Steinhaus}
Sei $X$ Banachraum sowie $Y$ normierter Raum. Dann sind äquivalent:
\begin{enumerate}
\item $B\subseteq \mathcal L(X,Y)$ ist gleichmäßig beschränkt.
\item $B\subseteq \mathcal L(X,Y)$ ist stark beschränkt.
\end{enumerate}
\end{fsatz}

\begin{proof}
Siehe \cite{werner_2007}, [...]
\end{proof}


% \par
% Mithilfe des Satzes vom \textbf{abgeschlossenen Graphen} können wir auf ganz $X$ abgeschlossene Operatoren charakterisieren und erhalten ein Kriterium relativ beschränkter Operatoren.
 
% \begin{satz}[Abgeschlossener Graph]
% Ist $A$ Operator mit $D(A)=X$, dann ist $A$ genau dann beschränkt, wenn $G(A)$ abgeschlossen ist.
% \end{satz}

% Sind $(A, D(A))$ und $(B, D(B))$ Operatoren, so bezeichne $B$ als \textbf{$A$-beschränkt}, falls $D(A)\subseteq D(B)$ 

\begin{defi}
Seien $X, Y$ Banachverbände. Ein Operator $A\colon X\to Y$ heißt \index{Operator!positiv}\textbf{positiv}, kurz $A\geq0$, falls $Ax\geq0$ für alle $x\geq0 $ gilt.
\end{defi}

\begin{bsp}
Sei $X=L_1(\Omega)$. Dann ist für jede positive messbare Funktion $k$ auf $\Omega$ der \textbf{Integraloperator} ein positiver Operator. Dieser ist punktweise gegeben durch:
\begin{equation*}
(Af)(x):=\int_\Omega k(x,y)f(y)\text dy,\quad\forall f\in L_1(\Omega),\forall x\in \Omega.
\end{equation*}
\end{bsp}




\begin{bem}
Ein Operator $A$ auf einem Banachverband $X$ ist genau dann positiv, wenn $|Ax|\leq A|x|$ für alle $x\geq0$ gilt: 

\par Wegen $-|x|\leq x\leq |x|$ erhalten wir $-A|x|\leq Ax\leq A|x|$, also $|Ax|\leq A|x|$. Umgekehrt ist $0\leq |Ax|\leq A|x|= Ax$ für alle $x\geq0$.
\end{bem}

% \par
% Ist $A$ positiver Operator, so ist dieser auf dem positiven Kegel $X_+$ bereits vollständig beschrieben:

\begin{fsatz}\label{Fortsetzung positiver Operatoren}
Sei $A\colon X_+\to Y_+$  eine additive Abbildung von Banachverbänden. Dann existiert ein eindeutig bestimmter Operator $\widetilde A$ auf $X$, welcher Fortsetzung von $A$ ist. Für diesen gilt:
\begin{equation*}
\widetilde Ax = Ax_{+} - Ax_{-},\quad\forall x\in X.
\end{equation*}
\end{fsatz}

\begin{proof}
Siehe \cite{banasiak_arlotti_2006}, Aussage 2.65.
\end{proof}

% Eine weitere nützliche Eigenschaft positiver Operatoren ist diese:

\begin{satz}
Sei $X$ Banachverband und  $Y$ normierter Vektorverband. Dann ist jeder positive Operator $A\colon X\to Y$ beschränkt.
\end{satz}

\begin{proof}
Angenommen, $A$ ist nicht beschränkt. Dann gibt es eine Folge $(x_n)_{n\in\N}$ in $X$, welche $\|x_n\|=1$ und $\|Ax_n\|\geq n^3$ für alle $n\in\N$ erfüllt. Da $X$ vollständig ist, wissen wir  $x:=\sum_{n=1}^\infty n^{-2}|x_n|\in X$. Mit $0\leq |x_n|/n^2\leq x$ erhalten wir 
\begin{equation*}
n \leq \|A(x_n/n^2)\| \leq \|A(|x_n|/n^{2})\|\leq \|A|x|\|<\infty,\quad\forall n\in\N.
\end{equation*}
Dies ist ein  \textbf{Widerspruch}.
\end{proof}

\begin{prop}\label{Norm positiver Operatoren}
Sei $A$ positiver Operator auf einem Banachverband $X$. Dann ist die Operatornorm von $A$ bereits gegeben durch:
\begin{equation*}
\|A\|=\sup_{x\geq 0, \| x \| \leq 1} \|Ax\|.
\end{equation*}
\end{prop}

\begin{proof}
Zu "$\geq$": Klar, da das Supremum über eine kleinere Menge genommen wird, also ist $\| A\| = \sup_{\|x\|\leq 1}\|Ax\|\geq\sup_{x\geq 0,\|x\|\leq 1}\|Ax\| = \|Ax\|$.

\par 
Zu "$\leq$": Sei $x\in X$ mit $\|x\|\leq 1$. Dann ist $|x| = x_+ + x_-\geq0$. Da $A$ positiv ist, wissen wir $A|x|\geq |Ax|$ und somit $\|A|x|\|\geq \||Ax|\|=\|Ax\|$. Folglich ist $\sup_{\|x\|\leq 1}\|Ax\|\leq\sup_{\|x\|\leq 1}\|A|x|\|=\sup_{x\geq 0,\|x\|\leq 1}\|Ax\|$.
\end{proof}

\begin{bem}
Sind $A$ und $B$ positive Operatoren auf $X$, so schreibe $A\leq B$, falls $B-A\geq0$ gilt.
\end{bem}

\begin{bem}
Für positive Operatoren $A, B$ von Banachverbänden gilt stets:
\begin{equation*}
A\leq B\Rightarrow \|A\|\leq \|B\|. 
\end{equation*}
Wegen $\|B-A\|\geq0$ erhalten wir mit der Dreiecksungleichung, dass $\|B\| + \|A\| \geq 0$ für alle $x\geq0$ gilt, somit bereits für alle $x\in X$.
\end{bem}

\begin{bem}
Gibt es ein $K\in \R$ mit $\|Ax\|\leq K\|x\|$ für alle $x\geq0$ für einen positiven Operator $A$ auf einem Banachverband, so gilt dies bereits für alle $x\in X$: [...]
\end{bem}




\begin{defi} Für einen beschränkten Operator $A$ in einem (komplexen) Banachraum $X$ definiere:
\begin{enumerate}
\item Die \index{Resolvente!Resolventenmenge}\textbf{Resolventenmenge}  von $A$ ist die Menge $\varrho(A):=\{\lambda \in\mathbb C; (\lambda I-A)^{-1}\in \mathcal L(X)\}$.
\item  Für alle $\lambda\in\varrho(A)$  ist $R(\lambda, A):=(\lambda I-A)^{-1}$ die \index{Resolvente}\textbf{Resolvente}  von $A$ . 
\item Das \index{Resolvente!Spektrum}\textbf{Spektrum} von $A$ ist das Komplement $\sigma(A):=\mathbb C\setminus \varrho(A)$.
\item $r_\sigma (A):=\sup_{\lambda\in \sigma(A)} |\lambda |$ heißt \index{Resolvente!Spektralradius}\textbf{Spektralradius} von $A$.
\end{enumerate}
\end{defi}

\begin{bem}
Das Spektrum $\sigma(A)$ eines beschränkten Operators $A$ wird folgendermaßen unterteilt:
\begin{enumerate}
\item Das \textbf{Punktspektrum} $\sigma_p(A)$ ist die Menge aller $\lambda\in\sigma(A)$, für die $\lambda I-A$ nicht injektiv ist.
\item Das \textbf{Residualspektrum} $\sigma_r(A)$ ist die Menge aller $\lambda\in\sigma(A)$, für die $\lambda I-A$ injektiv ist, aber das $\text{Bild}(\lambda I-A)$ nicht dicht in $X$ liegt.
\item Das \textbf{approximative Spektrum} $\sigma_a(A)$ ist die Menge aller $\lambda\in\sigma (A)$, für die $\lambda I-A$ injektiv ist und das $\text{Bild}(\lambda I-A)$ dicht in $X$ liegt, aber $\overline{\text{Bild}(\lambda I-A)}\neq X$ gilt.
\end{enumerate}
\end{bem}

\begin{defi}
Sei $A$ ein nicht beschränkter Operator in einem Banachraum $X$. Dann ist die  \textbf{Spektralschranke} von $A$ gegeben durch:
\begin{equation*}
s(A):=\sup\{\mathfrak R\lambda; \lambda\in\sigma(A)\}.
\end{equation*}
\end{defi}

\begin{bem}
Ist $A$ Operator in $X$, dann gilt für alle $\lambda,\mu\in \varrho(A)$ die \index{Resolvente!Resolventengleichung}\textbf{Resolventengleichung}:
\begin{equation*}
R(\lambda, A)-R(\mu, A)=(\mu-\lambda)R(\lambda, A) R(\mu, A).
\end{equation*}
[...]
\end{bem}



\section{Etwas Funktionalanalysis}


% \begin{satz}[Neumann'sche Reihe]\label{Satz von der Neumann'schen Reihe}\index{Neumann'sche Reihe}
% Für alle $A\in \mathcal L(X)$ mit $\|A\| < 1$ ist $I-A$ invertierbar und gegeben durch $(I-A)^{-1}=\sum_{k=0}^\infty A^k$.
% \end{satz}


\begin{defi}
Sei $X$ ein (reellwertiger) Banachraum. Eine Funktion $x^*\colon X\to \R$ heißt \textbf{Funktional}. Bezeichne die Menge $X^*:=\mathcal L(X,\R)$ aller stetigen linearen Funktionale als  \textbf{Dualraum} zu $X$. Weiter bezeichne mit $X^{**}:=(X^*)^*$ das \textbf{Bidual} zu $X$. 
\par
\hl{Wir können jedes $x\in X$ mit einem Element von  $X^{**}$ mithilfe der \textbf{dualen Paarung} identifizieren:}
\begin{equation*}
x(x^*):=\langle x^*, x\rangle := x*(x).
\end{equation*}
\par
\hl{Damit lässt sich $X$ als Teilraum von $X^{**}$ auffassen. Für gewöhnlich gilt $X\neq X^{**}$, andernfalls bezeichne $X$ als \textbf{reflexiv}.}
\end{defi}

\begin{bsp}
Sei $1<p<\infty$. Dann können wir das Dual zu $L_p(\Omega)$ mit $L_q(\Omega)$ identifizieren, falls $1/p + 1/q = 1$ gilt. Die duale Paarung ist gegeben durch:
\begin{equation*}
\langle f,g\rangle :=\int_\Omega f(x)g(x)\text dx,\quad\forall f\in L_p(\Omega),\forall g\in L_q(\Omega).
\end{equation*}
Insbesondere sind die Räume $L_p(\Omega)$ für alle $1<p<\infty$ reflexiv. 
\end{bsp}

\begin{bsp}
Reflexivität von $l_p$.
\end{bsp}

\begin{bsp}
$L_\infty$ nicht reflexiv.
\end{bsp}

\begin{fsatz}[Hahn-Banach]\index{Hahn-Banach}
Sei $X$ ein normierter Vektorraum und $X_0\subseteq X$ ein linearer Teilraum. Ist $x_1^*$ ein stetiges lineares Funktional auf $X_0$, so gibt es ein stetiges lineares Funktional $x^*$ auf ganz $X$ mit $x^*(x)=x_1^*(x)$ für alle $x\in X_0$ und $\|x^*\|=\|x_1^*\|$.
\end{fsatz}

\begin{proof}
Siehe \cite{werner_2007}, Aussage [...]
\end{proof}

\begin{bem}
\hl{Wegen \textbf{Hahn-Banach} ist der Raum $X^*\neq\emptyset$.}
\end{bem}


\begin{lem}
Sei $X$ ein Banachraum und $X^*$ das zugehörige Dual. Dann gibt es für jedes $x\in X$ eine $x^*$ mit $\langle x^*, x\rangle=\|x\|^2 = \|x^*\|^2$.
\end{lem}

\begin{lem}\label{Lemma nach Hahn-Banach}
Sei $0\neq x\in X_+$. Dann gibt es $x^*\in X^*_+$ mit $\|x^*\|= 1$ und $\langle x^*, x\rangle = \|x\|$.
\end{lem}

\begin{proof}
Mit \textbf{Hahn-Banach} gibt es  $x^*\in X^*$ mit $\|x^*\| =1$ und  $\|x\| = \sup_{\|y^*\|\leq 1}\langle y^*, x\rangle = \langle x^*, x\rangle$. Ist $0\neq x^*\not\in X_+^*$, dann gilt
\begin{equation*}
0 < \|x\|=\langle x^*, x\rangle = \langle x_+^*, x\rangle - \langle x_+^*, x\rangle\leq \langle x_+^*, x\rangle
\end{equation*}
mit $\|x_+^*\|\leq \|x^*\|\leq 1$, da $x_+^*\leq |x^*|$. Folglich ist $\langle x_+^*, x\rangle = \langle x^*, x \rangle = \|x\|$. Angenommen,  $\|x^*_+\| < 1$. Dann ist $\|\widetilde{x}^*\|=1$ für $\widetilde{x}^*:= \|x_+^*\|^{-1} x_+^*$  und damit $\langle \widetilde{x}^*, x\rangle> \langle x^*, x\rangle$. \textbf{Widerspruch}. 
\end{proof}


\section{Positive $C_0$-Halbgruppen}

\begin{defi}
Eine Familie $(T(t))_{t\geq0}$ von beschränkten Operatoren in $X$  heißt \index{Operatorhalbgruppe!stark stetig}\textbf{stark stetige Operatorhalbgruppe}, kurz \textbf{$C_0$-Halbgruppe}, falls gilt:
\begin{enumerate}
\item $T(0)=I$.
\item $T(t+s)=T(t)T(s)$ für alle $t,s\geq0$.
\item $\lim_{t\downarrow 0}T(t)x=x$ für alle $x\in X$.
\end{enumerate}
\end{defi}

\begin{bem}
\hl{Ist $(T(t))_{t\geq0}$ eine $C_0$-Halbgruppe, so liefert \textbf{Banach-Steinhaus}, dass diese bzgl. der Operatornorm auf kompakten Intervallen in $\R_+$ stets beschränkt ist.}
\end{bem}

\begin{defi}
Wir bezeichnen eine $C_0$-Halbgruppe $(T(t))_{t\geq0}$ als \textbf{Kontraktionshalbgruppe}, falls gilt:
\begin{equation*}
\|T(t)x\|\leq\|x\|,\quad\forall t\geq0, \forall x\in X.
\end{equation*}
\end{defi}

\begin{bsp}
Sei $X=L_p(I)$ mit $I=\R$. Dann ist die \textbf{Translationshalbgruppe} gegeben durch:
\begin{equation*}
(T(t)f)(s):=f(t+s),\quad \forall f\in X,\quad \forall s,t\in I.
\end{equation*}
Die Halbgruppeneigenschaft für $T\colon I\to X$ ist klar. Für alle $t\geq0$ gilt:
\begin{equation*}
\|T(t)f\|_p^p=\int_I|f(t+s)|^p\text ds\leq\int_I|f(r)|^p \text dr =\|f\|_p^p.
\end{equation*}
Folglich genügt $(T(t))_{t\geq0}$ der Abschätzung $\|T(t)\|\leq 1$.

\par
Zur starken Stetigkeit: Hierfür verwende \Cref{}. Sei dazu $\phi\in C_0^\infty(I)$. Dann ist $\phi$ gleichmäßig stetig (mit kompaktem Träger), folglich gibt es für jedes $\epsilon>0$ eine $\delta>0$ so, dass für alle $s\in I$ und $0<t<\delta$ gilt:
\begin{equation*}
|\phi(t+s)-\phi(s)|<\epsilon.
\end{equation*}
Damit sehen wir
\begin{equation*}
\int_I|\phi(t+s)-\phi(s)|^p\text ds\leq M_\phi \text{exp}(p).
\end{equation*}
\hl{Hierbei sei $M_\phi$ das Maß} [...]. Da $\mathbb C_0^\infty(I)$ für alle $1\leq p<\infty$ dicht in $L_p(I)$ liegt und $(T(t))_{t\geq0}$ kontraktiv ist, liefert \Cref{}, die starke Stetigkeit von $(T(t))_{t\geq0}$.
\end{bsp}

\begin{defi}
Eine Abbildung $A$ in $X$ heißt \index{Operatorhalbgruppe!Generator}\textbf{Generator} einer $C_0$-Halbgruppe $(T(t))_{t\geq0}$, falls gilt:
\begin{equation*}
Ax=\lim_{h\downarrow 0}\frac{T(h)x - x}{h},\quad x\in D(A).
\end{equation*}
Hierbei sei $D(A)$ die Menge aller $x\in X$, für die $\lim_{h\downarrow 0} h^{-1}(T(h)x-x)$ existiert. Dann ist $(A, D(A))$ Operator. \hl{Wir bezeichnen eine $C_0$-Halbgruppe, welche $(A, D(A))$ erzeugt wird, mit $(T_A(t))_{t\geq0}$. }
\end{defi}

\begin{bem}
Für eine $C_0$-Halbgruppe $(T(t))_{t\geq0}$ gibt es stets $M>0$ und $\omega\in \R$ derart, dass für alle $t\geq0$ gilt:
\begin{equation}\label{Abschätzung}
\|T(t)\|_X\leq M\text{exp}(\omega t).
\end{equation}
Schreibe $A\in \mathcal G(M,\omega)$, falls $A$ Generator einer $C_0$-Halbgruppe $(T_A(t))_{t\geq0}$ ist, welche \ref{Abschätzung} erfüllt.
\end{bem}

\begin{proof}
Siehe \cite{engel_nagel_2006}, [...].
\end{proof}

% Ist $(T_A(t))_{t\geq0}$ eine Halbgruppe mit Generator $A$, so gelten folgende Eigenschaften, welche wir im Folgenden Text verwenden (siehe \cite{engel_nagel_2006}):  

\begin{prop}Für jede$C_0$-Halbgruppe $(T_A(t))_{t\geq0}$ gilt:
\begin{enumerate}
\item Für alle $x\in X$ ist $\lim_{h\to 0}\frac{1}{h}\int_t^{t+h} T_A(s)x\textnormal ds = T_A(t)x$.
\item [...]
\end{enumerate}
\end{prop}

\begin{proof}
Siehe \cite{engel_nagel_2006}, [...].
\end{proof}

% \par
% Weiter ist, falls $A$ Generator einer positiven Halbgruppe $(T_A(t))_{t\geq0}$ auf einem Banachverband $X$ ist, dass $\lambda\in\varrho(A)$ für alle $\mathfrak R(\lambda)> \sigma(A)$ und es gilt die \index{Resolvente!Integraldarstellung}\textbf{Integraldarstellung der Resolvente} (siehe \cite{banasiak_arlotti_2006}, Aussage 3.34) mit
% \begin{equation*}\label{eq:}
% R(\lambda, A)x=\int_0^\infty \exp(-\lambda t)T_A(t)x\textnormal dt,\quad \forall x\in X
% \end{equation*}


\section{Hille-Yosida}

\begin{fsatz}[Hille-Yosida]\label{Hille-Yosida}\index{Hille-Yosida}
Für einen Operator $(A, D(A))$ ist $A\in \mathcal G(M,\omega)$ genau dann, wenn folgende Aussagen gelten:
\begin{enumerate}
\item $(A, D(A))$ ist abgeschlossen und dicht definiert.
\item Es gibt $M> 0$ und $\omega \in\mathbb R$ mit $(\omega, \infty)\subseteq \varrho(A)$ und es gilt:
\begin{equation*}
\|(\lambda I- A)^{-n}\|\leq\frac{M}{ (\lambda - \omega)^{-n}},\quad\forall n\in\N,\lambda > \omega. 
\end{equation*}
\end{enumerate}
\end{fsatz}

\begin{proof}
Siehe \cite{engel_nagel_2006}, [...].
\end{proof}

\begin{bem}\label{Toll, Hille}
Ist $A$ Generator von $(T_A(t))_{t\geq0}$, so gilt:
\begin{equation*}
R(\lambda, A)x = \int_0^\infty \exp(-\lambda t)T_A(t)x\text dt,\quad  \forall x\in X, \mathfrak R \lambda >\omega.
\end{equation*}
\end{bem}

\begin{proof}
[...]
\end{proof}

\begin{satz}\label{Darstellung von T}
Ist $A$ Generator von  $(T_A(t))_{t\geq0}$, so gilt:
\begin{equation*}\label{Darstellung der Gruppe mithilfe der Resolvente}
T_A(t)x = \lim_{n\to\infty}\Big(I- \frac t n A\Big)^{-n}x = \lim_{n\to\infty} \Big(\frac n t R\Big(\frac n t, A\Big)\Big)^n x,\quad\forall x\in X.
\end{equation*}
Hierbei gilt  auf allen kompakten Intervallen bereits gleichmäßige Konvergenz.
\end{satz}

\begin{proof}
Siehe \cite{pazy_1983}, Theorem 1.4.3.
\end{proof}

% \begin{bsp}
% Ist $A$ Generator einer $C_0$-Halbgruppe $(T_A(t))_{t\geq0}$, so auch  $B:=aA+b$ mit $a>0$ und $b\in\mathbb C$: Es gilt
% \begin{equation}
% R(\lambda, B)=\frac{1}{a}R\Big(\frac{\lambda-b}{a}, A\Big)
% \end{equation}
% und 
% \end{bsp}

% \par
% Mithilfe des Satzes von \textbf{Hille-Yosida}, können wir Generatoren \index{Operatorhalbgruppe!kontraktiv}\textbf{kontraktiver} Halbgruppen, falls also $\|T(t)x\|\leq 1$ für alle $x\in X$ gilt, charakterisieren. Demnach ist $A$ genau dann Generator einer stark stetigen Kontraktionshalbgruppe, wenn $A$  abgeschlossen, dicht definiert mit $\sigma(A)\leq 0$ ist und es gilt $\|R(\lambda, A)\|\leq \mathfrak R(\lambda)^{-1}$ für alle $\mathfrak R(\lambda) > 0$.

\begin{defi}
Sei $X$ Banachverband. Dann bezeichne eine $C_0$-Halbgruppe $(T(t))_{t\geq0}$ auf $X$ als \textbf{positiv}, falls für alle $x\in X_+$ gilt:
\begin{equation*}
T(t)x\geq0, \quad t\geq 0.
\end{equation*}
\end{defi}

\begin{defi}
Ein Operator $(A,D(A))$ heißt \textbf{resolventenpositiv}, falls es $\omega\in\R$ mit $(\omega,\infty)\subseteq \varrho(A)$ gibt mit:
\begin{equation*}
R(\lambda, A)\geq 0, \quad \forall \lambda >  \omega.
\end{equation*}
\end{defi}

\begin{bem}
Eine $C_0$-Halbgruppe $(T_A(t))_{t\geq0}$ ist genau dann positiv, wenn es ein $\lambda\in\R$ hinreichend groß gibt, für dass $(A, D(A))$ resolventenpositiv ist:

\par
\hl{Die Positivität der Resolvente für $\lambda >\omega$ folgt mit \ref{Toll, Hille} und der Abgeschlossenheit des positiven Kegels. Umgekehrt ist mit $(A, D(A))$ resolventenpositiv wegen \ref{Darstellung von T} auch  $T_A(t)$ positiv.}
\end{bem}

\begin{satz}
Sei $(T_A(t))_{t\geq0}$ eine positive $C_0$-Halbgruppe auf einem Banachverband $X$ mit Generator $(A, D(A))$. Dann gilt für alle $\lambda\in\mathbb C$ mit $\mathfrak R\lambda > s(A)$:
\begin{equation*}
R(\lambda, A)x=\int_0^\infty \textnormal{exp}(-\lambda t)T_A(t)x\textnormal dt,\quad\forall x\in X.
\end{equation*}
Weiter gilt:
\begin{enumerate}
\item Es ist entweder $s(A)=-\infty$ oder $s(A)\in\sigma(A)$.
\item Sei $\lambda \in \varrho(A)$. Dann gilt $R(\lambda, A)\geq0$ genau dann, wenn $\lambda > s(A)$.
\item Für alle $\mathfrak R\lambda >s(A)$ und $x\in X$ gilt $|R(\lambda, A)x|\leq R(\mathfrak \lambda, A)|x|$.
\end{enumerate}
\end{satz}

\begin{proof}
[...] Siehe \cite{banasiak_arlotti_2006}, Theorem 3.34.
\end{proof}

\begin{bsp}
\hl{Betrachte die positive  \hl{Translationshalbgruppe} $(T(t))_{t\geq0}$ auf $[0,1]$ aus . Für $t>1$ gilt dann $T(t)f=0$ für alle $f\in X$ und damit  ist  \hl{$\omega_1(T)=-\infty$}. Folglich ist $s(A)=-\infty$ und $\sigma(A)=\emptyset$.}
\end{bsp}

\section{Dissipative Operatoren}

\begin{defi}
Sei $X$ ein Banachraum. Die \textbf{Dualmenge} von $x\in X$ ist gegeben durch:
\begin{equation*}
\mathcal J(x):=\{x^*\in X^*; \langle x^*, x\rangle=\|x\|^2 = \|x^*\|^2\}.
\end{equation*}
\end{defi}

\begin{bem}
Wegen \Cref{} ist $\mathcal J(x)\neq \emptyset$ für alle $x\in X$.
\end{bem}


\begin{defi}
Ein Operator $(A, D(A))$ heißt \index{Operator!dissipativ}\textbf{dissipativ}, falls es für alle $x\in D(A)$ ein $x^*\in \mathcal J(x)$ gibt mit:
\begin{equation*}
\mathfrak R\langle Ax, x^*\rangle \leq 0.
\end{equation*}
\end{defi}

\begin{prop}\label{Charakterisierung der Dissipativität}
Für einen Operator $(A, D(A))$ sind äquivalent:
\begin{enumerate}
\item $A$ ist dissipativ.
\item Für alle $x\in D(A)$ und $\lambda >0$ gilt $\|(\lambda I- A)x\|\geq \lambda \|x\|$.
\end{enumerate}
\end{prop} 

\begin{proof}
Siehe  \cite{pazy_1983}, Aussage 4.2.
\end{proof}

\begin{fsatz}[Lumer-Phillips]
Sei $(A, D(A))$ ein dicht definierter, dissipativer Operator in einem Banachraum $X$. Dann sind äquivalent:
\begin{enumerate}
\item $\overline{A}$ ist Generator einer Kontraktionshalbgruppe $(T_{\overline{A}} (t))_{t\geq0}$.
\item Es gibt $\lambda>0$ mit $\overline{\textnormal{Bild}(\lambda I- A)}=X$.
\end{enumerate}
\hl{In jedem Falle ist $\mathfrak R\langle x^*, Ax\rangle\leq 0$ bereits für alle $x^*\in\mathcal J(x)$ erfüllt.}
\end{fsatz}

\begin{proof}
Siehe \cite{engel_nagel_2006}, [...].
\end{proof}

\begin{bsp}
Der Differentialoperator $T_1$ aus \Cref{} ist dicht definiert und mit \ref{} dissipativ. Folglich ist $T_1$ Generator einer Kontraktionshalbgruppe in $L_p([0,1])$. [...]
\end{bsp}

\begin{bsp}
Sei $(A, D(A))$ Operator mit $\overline{D(A)}=X$. Ist sowohl $A$ als auch die Adjungierte $A^*$ dissipativ, dann ist $\overline A$ Generator einer Kontraktionshalbgruppe in $X$: [...]
\end{bsp}

\section{Trotter-Kato} %Approximation von $C_0$-Halbgruppen}


\begin{defi}
Sei $X$ Banachraum und $\Delta\subseteq\mathbb C$ eine Teilmenge. Eine Familie $\{J(\lambda)\}_{\lambda\in\Delta}$ beschränkter Operatoren in $X$ wird als \textbf{Pseudoresolvente} auf $\Delta$ bezeichnet, falls gilt:
\begin{equation*}
J(\lambda)-J(\mu)=(\mu-\lambda)J(\lambda)J(\mu),\quad\forall\lambda,\mu\in\Delta.
\end{equation*}
\end{defi}

\begin{satz}
Sei $\{J(\lambda)\}_{\lambda\in\Delta}$ eine Pseudoresolvente auf $\Delta\subseteq\mathbb C$ in $X$ Banachraum. Dann gilt:
\begin{enumerate}
\item Es sind $\textnormal{Bild}(J(\lambda))$ und $\textnormal{Kern}(J(\lambda))$ von $\lambda\in\Delta$ unabhängig.
\item $J(\lambda)$ ist die Resolvente eines eindeutig bestimmten, abgeschlossenen Operator $A$ in $X$ genau dann, wenn $\textnormal{Kern}(J(\lambda))=\{0\}$ und $\overline{\textnormal{Bild}(J(\lambda))}=X$ gilt. 
\end{enumerate}
\end{satz}

\begin{proof}
Zu (1): Betrachte $J(\lambda)=J(\mu)(I+(\mu-\lambda)J(\lambda))$. Dann ist $\text{Bild}J(\lambda))\subseteq \text{Bild}(J\mu))$. Die umgekehrte Inklusion folgt, indem wir $\lambda$ und $\mu$ vertauschen. Vergleichsweise ist $J(\lambda)=(I+(\mu-\lambda)(J\lambda))J(\mu)$ erfüllt, was $\text{Kern}(J(\lambda))\supseteq \text{Kern}(J(\mu))$ liefert. Die umgekehrte Inklusion folgt wieder mit dem Austausch von  $\lambda$ und $\mu$.

\par
Zu (2): Die Hinreichtung ist klar. Sei also $\text{Kern}J(\lambda)=\{0\}$ und $\text{Bild}(J(\lambda))$ dicht in $X$. Dann ist $J(\lambda)$ injektiv und es gibt $\lambda_0\in\Delta$ so, dass $A:=\lambda_0I - J(\lambda_0)^{-1}$ wohldefiniert ist. Damit ist $A$ linear und abgeschlossen und wegen $D(A)=\text{Bild}(J(\lambda_0)$ dicht definiert mit $\overline {D(A)}=X$. Nach Konstruktion ist $R(\lambda_0, A) = J(\lambda_0)$. Für beliebiges $\lambda\in\Delta$ gilt:
\begin{align*}
(\lambda I- A)J(\lambda)
&=((\lambda-\lambda_0)I+(\lambda_0-A))J(\lambda)\\
&=((\lambda-\lambda_0)I+(\lambda_0-A))J(\lambda_0)(I+(\lambda_0-\lambda)J(\lambda))\\
&=(\lambda-\lambda_0)J(\lambda_0)(I+(\lambda_0-\lambda)J(\lambda))+I+(\lambda_0-\lambda)J(\lambda)\\
&=I + (\lambda-\lambda_0)(J(\lambda_0)- J(\lambda)- (\lambda-\lambda_0)J(\lambda)J(\lambda_0))=I.
\end{align*}
Ebenso erhalten wir:
\begin{align*}
J(\lambda)(\lambda I- A)
&=(I+(\lambda_0-\lambda)J(\lambda))J(\lambda_0)((\lambda-\lambda_0)I+(\lambda_0 I- A))\\
&= (I+(\lambda_0 - \lambda)J(\lambda))((\lambda-\lambda_0)J(\lambda_0)+I)\\
&=I+(\lambda_0-\lambda)(-J(\lambda_0)+J(\lambda)+(\lambda-\lambda_0)J(\lambda)J(\lambda_0))=I.
\end{align*}
Damit ist $J(\lambda)=R(\lambda, A)$ für alle $\lambda\in\Delta$. \hl{Insbesondere ist $A$ unabhängig von $\lambda$ und durch $J(\lambda)$ eindeutig bestimmt}
% Siehe \cite{banasiak_arlotti_2006}, Theorem 3.41.
\end{proof}

\begin{folg}
Sei $\Delta\subseteq \mathbb C$ unbeschränkt und $J(\lambda)$ eine Pseudoresolvente auf $\Delta$. Angenommen, es gibt eine Folge $(\lambda)_{n\in\N}$ in $\mathbb C$ mit $|\lambda|\to\infty$ so, dass eine der beiden Aussagen gilt:
\begin{enumerate}
\item $\|\lambda_nJ(\lambda_n)\|\leq M$ für ein $M<+\infty$ und $\overline{\textnormal{Bild}(J(\lambda))}=X$.
\item $ \lim_{n\to\infty}\lambda_n J(\lambda_n)x=x$ für alle $x\in X$.
\end{enumerate}
Dann ist $J(\lambda)$ die Resolvente eines eindeutig bestimmten, dicht definierten und abgeschlossenen Operator $A$ in $X$.
\end{folg}

\begin{proof}
Mit (i): Wir müssen $\text{Kern}(J(\lambda))=\{0\}$ zeigen. Wegen $|\lambda_n|\to\infty$ ist $\|J(\lambda_n)\|\to 0$ für $n\to\infty$.  Betrachte
\begin{equation*}
    J(\lambda_n)-\mu J(\mu)J(\lambda_n)=J(\mu)-\lambda_n J(\lambda_n)J(\mu),\quad\forall \mu\in\Delta.
\end{equation*}
Dann liefert  dies $\lim_{n\to\infty}\|(\lambda_n J(\lambda_n)-I)J(\mu)\|=0$. Für $x\in \text{Bild}(J(\mu))$ ist hiermit $\lim_{n\to\infty}\lambda_n J(\lambda_n)x=x$ erfüllt. Da $\text{Bild} J(\mu)$ dicht in $X$ liegt und $\lambda_n J(\lambda_n)$ gleichmäßig beschränkt ist, gilt $\lambda_n J(\lambda_n)x\to x$ bereits für alle $x\in X$. Sei nun $x\in \text{Kern}(J(\mu))$. Da $\text{Kern}(J(\lambda))$ unabhängig von $\lambda$  und $\lambda_n J(\lambda_n)x=0$ für alle $n\in\N$ ist, folgt $x=0$.

\par
Mit (ii): \hl{Wir wissen, dass der Teilraum $\text{Bild}(\mu)$ unabhängig von $\mu$ ist und $\lambda_n J(\lambda_n)x\in \text{Bild}(J(\mu))$ für alle $x\in X$ gilt.} [...]
\end{proof}

\begin{fsatz}[Trotter-Kato]\label{Trotter-Kato}\index{Trotter-Kato}
Sei $A_n\in \mathcal G(M,\omega)$ eine Folge von Generatoren. Angenommen, es gibt $\lambda_0$ mit $\mathfrak R\lambda_0 >\omega$ und weiter  gilt:
\begin{enumerate}
\item $\lim_{n\to\infty}R(\lambda_0, A_n)x =:R(\lambda_0)x$ für alle $x\in X$.
\item $\textnormal{Bild}(R(\lambda_0))$ liegt dicht in $X$.
\end{enumerate}
Dann existiert ein eindeutig bestimmter Operator $A\in\mathcal G(M, \omega)$ mit:
\begin{equation*}
R(\lambda_0)x=R(\lambda_0, A)x,\quad\forall x\in X.
\end{equation*}
Sind $(T_n(t))_{t\geq0}$ die von $A_n$ sowie $(T_A(t))_{t\geq0}$ die von $A$ erzeugten $C_0$-Halbgruppen, dann ist $(T_n(t)x)_{n\in\N}$ für alle $x\in X$ gleichmäßig auf kompakten Intervallen konvergent mit Grenzwert:
\begin{equation*}
\lim_{n\to\infty}T_n(t)x=T_A(t)x,\quad\forall x\in X.
\end{equation*}
\end{fsatz}

\begin{proof}
Siehe \cite{banasiak_arlotti_2006}, Aussage  3.43. [...]
\end{proof}


 %\section{Adjungierte Operatoren}

% \hl{Eine wichtige Rolle in Funktionalanalysis wir gespielt bei dem Operator welcher den Adjungnierten Operator als Argument nimmt. Wenn $A\in\mathcal L(X,Y)$, dann ist der adjungierte Operator $A^*$ definiert durch}
% \begin{equation*}
% \langle y^*, Ax\rangle =\langle A^*y*, x\rangle
% \end{equation*}
% und man kann zeigen, dass $A^*\in\mathcal L(Y^*, X^*)$ mit $\|A^*\|=\|A\|$.

% \par 
% Jedoch, wenn $D(A)$ dicht definiert in $X$ ist, dann gibt es einen eindeutigen maximalen Operator $A^*$ adjunigert zu $A$; das heißt, für jeden anderen Operator $B$ sodass $A$ und $B$ adjunigert zueinander sind, muss gelten $B\subseteq A^*$.  Dieser Operator $A^*$  ist genannt der Adjunigerte Operator zu $A$.





% \begin{proof}
% Wir nehmen $\omega=0$ an. Zeige zunächst, dass die Konvergenz für alle $\lambda$ mit $\text{Re}(\lambda)>0$ existiert. Sei dazu $S$ die Menge alle $\lambda$, für die $(R(\lambda, A_n)x)_{n\in\N}$ konvergiert. Wähle $\mu\in S$ und stelle $R(\lambda, A_n)$ mithilfe der \index{}\textbf{Taylor-Reihe} um $\mu$ dar, also
% \begin{equation}
% R(\lambda, A_n)=\sum_{k=0}^\infty (\mu-\lambda)^k R(\mu, A_n){k+1}.
% \end{equation}
% Wegen $A_n\in\mathcal G(M,\omega)$ ist weiter
% \begin{equation}
% \|R(\mu, A_n)^k\|\leq M\text{Re}(\mu){-k},
% \end{equation}
% also konvergiert die Reihe für alle $\lambda$ mit $|\lambda-\mu|
% \end{proof}

\hl{Nach ist folgende Behauptung hinreichend für Bedingung (2)in \textbf{Trotter-Kato}:}

\begin{folg}\label{Hinreichende Bedingung für Trotter-Kato}
Angenommen, für alle $x\in X$ ist 
\begin{equation*}
\lim_{\lambda\to\infty}\lambda R(\lambda, A_n)x=x
\end{equation*}
gleichmäßig konvergent in $n$. Dann ist $R(\lambda)$ Resolvente eines dicht definierten abgeschlossenen Operator in $X$.
\end{folg}

\begin{proof}
% Siehe \cite{banasiak_arlotti_2006}, Aussage 3.44.
Ist der Grenzwert $\lim_{\lambda\to\infty}\lambda R(\lambda, A_n)x=x$ gleichmäßig in $n\in\N$, so gibt es für alle $\epsilon>0$  ein $\lambda_0$ so, dass für alle $\lambda > \lambda_0$ und $n\in\N$ gilt:
\begin{equation*}
\|\lambda R(\lambda, A_n)x-x\|\leq \epsilon.
\end{equation*}
Mit gleichmäßiger Konvergenz ist somit $\|\lambda R(\lambda)x-x\|\leq \epsilon$ erfüllt. Dies entspricht gerade \Cref{}, Aussage (2).
\end{proof}



% \begin{prop}\cite{}\label{Gleichmäßige Stetigkeit von Halbgruppen auf Kompakta}
% Sei $\big(T(t)\big)_{t\geq0}$ Halbgruppe. Dann gilt für alle Kompakta $K\subseteq \mathbb R_{\geq0}$, dass die Einschränkung $\big(T(t)\big)|_K$ gleichmäßig beschränkt ist.
% \end{prop}

% \begin{prop}\cite{}\label{Charakterisierung der starken Stetigkeit}
% Sei $\big(T(t)\big)_{t\geq0}$ Halbgruppe. Dann sind äquivalent:
% \begin{enumerate}
% \item $T$ ist stark stetig.
% \item Für $t\downarrow 0$ und $x\in X$ gilt $T(t)x\to x$.
% \item $T$ ist auf einem Kompaktum $[0,\delta]$ beschränkt und es gibt eine dichte Teilmenge $U\subseteq X$ mit $T(t)x\to x$ für alle $x\in U$.
% \end{enumerate}
% \end{prop}

% \begin{prop}\cite{}\label{satz vom abgeschlossenen Graphen}
% Seien $X, Y$ Banachräume, $(A, D(A))$ Operator. Ist $A$ abgeschlossen mit $D(A)=X$, dann ist $A$ stetig.
% \end{prop}

% \begin{proof}
% Setze $Y:=D(A)$. Da $\text{Bild}(A)\subseteq Y$ linearer Teilraum ist, haben wir $G(A)\subseteq X\times Y$ linearer Teilraum. Da $A$ abgeschlossen,  ist $G(A)$ abgeschlossen und insgesamt ist $G(A)$ Banachraum. Setze \begin{equation}\label{eq:}
% \pi_X\colon G(A)\to X,\quad \pi_Y\colon G(A)\to Y.
% \end{equation}
% Dann sind die Projektionen $\pi_X, \pi_Y$ stetig sowie $\pi_X$ bijektiv mit $D(A)=X$. Mithilfe \index{}\textbf{Banachs Homomorphiesatz} ist die Stetigkeit von $\pi_X^{-1}\colon X\to G(A)$. Dann folgt die Stetigkeit von $A$ wegen $A=\pi_Y\circ \pi_X^{-1}$.
% \end{proof}

% \begin{prop}\cite{}\label{Hinreichende Bedingung für Abgeschlossenheit der Inverse eines Operators}
% Sei $X$ normiert Vektorraum sowie $(A, D(A))$ injektiv und abgeschlossener Operator. Dann ist $A^{-1}$ ebenfalls abgeschlossen.
% \end{prop}

% \begin{proof}
% Sei $y_n\in\text{Bild}(A)$ Folge mit $y_n\to y$ und $D(A)\ni A^{-1}y_n \to x$ mit $x\in X$. Mit $A$ injektiv setze $x_n:=A^{-1} y_n$ und es folgt $Ax_n = y_n\to y$, d. h. $x\in D(A)$ sowie $y = Ax\in\text{Bild}(A)=D(A^{-1})$, also auch $x=A^{-1}y$.
% \end{proof}

% \begin{lem}\cite{}\label{Punktweise Konvergenz von Folgen und Operatoren}
% Sei $X$ Banachraum sowie $(A_n)_{n\in\N}$ Folge in $\mathcal L(X)$. Ist $A_n\to A$ punktweise, dann gilt für jede Folge $(x_n)_{n\in\N}$ in $X$ mit $x_n\to x$, dass $A_n(x_n)\to A(x)$.
% \end{lem}

% \begin{proof}
% Es ist $(A_n)_{n\in\N}$ punktweise beschränkt und mit dem \index{}\textbf{PGB} ist $(A_n)_{n\in\N}$ gleichmäßig beschränkt. Wegen $\sup_n\|A_n\|<\infty$ gibt es $C\in \R$ mit 
% \begin{align}
% \|A_n(x_n)-A(x)\|
% &\leq\|A_n(x_n)-A_n(x)\|+\|A_n(x)-A(x)\|\\
% &\leq C\cdot \|x_n-x\| + \|A_n(x)-A(x)\|\to 0.
% \end{align}
% \end{proof}

% \begin{defi}[Beschränktheit]
% Sei $B\subseteq \mathcal L(X, Y)$. Dann heißt
% \begin{enumerate}
% \item $B$ \index{}\textbf{punktweise beschränkt}, falls $\{\|Ax\};A\in B\}$ für alle $x\in X$ beschränkt ist.
% \item $B$ \index{}\textbf{gleichmäßig beschränkt}, falls $\{\|A\|; A\in B\}$ beschränkt ist.
% \end{enumerate}
% \end{defi}

% \begin{satz}[PGB]\cite{banasiak_arlotti_2006}\label{Satz von PGB}
% Sei $X$ Banachraum sowie $Y$ normiert. Dann sind für jede Teilmenge $B\subseteq\mathcal L(X,Y)$ äquivalent:
% \begin{enumerate}
% \item $B$ ist punktweise beschränkt.
% \item $B$ ist gleichmäßig beschränkt.
% \end{enumerate}
% \end{satz}


% \begin{prop}\label{Charakterisierung der punktweisen Konvergenz beschränkter Operatoren}
% Sei $(F_n)_{n\in\N}$ Folge in $\mathcal L(X,Y)$. Dann sind äquivalent:
% \begin{enumerate}
% \item $(F_n)_{n\in\N}$ ist punktweise konvergent auf $X$.
% \item $\{F_n; n\in\N\}$ ist beschränkt und es gibt $W\subseteq X$ dicht mit $F_n$ punktweise konvergent auf $U$.
% \end{enumerate}
% \end{prop}


% \begin{satz}[Offene Abbildung]\cite{banasiak_arlotti_2006}\label{Satz von der offenen Abbildung}
% Seien $X$, $Y$ Banachräume sowie $A\in\mathcal L(X,Y)$. Ist $A$ surjektiv, dann ist $F$ bereits offen.
% \end{satz}

% \begin{proof}

% \end{proof}

% \begin{satz}[Banachs Homomorphiesatz]\cite{banasiak_arlotti_2006}\label{Banachs Homomorphiesatz}
% Seien $X$, $Y$ Banachräume. Ist $A\in\mathcal L(X,Y)$ bijektiv, dann gilt bereits $A^{-1}\in \mathcal L(Y,X)$.
% \end{satz}

% \begin{proof}
% Es ist $F\in L(X,Y)$ klar. Dann folgt die Stetigkeit von $F$ mit dem \index{}\textbf{satz der offenen Abbildung}.
% \end{proof}

% \begin{satz}
%   Sei $A\colon X\to Y$ ein linearer Operator. Dann sind äquivalent:
%   \begin{enumerate}
%       \item $A$ ist stetig.
%       \item $A$ ist stetig für ein $x\in X$.
%       \item $\sup_{\|x\|=1}\|Ax\|<\infty$.
%       \item $A$ ist beschränkt.
%   \end{enumerate}
% \end{satz}

% \begin{prop}
%   Seien $X,Y,Z$ normierte Räume und  $A\in\mathcal L(X,Y)$, $B\in\mathcal L(Y,Z)$ zwei beschränkte lineare Operatoren. Dann ist die \index{}\textbf{Komposition} $BA(x):=B(Ax)$ ebenfalls  ein beschränkter linearer Operator mit $\|BA\|\leq \|B\|\cdot\|A\|$.
% \end{prop}

% \begin{prop}
%   Seien $X,Y$ Banachräume, $x\colon[a,b]\to X$ eine stetige Abbildung und $A\in\mathcal L (X,Y)$ beschränkter linearer Operator. Dann gilt $$A\int_a^b x(t)\text dt=\int_a^b Ax(t)\text dt.$$
% \end{prop}

% \begin{defi}[Normkonvergenz]
%   Eine Folge $(A_n)_{n\in \N}$ in $\mathcal L(X,Y)$ heißt \index{}\textbf{(norm-)konvergent}, falls es $A\in\mathcal L(X,Y)$ gibt mit $\lim_{n\to\infty}\|A_n-A\|=0$.
% \end{defi}

% \begin{prop}
%   Seien $(A_n)_{n\in \N}$ und $(B_n)_{n\in \N}$ zwei konvergente Folgen  in $\mathcal L(X,Y)$. Dann konvergiert $(A_n\cdot B_n)_{n\in\mathbb N}$ mit $\lim_{n\to\infty}A_n B_n = A\cdot B$.
% \end{prop}

% \begin{defi}[Starke Konvergenz]
%   Eine Folge $(A_n)_{n\in\N}$ in $\mathcal L(X,Y)$ heißt \index{}\textbf{stark konvergent}, falls es $A\in\mathcal L(X,Y)$ gibt mit $\lim_{n\to\infty} A_nx = Ax$ für alle $x\in X$.
% \end{defi}

% \begin{prop}
%   Sei $(A_n)_{n\in \N}$ eine Folge in $\mathcal L(X,Y)$. Ist $(A_n)_{n\in\mathbb N}$ konvergent, so gilt bereits starke Konvergenz. Die Umkehrung gilt im Allgemeinen nicht. 
% \end{prop}


% \begin{prop}\cite{}\label{Hintereinanderausführung von Halbgruppen}
% Sind $S, T$ zwei Halbgruppen, dann ist deren Hintereinanderausführung $S\cdot Q$ ebenfalls Halbgruppe.
% \end{prop}

% \begin{proof}
% Sei $(x_n)_{n\in \N}$ konvergente Folge in $\R$ und sei $x\in X$. Definiere $y_n:=T(x_n)(v)$ und $y:=T(t)(v)$ sowie $A_n:= S(x_n)$ und $A:=S(x)$. Dann konvergieren $y_n\to  y$ sowie $A_n\to A$ punktweise. Dann folgt
% \begin{align}
% (S T)(x_n)(x)&=S(x_n)(T(x_n)(x))=A(y_n)\\
% &\to A(y)=S(T(t)(x))=(S T)(x).
% \end{align}
% \end{proof}

\chapter{Störungstheoreme}

\hl{Sei $(A, D(A))$ Generator $C_0$-Halbgruppe in einem Banachraum $X$ und $(B, D(B))$ ein weiterer Operator in $X$. Wir möchten untersuchen, unter welchen Bedingungen die Summe $K=A+B$ ebenfalls ein Generator ist, oder zumindest eine Fortsetzung $K\subseteq A+B$ existiert, welche Generator ist.

\par
Die Addition einer \textbf{Störung} $B$ an den Generator $A$ ändert diesen meist soweit ab, dass $A+B$ nicht mehr erzeugend ist:
\par Sei $(A, D(A))$ ein nicht beschränkter Generator. Für eine Störung $B:=-A$ ist der auf $D(A)$ dicht definierte Operator $A+B=0$ nicht  abgeschlossen, insb. ist $A+B$ nicht Generator einer $C_0$-Halbgruppe.
\par}


\begin{verein}
In den nachfolgenden Abschnitten dieses Kapitels gelte für die Operatoren $(A, D(A))$ und $(B, D(B))$ stets $D(A)\subseteq D(B)$.
\end{verein}

\section{Spektrumskriterien}

\begin{fsatz}
 Sei weiter $K$ eine Fortsetzung von $A+B$, welche Generator einer $C_0$-Halbgruppe in $X$ ist. Angenommen,  $\Lambda:=\varrho(A)\cap \varrho(K)\neq \emptyset$. Dann gilt:
\begin{enumerate}
\item $1\not\in \sigma_p(BR(\lambda, A))$ für alle $\lambda\in\Lambda$.
\item $1\in \varrho(BR(\lambda, A))$ für einige (und damit für alle) $\lambda\in\Lambda$ genau dann, wenn $D(K)=D(A)$ und $K=A+B$ gilt.
\item $1\in\sigma_a(BR(\lambda, A))$ für einige (und damit für alle) $\lambda\in \Lambda$ genau dann, wenn $D(A)\subsetneq D(K)$ und $K=\overline{A+B}$ gilt.
\item $1\in\sigma_r(BR(\lambda, A))$ für einige (und damit für alle) $\lambda\in\Lambda$ genau dann, wenn $K\supsetneq \overline{A+B}$ gilt.
\end{enumerate}
\end{fsatz}

\begin{proof}
Siehe \cite{frosali_van_der_mee_mugelli_2004}, Theorem 3.2.
\end{proof}

\begin{satz}
Sei $X$ Banachraum. Angenommen, es ist mindestens eine der beiden Aussagen erfüllt:
\begin{enumerate}
\item Es gibt $\lambda$ so, dass $BR(\lambda, A)$ kompakt ist. 
\item Es gibt $\lambda$ mit $r(BR(\lambda, A))< 1$.
\end{enumerate}
Dann ist $K=A+B$ Generator einer $C_0$-Halbgruppe in $X$.
\end{satz}

\begin{proof}
Mit (2): Dann ist  $I-BR(\lambda, A)$ nach \Cref{} invertierbar. Die Inverse ist gegeben durch die \textbf{Neumann'sche Reihe}:
\begin{equation*}
(I-BR(\lambda, A))^{-1}=\sum_{n=0}^\infty (BR(\lambda, A))^n.
\end{equation*}
Wegen \Cref{} (2) ist somit $K=A+B$ Generator. Darüber hinaus sehen wir:
\begin{equation*}
R(\lambda, A+B)=R(\lambda, A)(I-BR(\lambda, A))^{-1} =R(\lambda, A)\sum_{n=0}^\infty (BR(\lambda, A))^n.
\end{equation*}

\par
Mit (1): Angenommen, $I-BR(\lambda, A)$ ist nicht invertierbar. Dann ist $1$ ein Eigenwert von $BR(\lambda, A)$. Nach \Cref{} (1) gilt $1\not\in\sigma_p(BR(\lambda, A))$ für alle $\lambda\in\Lambda$. \textbf{Widerspruch}.
\end{proof}

\begin{fsatz}[Störungstheorem beschränkter Operatoren]
Es sei $A\in\mathcal G(M,\omega)$. Ist $B\in\mathcal L(X)$, so ist $K:=A+B$ mit $D(K):=D(A)$ Generator einer $C_0$-Halbgruppe $(T_K(t))_{t\geq0}$ mit $K\in\mathcal G(M, \omega +M\|B\|)$. 
\end{fsatz}

\begin{proof}
Siehe \cite{engel_nagel_2006}, Theorem III.1.3.
\end{proof}

\begin{satz}
Sei Banachraum $X$. Angenommen, es gilt:
\begin{enumerate}
\item $A+tB$ ist dissipativ für alle $0\leq t\leq 1$.
\item \hl{Für alle $x\in D(A)$ und $0\leq a< 1$ gilt $\|Bx\|\leq a\|Ax\| + b\|x\|$.}
\item Es gibt $t_0\in[0,1]$ so, dass $(A+t_0B, D(A))$ Generator einer Kontraktionshalbgruppe in $X$ ist.
\end{enumerate}
Dann ist $K_t:=A+tB$ für alle $t\in[0,1]$ Generator einer Kontraktionshalbgruppe in $X$.
\end{satz}

\begin{proof}
Siehe \cite{} [...]
\end{proof}

\begin{satz}
Sei $X$ Banachraum. Angenommen, es gilt:
\begin{enumerate}
\item $A+tB$ ist dissipativ für alle $0\leq t\leq 1$.
\item Für alle $x\in D(A)$ gilt $\|Bx\|\leq \|Ax\| + b\|x\|$.
\item $B^*$ ist dicht definiert.
\end{enumerate}
Dann ist $K:=\overline{A+B}$ Generator einer Kontraktionshalbgruppe in $X$.
\end{satz}

\begin{proof}
Siehe \cite{} [...]
\end{proof}

\section{Positive Störungen}

\begin{satz}
Sei $X$ Banachverband. Angenommen, es gilt:
\begin{enumerate}
\item $A$ ist resolventenpositiver Operator mit $\lambda > s(A)$.
\item $Bx\geq0$ für alle $x\in D(A)$. 
\end{enumerate}
Dann sind äquivalent:
\begin{enumerate}
\item $r(BR(\lambda, A))<1$.
\item $\lambda\in\varrho(A+B)$ und $(\lambda I-(A+B))^{-1}\geq0$.
\end{enumerate}
In jedem Falle ist dann:
\begin{equation*}
(\lambda I-A-B)^{-1}=(\lambda I-A)^{-1}\sum_{n=0}^\infty (BR(\lambda, A+B))^n\geq(\lambda I-A)^{-1}
\end{equation*}
\end{satz}

\begin{proof}
Siehe \cite{banasiak_arlotti_2006}, Theorem 5.10.
\end{proof}

\begin{fsatz}[Störungstheorem nach Desch]
Sei $X=L_1(\Omega)$. Angenommen, es gilt:
\begin{enumerate}
\item $B$ ist positiver, beschränkter Operator mit $D(B)=D(A)$.
\item Es gibt $\lambda>s(A)$ so, dass $\lambda I-(A+B)$ resolventenpositiv ist.
\end{enumerate}
Dann ist $K:=A+B$ Generator einer positiven $C_0$-Halbgruppe in $X$.
\end{fsatz}

\begin{lem}
Sei $A$ Generator einer positiven $C_0$-Halbgruppe in $X=L_1(\Omega)$.  Ist $B\in\mathcal L(D(A), X)$ positiver Operator und es gibt $\lambda > s(A)$ mit $\|B(\lambda I-A)^{-1}\|< 1$, dann ist $(A+B, D(A))$ Generator einer positiven $C_0$-Halbgruppe in $X$.
\end{lem}

\begin{proof}
Siehe \cite{banasiak_arlotti_2006}, Lemma 5.12.
\end{proof}

\begin{proof}[Beweis des Satzes]
Mit \Cref{} ist $r(BR(\lambda, A))< 1$ und somit die Resolvente von $A+B$ gegeben durch:
\begin{equation*}
(\lambda I-A-B)^{-1}=(\lambda I-A)^{-1}\sum_{n=0}^\infty (BR(\lambda, A+B))^n\geq(\lambda I-A)^{-1}
\end{equation*}
Für $s\in[0,1]$ ersetze $B$ in obiger Abschätzung durch $sB$. Dies ergibt:
\begin{equation*}
(\lambda I-A)^{-1}\leq (\lambda I- A-SB)^{-1}\leq (\lambda I-A-B)^{-1}.
\end{equation*}
Da $B$ positiv ist und das Bild von $(\lambda I-A-B)^{-1}$ dem Definitionsbereich $D(A)$ entspricht, ist der Operator $B(\lambda I-A-B)^{-1}$ beschränkt. Damit gibt es eine Zahl $n\in\N$ so, dass $\|B(\lambda I- A-B)^{-1}\|< n$ gilt. Folglich gilt für alle $s\in[0,1]$, dass $\|n^{-1}B(\lambda I-(A+sB))^{-1}\|<1$. Insb. erhalten wir:
\begin{equation*}
\|n^{-1}B\Big(\lambda I-\Big(A+\frac{j}{n}B\Big)\Big)^{-1}\|<1,\quad\forall  j=0,1,\dots,n-1.
\end{equation*}
[...]
\end{proof}

\section{Störungstheorem in KB-Räumen}

\begin{fsatz}[Störungstheorem nach Kato]\cite{banasiak_lachowicz_2007}\label{Hauptaussage}
Sei $X$ ein reeller $KB$-Raum. Angenommen, es gilt:
\begin{enumerate}
\item $A$ ist Generator einer positiven Kontraktionshalbgruppe  $(T_A(t))_{t\geq0}$.
\item Es gilt $r_\sigma(BR(\lambda, A))\leq1$ für ein $\lambda >0(=s(A))$.
\item $Bx\geq0$ für alle $x\in D(A)_+$.
\item \hl{$\langle x^*, (A+B)x\rangle\leq 0$ für alle $x\in D(A)_+$, wobei  $x^*\in X^*_+$ mit   $\langle x^*, x\rangle = \|x\|$.}
% \begin{align*}
%     \langle x^*, x\rangle = \|x\|\quad\text{und}\quad \langle x^*, (A+B)x\rangle\leq 0.
% \end{align*}
\end{enumerate}
Dann existiert eine Fortsetzung $(K, D(K))$ von $\big(A+B, D(A)\big)$, welche Generator einer positiven Kontraktionshalbgruppe in $X$ ist. Für alle $\lambda >0$ ist $\lambda\in\varrho(K)$ und $R(\lambda, K)$ ist gegeben durch:
\begin{align*}\label{Darstellung der Resolvente von K}
R(\lambda, K)x
%&= \lim_{n\to\infty} R(\lambda, A)\sum_{k=0}^n (BR(\lambda, A))^k x\\
&=\sum_{k=0}^\infty R(\lambda, A)(BR(\lambda, A))^kx,\quad \forall x\in X.
\end{align*}
\end{fsatz}

\begin{bem}\label{stärkere Annahme (2)}
Ist $-A$ positiver Operator, dann kann die Voraussetzungen (2) ersetzt werden durch (2') ''$\|Bx\|\leq \|Ax\|,\quad \forall x\in D(A)_+$'':
\par

\end{bem}

\begin{proof}
\par
Für $\lambda >0$ gilt
\begin{equation*}
0\leq -AR(\lambda, A)=I-\lambda R(\lambda, A)\leq I,
\end{equation*}
also $\|AR(\lambda, A)y\|\leq\|y\|$. Sei $y\in X_+$ und setze $x:=R(\lambda, A)y$. Dann ist $x\in D(A)_+$ und somit \begin{equation*}
\|Bx\|\leq \|Ax\|\leq \|(\lambda I-A)x\|.    
\end{equation*}
Wir erhalten $\|BR(\lambda, A)y\|\leq \|y\|$ und mit $BR(\lambda, A)\geq0$ gilt dies bereits für alle $y\in X$. 
\end{proof}

\begin{bem}\label{BR(lambda, A) fallend}
Ist die Voraussetzung (2) für ein $\lambda_0>0$ erfüllt, so gilt $r_\sigma(BR(\lambda, A))\leq1$ bereits für alle $\lambda > \lambda_0$:
\par
Sei $\lambda > \lambda_0$. Mit der \index{}\textbf{Resolventengleichung} ist
\begin{equation*}
BR(\lambda, A)-BR(\lambda_0, A)=(\lambda_0 - \lambda)BR(\lambda_0, A)R(\lambda, A).
\end{equation*}
Da $BR(\lambda_0, A)$ und $R(\lambda,A)$ positiv sind, gilt mit  \Cref{Fortsetzung positiver Operatoren} $BR(\lambda_0, A)\geq BR(\lambda,A)$.
\end{bem}

\begin{proof}[Beweis des Satzes]
\par
Für $0\leq r < 1$ definiere den Operator $(K_r, D(K_r))$ mit Definitionsbereich $D(K_r):=D(A)$ durch
\begin{equation*}
K_r:= A+rB.
\end{equation*}

\par
Zeige mithilfe von \textbf{Hille-Yosida}, dass $(K_r, D(K_r))$ Erzeuger einer positiven Kontraktionshalbgruppe ist:
\par 
Nach (1) ist $A$ Generator, also ist $K_r$ nach Konstruktion dicht definiert. 

\par 
Zur Existenz und Positivität von $R(\lambda, K_r)$: Für $\lambda>0$ ist
\begin{equation*}
\lambda I - K_r = \lambda I- (A+rB)=(I-rBR(\lambda, A))(\lambda I-A).
\end{equation*}
Wegen $r(rBR(\lambda, A))\leq r< 1$ist $I-rBR(\lambda, A)$ nach invertierbar und die zugehörige Inverse gegeben durch die \textbf{Neumann'sche Reihe} 
\begin{equation*}
(I-rBR(\lambda, A))^{-1}=\sum_{k=0}^\infty r^k (BR(\lambda, A))^k.
\end{equation*}
Damit ist $\lambda\in\varrho(K_r)$ für alle $\lambda>0$ und wir erhalten $R(\lambda, K_r)$ mit
\begin{align*}
R(\lambda, K_r)
&=\big((I-rBR(\lambda, A))(\lambda I-A)\big)^{-1}\\
&=R(\lambda, A)(I-rBR(\lambda, A))^{-1}\\
&=R(\lambda, A)\sum_{k=0}^\infty r^k (BR(\lambda, A))^k.
\end{align*}
Sei $x\in X_+$. Mit (1) ist für $\lambda>0$ hinreichend groß  $R(\lambda, A)x\in D(A)_+$. Nach (3) gilt zudem $BR(\lambda,A)x\geq0$, und damit ist $R(\lambda, K_r)$ positiver Operator auf $X_+$. Mit \Cref{Fortsetzung positiver Operatoren} ist $R(\lambda, K_r)$ bereits auf ganz $X$ definiert.

\par 
Zur Beschränktheit von $R(\lambda,K_r)$: Sei $x\in D(A)_+$. % Mit \mbox{\Cref{Lemma nach Hahn-Banach}} gibt es $x^*\geq0$ mit $\langle x^*, x\rangle = \|x\|$. 
Dann gibt es $x^*\in X_+^*$ mit $\langle x^*, (A+B)x\rangle\leq 0$. Wegen $Bx\geq0$ ist $\langle x^*, Bx\rangle \geq0$. Damit folgt:
\begin{equation*}
\langle x^*, (A+rB)x\rangle = \langle x^*, (A+B)x\rangle + (r-1)\langle x^*, Bx\rangle \leq  0.
\end{equation*}
Nach \Cref{} ist $K_r=A+rB$ dissipativ. Mit \Cref{Charakterisierung der Dissipativität} ist dies äquivalent zu 
\begin{equation*}\label{equation1}
\|(\lambda I - K_r)x\|\geq \lambda \|x\|.
\end{equation*}
Sei nun $y\in X_+$ und setze $x:=R(\lambda, K_r)y\in D(A)_+$. Mit obiger Abschätzung erhalten wir
\begin{equation}\label{Resolvente von K_r ist relativ beschränkt}
\|R(\lambda, K_r)y\|\leq \lambda^{-1}\|y\|.
\end{equation}
Mit \Cref{Fortsetzung positiver Operatoren} gilt dies mit $R(\lambda, K_r)\geq0$  bereits für alle $y\in X$.

% \par
% Zur Abgeschlossenheit von $K_r$: Sei $x_n\in D(K_r)$ mit $x_n\to x$ und $K_rx_n\to y$. Da $R(\lambda, K_r)$ beschränkt ist, ist
% \begin{align*}
% x 
% &= \lim_{n\to\infty} x_n\\
% &=\lim_{n\to\infty}R(\lambda, K_r)(\lambda x_n - K_rx_n)\\
% &=R(\lambda, K_r)(\lambda x-y),
% \end{align*}
% womit $x\in D(K_r)$ und $y=K_r x$.


\par
Somit ist  der Operator  $(K_r, D(K_r))$ mit \textbf{Hille-Yosida} (siehe \Cref{Hille-Yosida}) Generator einer Kontraktionshalbgruppe $(T_r(t))_{t\geq0}$ in $X$, die zudem positiv ist.

\par
Zeige mithilfe von \textbf{Trotter-Kato} (siehe \Cref{Trotter-Kato}), dass der starke Limes von $K_r$ in $r\uparrow 1$ ebenfalls Generator einer Kontraktionshalbgruppe in $X$ ist.

\par
Zu (1) in \Cref{Trotter-Kato}: Sei $x\in X_+$. Für alle  $\lambda >0$ gilt:
\begin{equation*}
R(\lambda, K_r)x=R(\lambda, A)\sum_{k=0}^\infty r^k (BR(\lambda, A))^kx.
\end{equation*}
für $r\uparrow 1$ monoton steigend. Weiter zeigt $\|R(\lambda, K_r)x\|\leq \lambda^{-1}\|x\|$, dass $(\|R(\lambda, K_r)x\|)_{0\leq r<1}$ gleichmäßig beschränkt in $r$ ist. Da $X$ KB-Raum ist, existiert für alle $x\in X_+$ und $\lambda >0$ ein Element $y_{\lambda, x}\in X_+$ derart, dass
\begin{equation}\label{Grenzwert der Resolventen}
\lim_{r\uparrow 1}R(\lambda, K_r)x = y_{\lambda, x}
\end{equation}
gilt. Mit der Linearität des Limes kann das Ergebnis auf ganz $X$ fortgesetzt werden. Sei $R(\lambda)\colon x\mapsto y_{\lambda,x}$ die hierdurch punktweise definierte Abbildung auf $X_+$. Mithilfe der Monotonie und Linearität des Limes ist $R(\lambda)$ bereits auf ganz $X$ positiver Operator.

\par
Zu (2) in \Cref{Trotter-Kato}: Nach \Cref{Hinreichende Bedingung für Trotter-Kato} genügt es, zu zeigen, dass $\lambda R(\lambda, K_r)x\to x$ mit $\lambda\to\infty$ gleichmäßig für alle $x\in X$ konvergiert. Betrachte hierzu:
\begin{equation*}
K_r R(\lambda, K_r)=\lambda R(\lambda, K_r)-I.
\end{equation*}
Sei $x\in D(A)$. Für alle $\lambda>0$ ist:
\begin{align*}
\|\lambda R(\lambda, K_r)x -x\|
&=\|K_r R(\lambda, K_r)x\|\\
&=\|R(\lambda, K_r)K_rx\|\\
&\leq \lambda^{-1}\|(A+rB)x\|\\
&\leq \lambda^{-1}(\|Ax\| + \|Bx\|)
\end{align*}
und damit   $\lim_{\lambda\to\infty}\lambda R(\lambda, K_r)x \to x$ gleichmäßig für alle $x\in D(A)$. Für $y\in X$ beliebig  sei $\epsilon >0$. Wegen $\overline{D(A)}= X$ gibt es ein $x\in D(A)$ mit $\|y - x\|< \epsilon$. Obige Abschätzung liefert:
\begin{align*}
\|\lambda R(\lambda, K_r)y - y\|
&\leq \lambda \|R(\lambda, K_r)(y-x)\| + \|y-x\| + \|\lambda R(\lambda, K_r)x -x\|\\
&\leq 2\epsilon + \lambda^{-1}(\|Ax\| + \|Bx\|).
\end{align*}
Damit erhalten wir gleichmäßige Konvergenz auf ganz $X$.

\par
Mit \textbf{Trotter-Kato} existiert ein eindeutig bestimmter Operator $(K, D(K))$, welcher Generator einer Kontraktionshalbgruppe $(T_K(t))_{t\geq0}$ in $X$ ist. Für $\lambda>0$ ist $\lambda\in\varrho(K)$ und es gilt $R(\lambda, K)=R(\lambda)$. Weiter erhalten wir für alle $x\in X$ die Identität
\begin{equation*}
\lim_{r\uparrow 1} T_r(t)x = T_K(t)x,\quad\forall t\geq0.
\end{equation*}
Insbesondere ist mit der Monotonie des Limes $(T_K(t))_{t\geq0}$ positiv.

\par
Zur Darstellung von $R(\lambda, K)$: Mit \textbf{Trotter-Kato} ist
\begin{align*}
R(\lambda, K)=\lim_{r\uparrow1}R(\lambda, K_r)=\lim_{r\uparrow 1}\sum_{k=0}^\infty r^k R(\lambda, A)(BR(\lambda, A))^k.
\end{align*}
Sei $x\in X_+$. Dann ist $\lim_{r\uparrow 1}\sum_{k=0}^\infty r^k R(\lambda, A)(BR(\lambda, A))^k x\in X$. Mit der Monotonie von $R(\lambda, K_r)$ liefert \Cref{}, dass  
\begin{align*}
R(\lambda, K)x=\sum_{k=0}^\infty R(\lambda, A)(BR(\lambda, A))^kx
\end{align*}
gilt. Mit der Positivität von $R(\lambda, K)$ gilt dies bereits für alle $x\in X$.

\par
Zu $K\supseteq A+B$: Für $\lambda>0$ und  $n\in \mathbb N$ setze
\begin{equation*}
R^{(n)}(\lambda) := \sum_{k=0}^n R(\lambda, A)(BR(\lambda, A))^k.
\end{equation*}
Dies lässt sich umschreiben zu
\begin{align*}
R^{(n)}(\lambda) 
&= R(\lambda, A) + \Bigg(\sum_{k=0}^{n-1}R(\lambda, A) (BR(\lambda, A))^k\Bigg) BR(\lambda,  A)\\
&= R(\lambda, A) + R^{(n-1)}(\lambda) B R(\lambda, A).
\end{align*}
Sei $x\in D(A)$. Dann gilt:
\begin{equation*}
R^{(n)}(\lambda)(\lambda I -A)x = x + R^{(n-1)}(\lambda)Bx.
\end{equation*}
Für $n\to \infty$ erhalten wir $R(\lambda, K)(\lambda I- A)x=x+R(\lambda, K)Bx$, also
\begin{equation*}
R(\lambda, K)(\lambda I -A-B)x=x.
\end{equation*}
Damit ist $x\in D(K)$ mit $(\lambda I -K)x = (\lambda I -A-B)x$.
\end{proof}

% \par
% Die $C_0$-Halbgruppe $(T_K(t))_{t\geq0}$, deren Generator $(K, D(K))$ eine Fortsetzung von $(A+B, D(A))$ ist, ist in folgendem Sinne die "kleinste" $C_0$-Halbgruppe dieser Art:

\begin{satz}\cite{banasiak_arlotti_2006}\label{Minimalität von K}
Es seien  die Voraussetzungen in \Cref{Hauptaussage} gegeben. Ist $D$ Gen von $A$ und $(K', D(K'))$ eine weitere Fortsetzung von $(A+B, D)$, welche Generator einer  positiven Kontraktionshalbgruppe $(T(t))_{t\geq0}$ in $X$ ist, dann gilt:
\begin{equation*}
T(t)\geq T_K(t),\quad \forall t\geq0.
\end{equation*}
\end{satz}

\begin{proof}
\par 
Zeige, dass $(K', D(K'))$ bereits Fortsetzung von $(A+B, D(A))$ ist: Sei  $x\in D(A)$. Dann gibt es eine Folge $(x_n)_{n\in\N}$ in $D$ mit $x_n \to x$ und $Ax_n\to Ax$.\hl{ Wegen $D(A)\subseteq D(B)$ sowie $r_\sigma(BR(\lambda, A))\leq 1$ ist $B$ stetig auf $D(A)$, also $Bx_n\to Bx$}. Wir erhalten somit $(A+B)x_n\to (A+B)x$. Da $K'$ abgeschlossen ist mit $K'x_n  = (A+B)x_n$, folgt $x\in D(K')$ mit  $K'x=(A+B)x$

\par
Zu ''$T(t)\geq T_K(t)$'': Für $\lambda >0$ ist $\lambda\in \varrho(K')$, wobei $R(\lambda, K')$ ist für hinreichend großes $\lambda$ positiv ist. Für alle $0\leq r <1$ gilt dann
\begin{align*}
R(\lambda, K') - R(\lambda, K_r)
&= \big(R(\lambda, K')(\lambda I- K_r)-I\big)R(\lambda, K_r)\\
&= R(\lambda, K')(\lambda I- K_r - \lambda I+  K')R(\lambda, K_r)\\
&= R(\lambda, K')(K' - K_r)R(\lambda, K_r)\\
&= R(\lambda, K')(A+B - A - rB)R(\lambda, K_r)\\
&= (1-r)R(\lambda, K')BR(\lambda,  K_r).
\end{align*}
Da alle beteiligten Terme positiv sind, erhalten wir
\begin{equation*}
R(\lambda, K')\geq R(\lambda, K_r).
\end{equation*}
Mit der Monotonie des Limes wird dies zu
\begin{equation*}
R(\lambda, K')\geq R(\lambda, K) = \lim_{r\uparrow 1}R(\lambda, K_r).
\end{equation*}
\Cref{Darstellung der Gruppe mithilfe der Resolvente} liefert dann $T(t)\geq T_K(t)$ für alle $t\geq0$.
\end{proof}

% \begin{lem}\cite{kato_1954}\label{Eigenschaften von T_K(t)}
% Sei $x\in X_+$. Dann gilt:
% \begin{enumerate}
% \item $\|T_K(t)x\|$ ist  fallende Funktion in $t$.
% \item Es gilt $\lim_{t\to\infty} \|T_K(t)x\| = \|x\| - \lim_{\lambda\downarrow 0} \lim_{n\to\infty}\|(BR(\lambda, A))^n x\|$.
% \item Für $0\neq x$ ist $\lambda^{-1}(\|x\| - \lim_{n\to\infty}\|(BR(\lambda, A))^n x\|)$ eine strikt positive, fallende Funktion in $\lambda$ mit
% \begin{equation*}
% \int_0^\infty \|T_K(t)x\|\textnormal dt = \lim_{\lambda\downarrow 0} \lambda^{-1}(\|x\| - \lim_{n\to\infty}\|(BR(\lambda, A))^n x\|)
% \end{equation*}
% \end{enumerate}
% \end{lem}

\begin{satz}\label{Hauptaussage im Komplexen}
Sei $X_C$ die Komplexifizierung eines $KB$-Raumes $X$. Sind die Voraussetzungen in  \Cref{Hauptaussage} gegeben, dann existiert eine Komplexifizierung $K_C$ von $K$, welche Generator einer positiven Kontraktionshalbgruppe $(T_{K_C}(t))_{t\geq0}$ in  $X_C$ ist.
\end{satz}

\begin{proof}
Siehe [...]
\end{proof}


\par
\begin{bem}
Sei $p\in[1,\infty)$. Dann gilt \Cref{Hauptaussage} nach \Cref{} insbesondere für die Räume $l_p$ und  $L_p(\Omega, \mu)$.
\end{bem}

\section{Störungstheorem in AL-Räumen}

% Sei $X$ AL-Raum. Wegen \cite{aliprantis_burkinshaw_2006}, \Cref{} können wir ohne Einschränkung $X=L_1(\Omega,\mu)$ fordern. \Cref{Hauptaussage} lässt sich in diesem Fall wie folgt schreiben:

\begin{fsatz}[Kato]\label{Hauptaussage in AL}
Sei  $X=L_1(\Omega, \mu)$. Angenommen, es gilt:
\begin{enumerate}
\item $A$ ist Generator einer positiven Kontraktionshalbgruppe $(T_A(t))_{t\geq0}$.
\item $Bf\geq0$ für alle $f\in D(A)_+$.
\item Für alle $f\in D(A)_+$ ist $\int_\Omega (Af+Bf)\textnormal d\mu\leq 0$.
\end{enumerate}
Dann sind die Voraussetzungen in \Cref{Hauptaussage} erfüllt, es existiert also eine Fortsetzung $(K, D(K))$ von $\big(A+B, D(A)\big)$, welche Generator einer positiven Kontraktionshalbgruppe in $X$ ist.
%, es existiert also eine Fortsetzung $(K, D(K))$ von $(A+B, D(A))$, welche Generator einer positiven Kontraktionshalbgruppe $(T_K(t))_{t\geq0}$ in $L_1(\Omega, \mu)$ ist.
\end{fsatz}

\begin{proof}
\par
Wir müssen zeigen, dass die Voraussetzungen (3) und (4) in \Cref{Hauptaussage} erfüllt sind. 


\par
Zu (3): Sei $f\in L_1(\Omega,\mu)_+$. Dann ist $R(\lambda, A)f\in D(A)_+$. Betrachte
\begin{equation*}
(A+B)R(\lambda, A)f=-If+BR(\lambda, A)f+\lambda R(\lambda, A)f.
\end{equation*}
Mit $\int_\Omega (Af+Bf)\textnormal d\mu\leq 0$ für alle $f\in D(A)_+$ erhalten wir
\begin{equation*}
-\int_\Omega If\textnormal d\mu + \int_\Omega BR(\lambda, A)f\textnormal d\mu + \lambda \int_\Omega R(\lambda, A)f\textnormal d\mu\leq 0.
\end{equation*}
Mithilfe der kanonischen Norm in $L_1(\Omega, \mu)$ lässt sich dies umschreiben zu
\begin{equation*}
\lambda \|R(\lambda, A)f\| + \|BR(\lambda, A)f\| - \|f\|\leq 0.
\end{equation*}
Da $\|R(\lambda, A)f\|\leq \lambda^{-1}\|f\|$ wird dies zu
\begin{align*}
\|BR(\lambda, A)f\|\leq\|f\| - \lambda\|R(\lambda, A)f\|\leq \|f\|.
\end{align*}
mit \Cref{} ist dies bereits für alle $f\in L_1(\Omega,\mu)$ erfüllt.

\par
\hl{Zu (4)}: Sei $f\in D(A)_+$.  Mit \Cref{Lemma nach Hahn-Banach} gibt es $f^*\geq0$ mit $\langle f^*, f\rangle=\|f\|$. Damit erhalten wir 
\begin{equation*}
\langle f^*, (A+B)f\rangle=\int_\Omega (Af+Bf)\textnormal d\mu\leq 0.
\end{equation*}

\end{proof}


% \section{Kato's Theorem in $l_1$}

% \begin{lem}\cite{kato_1954}\label{Konstruktion von K_r}
% Sei $0\leq r < 1$. Dann definiert $K_r:=A + rB$ einen Operator mit Definitionsbereich, in Zeichen $(K_r, D(K_r))$, mit $D(K_r) = D(A)$. Für $K_r$ ist:
% \begin{enumerate}
% \item Für alle $\lambda >0$ existiert  $R(\lambda, K_r)$  mit $\|R(\lambda, K_r)\|\leq\lambda^{-1}$.
% \item $R(\lambda, K_r)$ ist positiv.
% \end{enumerate}
% Weiter ist $K_r$  Generator einer positiven Kontraktionshalbgruppe $(T_r(t))_{t\geq0}$.
% \end{lem}

% \begin{proof}
% Der Ausdruck $K_r$ ist als Linearkombination von Operatoren wieder Operator und es gilt $D(K_r)=D(A)\cap D(B)=D(A)$.

% \par Zu (1): Es ist 
% \begin{equation}
% R(\lambda, K_r) = (\lambda I - A - rB)^{-1} = R(\lambda, A) (I-r BR(\lambda, A))^{-1}
% \end{equation}
% und wegen $\|rBR(\lambda, A)\|< 1$ ist mit der \index{}\textbf{Neumann'schen Reihe} \Cref{}
% \begin{equation}
% (I-rBR(\lambda, A))^{-1}=\sum_{k=0}^\infty ( rBR(\lambda, A))^k ,
% \end{equation}
% insgesamt ist $R(\lambda, K_r)$ existent.

% \par Sei nun $x\in X_+$. Mit \Cref{} folgt die Abschätzung
% \begin{align}
% \|(\lambda I - K_r)x\| 
% & =\|(\lambda I -A-rB)x\|\\
% & \geq\|(\lambda I -A)x\| - r\|Bx\|\\
% & = \lambda\|x\| - \|Ax\| - r\|Bx\|\\
% & \geq \lambda \|x\|.
% \end{align}
% Demnach ist $\|R(\lambda, K_r)y\|\leq \lambda^{-1}\|y\|$ für alle $y\in X_+$ und mit \Cref{} gilt dies bereits für alle $y\in X$.

% \par 
% Zu (2): da nach \Cref{}, \Cref{} alle Faktoren der rechten Seite des Ausdrucks
% \begin{equation}
%  R(\lambda, K_r) = R(\lambda, A)\sum_{k=0}^\infty (rBR(\lambda, A))^k ,
% \end{equation}
% positiv sind, ist $R(\lambda, K_r)$ insgesamt positiv.
% \end{proof}

% \begin{lem}\cite{kato_1954}\label{K_r ist Generator}
% Sei $0\leq r< 1$. Dann ist  $(T_r(t))_{r\in[0,1)}$ mit $r\uparrow 1$ für alle $t\geq0$ stark und auf allen Kompakta $t\in[0, s]$ gleichmäßig konvergent. Der Grenzwert $\lim_{r\uparrow 1}T_r(t)=:T_K(t)$ ist positive Kontraktionshalbgruppe und es gilt:
% \begin{enumerate}
% \item $R(\lambda, K)$ existiert mit $\|R(\lambda, K)\|\leq \lambda^{-1}$.
% \item Für alle $r\in[0, 1)$ gilt  $0\leq R(\lambda, K_r)\leq R(\lambda, K)$.
% \item $R(\lambda, K_r)$ ist stark konvergent mit $\lim_{r\uparrow 1}R(\lambda, K_r)=R(\lambda, K)$.
% \end{enumerate}
% Für alle $x$ erfüllt zugehörige Resolvente $R(\lambda, K)$ außerdem
% \begin{equation}
% \lim_{n\to\infty}R(\lambda, K)x = R(\lambda, A)\sum_{k=0}^n ((BR(\lambda, A))^kx.
% \end{equation}
% \end{lem}

% \begin{proof}
% \par
% Zu (1): Sei $t\geq0$. Dann ist für alle $0\leq r< 1$ der Ausdruck $T_r(t)$ wegen \Cref{} beschränkt. Die Monotonie liefert dann den punktweisen Grenzwert $\lim_{r\uparrow 1}T_r(t)=:T_K(t)$.

% \par
% Zu (2): Beweis durch Widerspruch. Angenommen, es gibt $x\in X_+$ sowie $t\geq0$ so, dass $T_r(t)x\to T_K(t)x$ nicht gleichmäßig konvergiert. Dann gibt es konvergente Folgen $(t_n)_{n\in\mathbb N}$ sowie $(r_n)_{n\in\mathbb N}$ mit $t_n\to t_0$ und $r_n\uparrow 1$ mit $\|T(t_n)x - T_{r_n}(t_n)x\|\geq \epsilon$ für ein $\epsilon >0$. Wegen der Positivität der Summanden ist mit \Cref{}, dass 
% $\|T(t_n)x\| - \|T_{r_n}(t_n)x\| = \|T(t_n)x - T_{r_n}(t_n)x\|\geq \epsilon$.

% \par
% Umgekehrt gilt wegen $0\leq T_{r_m}(t_n)x\leq T_{r_n}(t_n)x$ für $m< n$, dass 
% \begin{equation}
% \|T_{r_m}(t_n)x\|\leq\|T_{r_n}(t_n)x\|\leq\|T(t_n) x\|-\epsilon.
% \end{equation}
% Mit der der starken Stetigkeit aus Aussage (1) ist für alle $m$ fest gewählt mit $n\to \infty$
% \begin{equation}
% \|T_{r_m}(t_0)x\|\leq\|T_K(t_0)x\| - \epsilon.
% \end{equation} 
% Es gilt aber $T_r(t_0)x\to T_K(t_0)x$. \index{}\textbf{Widerspruch}.
% \end{proof}

 
% \begin{proof}
% Die Eigenschaften der Positivität, Konktraktivität und Halbgruppeneigenschaft bleiben nach dem Grenzübergang erhalten.

% \par 
% Zur starken Stetigkeit: wegen \Cref{} ist nur Stetigkeit in $t=0$ zu zeigen. Sei $\epsilon >0$, $x\in X_+$. Da nach \Cref{} $T_r(t)$ stark stetig in $t=0$ für $r=0$ ist, gibt es $\delta >0$ mit $\|T_0(t)x -x\|< \epsilon / 2$ für alle $t\in[0,\delta)$. Für derartige $t$ sowie alle $r<1 $ ist  
% \begin{align}
% \|T_r(t)x - T_0(t)x\| 
% &= \|T_r(t)x\| - \|T_0(t) x\|\\
% &\leq \|x\| - \|T_0(t) x\|\\
% &\leq \|x - T_0(t)x\|\\
% &<\epsilon / 2,
% \end{align}
% wobei wir \Cref{} wegen $G_0(t)x\geq 0$ sowie $T_r(t)x - T_0(t)x\geq 0$ nutzen dürfen. Damit
% \begin{equation}
% \|T_r(t)x - x\|\leq \|T_r(t)x - T_0(t)x\| + \|T_0(t)x-x\| <\epsilon,
% \end{equation}
% und nach Grenzübergang für $r\uparrow 1$ für alle $t\in[0\delta)$  ist $\|T_K(t)x-x\| <\epsilon$. Damit ist $T_K(t)x\to x$ für $t\downarrow 0$ für alle $x\in X_+$, also auch für alle $x\in X$.
% \end{proof}

% \begin{proof}
% \par 
% Zu (1): Existenz sowie Abschätzung von $\|R(\lambda, K)\|$ liefert \textbf{Hille-Yosida}.

% \par
% Zu (2): Sei $x\in X$. Dann können wir mit \Cref{} die Differenz $R(\lambda, K)x- R(\lambda, K_r)$ darstellen als
% \begin{equation}\label{Integraldarstellung der Differenz der Resolventen}
% R(\lambda, K)x- R(\lambda, K_r)x=\int_0^\infty \exp (-\lambda t)(T_K(t)x - T_r(t)x)\textnormal dt
% \end{equation}
% und wegen $T_r(t)\leq T_K(t)$ \Cref{} ist $0\leq R(\lambda, K_r)\leq R(\lambda, K)$.

% \par
% Zu (3): zerlege das uneigentliche Integral obiger Darstellung der Differenz durch $\int_0^s + \int_s ^\infty$. Für hinreichend großes $s\geq0$ folgt wegen der uniformen Konvergenz \Cref{} sowie der Beschränktheit \Cref{} für alle $t\in[0,s]$ die starke Konvergenz von  $R(\lambda, K_r)\to R(\lambda, K)$.
% \end{proof}

% \begin{proof}
% Für $r\in[0, 1)$ setze \begin{equation}
% R_r^{(n)}(\lambda):= R(\lambda, A)\sum_{k=0}^n r^k (BR(\lambda, A))^k.   
% \end{equation}
% Mit \Cref{} sowie \Cref{}  ist
% \begin{equation}
% 0\leq R_r^{(n)}(\lambda)\leq R(\lambda, K_r)\leq R(\lambda, K). 
% \end{equation}
% Nach Grenzübergang mit $r\uparrow 1$ ist $R^{(n)}(\lambda)\leq R(\lambda, K)$. Weiter ist  ist $R^{(n)}(\lambda)$ nicht-fallende Funktion in $n$, also existiert der starke Grenzwert und es gilt
% \begin{equation}
% \lim_{n\to\infty} R^{(n)}(\lambda)\leq R(\lambda, K).
% \end{equation}
% Andererseits ist 
% \begin{equation}
% R_r^{(n)}(\lambda)\leq R^{(n)}(\lambda)\leq \lim_{n\to\infty} R^{(n)}(\lambda),
% \end{equation}
% also $R(\lambda, K_r)=\lim_{n\to\infty} R_r^{(n)}(\lambda)\leq \lim_{n\to\infty} R^{(n)}(\lambda)$ und damit auch 
% \begin{equation}
% R(\lambda, K)=\lim_{r\uparrow 1} R(\lambda, K_r)\leq \lim_{n\to\infty} R^{(n)}(\lambda).
% \end{equation}
% \end{proof}

% \begin{lem}\cite{kato_1954}\label{G ist abgeschlossene Fortsetzung von A + B}
% $K$ ist abgeschlossene Fortsetzung von $A + B$.
% \end{lem}

% \begin{proof}
% \par
% Als Generator ist $K$ abgeschlossen, \Cref{}.

% \par 
% Zeige $A +B \subseteq G$. Sei  $x\in D(A) = D(A+B)$. Für $n\in \mathbb N$ können wir mit $BR(\lambda, A)=AR(\lambda, A)$ den Ausdruck $R^{(n)}(\lambda)$ schreiben als
% \begin{align}
% R^{(n)}(\lambda) 
% &= R(\lambda, A) + R(\lambda, A)(\sum_{k=0}^{n-1} (BR(\lambda, A))^k) BR(\lambda,  A)\\
% &= R(\lambda, A) + R^{(n-1)}(\lambda) B R(\lambda, A),
% \end{align}
% also ist
% \begin{equation}
% R^{(n)}(\lambda)(\lambda I -A)x = x + R^{(n-1)}(\lambda)Bx
% \end{equation}
% und nach Grenzübergang für $n\to \infty$ ist mit \Cref{} $R(\lambda, K)(\lambda I -A-B)x=x$. Damit ist $(\lambda I -G)x = (\lambda I -A-B)x$ für alle $x\in D(A)$ und weiter ist damit $x\in D(K)$.
% \end{proof}

% \begin{prop}\cite{kato_1954}\label{T_K(t) ist minimal}
% Sei  $\widetilde G$ Generator einer Kontraktionshalbgruppe $(\widetilde T_K(t))_{t\geq 0}$  mit $Q_0\subseteq \widetilde G$. Dann gilt:
% \begin{enumerate}
% \item $\widetilde G$ ist abgeschlossene Fortsetzung von $A + B$.
% \item Für alle $t\geq 0$ ist $T_K(t)\leq \widetilde T_K(t)$.
% \end{enumerate}
% \end{prop}

% \begin{proof}

% \par 
% Zu (1): wegen \Cref{} ist $\widetilde G$ ebenfalls Fortsetzung von $A+B$. 

% % Da $A$ abgeschlossen ist, \Cref{}, gibt es für alle $z\in D(A)$ eine Folge $(z_n)_{n\in\mathbb N}$ in $D(Q_0)$ mit $z_n\to z$ sowie $Az_n\to Az$. Folglich also auch $\|B(z_n-z)\|\leq \|A(z_n-z)\|\to 0$, d. h. $Q_0 z_n =(A+B)z_n\to (A+B)z$. Da $\widetilde G$ abgeschlossene Fortsetzung von $Q_0$ ist, wissen wir $z\in D[\widetilde G]$ und $\widetilde Gz = (A+B)z$,  d. h. $A+B\subseteq \widetilde G$.

% \par
% Zu (2): Für hinreichend großes $\lambda>0$ existiert $R(\lambda, \widetilde G)$ und ist abgeschlossener, beschränkter Operator \Cref{}. Wegen $R(\lambda, K_r)X = D(A)\subseteq D(\widetilde G)$ ist wegen ref{}
% \begin{equation}
% R(\lambda, \widetilde G) - R(\lambda, K_r) = R(\lambda, \widetilde G)(\widetilde G-K_r) R(\lambda, K_r),
% \end{equation}
% und wegen $\widetilde G-K_r = A +B - (A + rB) = (1-r)B$ für alle $x\in D(A)$ ist 
% \begin{equation}
% R(\lambda, \widetilde G) - R(\lambda, K_r) = (1-r)R(\lambda, \widetilde G)BR(\lambda, K_r).
% \end{equation}
% Da alle Faktoren der rechten Seite positiv sind, ist $R(\lambda, K_r)\leq R(\lambda, \widetilde G)$ und mit der \index{}\textbf{Inversionsformel} \Cref{} ist $T_K(t)\leq \widetilde T_K(t)$.
% \end{proof}


% \section{Minimalität der Fortsetzung $K$}



% \par
% \hl{Dabei genügt es, Fortsetzungen von $(A+B, D)$ mit $D$ Kern von $A+B$ zu betrachten.}



% \begin{prop}\cite{kato_1954}\label{G ist Kontraktion von A}
% Der Operator $(Q, D(Q))$ ist Fortsetzung des Generators $K$.
% \end{prop}

% \begin{proof}
% \par
% Betrachte die Identität nach \Cref{}
% \begin{equation}
% R(\lambda, A)BR^{(n)}(\lambda) = R^{(n+1)}(\lambda) - R(\lambda, A).
% \end{equation}
% Für $n\to\infty$  mit \Cref{} ist dann die starke Konvergenz 
% \begin{equation}
% R(\lambda, A)BR^{(n)}(\lambda)\to R(\lambda, K)-R(\lambda, A)
% \end{equation}
% Sei $y\in X_+$ und setze $x=:R(\lambda, K)y$ sowie $x^{(n)}:=R^{(n)}(\lambda)y$. Dann wissen wir $R(\lambda, A) Bx^{(n)}\to x-R(\lambda, A)y$. Weiter bemerken wir, dass $0\leq x^{(1)}\leq x^{(2)}\leq \dots$ und $x^{(n)}\to x$. Komponentenweise lässt sich dies schreiben als 
% \begin{equation}
% (\lambda + q_k)^{-1}\sum_{j\neq k}\xi_j^{(n)}q_{jk}\to \xi_k -(\lambda + q_k)^{-1}\eta_k,
% \end{equation}
% was dasselbe ist wie 
% \begin{equation}
% \sum_{j\neq k}\xi_j^{(n)} q_{jk}\to(\lambda + q_k)\xi_k -\eta_k,
% \end{equation}
% wobei $0\leq \xi_j^{(1)}\leq\xi_j^{(2)}\leq\dots\leq \xi_j^{(n)}\to \xi_j$ wegen \Cref{}. Da $q_{jk}\geq 0$ für $j\neq k$ ist die absolute Konvergenz mit $n\to\infty$ von
% \begin{equation}
% \sum_{j\neq k}\xi_j q_{jk}=(\lambda + q_k)\xi_k - \eta_k,
% \end{equation}
% also $\sum_j \xi_j (\lambda \delta_{jk}-q_{jk})=\eta_k$. Damit ist also $x\in D(Q)$ und $(\lambda I - Q)x =y$. Wir haben also gezeigt, dass $R(\lambda, Q)y\in D(Q)$ und $(\lambda I - Q)R(\lambda, K)y = y$ für alle $y\in X_+$, folglich also auch für alle $y\in X$ nach \Cref{}. 
% \end{proof}

% \begin{proof}
% \par 
% Zu (1): mit \Cref{}, \Cref{} ist $K$ Generator einer Kontraktionshalbgruppe, welche $Q_0$ mit $Q_0\subseteq A+B\subseteq G$ fortsetzt. 

% \par
% Zu (2): mit \Cref{} ist die Minimalität und Eindeutigkeit der von $K$ erzeugten Kontraktionshalbgruppe.
% \end{proof}

% \section{Isometrieeigenschaft von $(T_K(t))_{t\geq0}$}

% \begin{satz}\cite{kato_1954}\label{Charakterisierung der Normiertheit von T_K(t)}
% Es seien die Voraussetzungen in \Cref{Hauptaussage in AL} für Operatoren $(A, D(A))$ und $(B, D(B))$ in $X$ AL-Raum mit $D(A)\subseteq D(B)$ gegeben. Ist der Operator $-A$ positiv, dann sind äquivalent:
% \begin{enumerate}
% \item $(T_K(t))_{t\geq0}$ ist eine Isometrie.
% \item Für alle $\lambda >0$ ist $\big((BR(\lambda, A))^n \big)_{n\in\N}$ eine stark konvergente Nullfolge. 
% \end{enumerate}
% \end{satz}

% % \begin{lem}\cite{kato_1954}\label{Eigenschaften von k}
% % Sei $x\in X_+$, $\lambda >0$. Dann existiert der Grenzwert
% % \begin{equation*}
% % \lim_{n\to\infty}\|(BR(\lambda, A))^n x\|=:\lim_{n\to\infty}\|(BR(\lambda, A))^n x\|
% % \end{equation*}
% % und weiter gilt:
% % \begin{enumerate}
% % \item Für $\lambda$ fest gewählt ist $\lim_{n\to\infty}\|(BR(\lambda, A))^n x\|$ eine positive, lineare Funktion in $x$.
% % \item Für $x$ fest gewählt ist $\lim_{n\to\infty}\|(BR(\lambda, A))^n x\|$ entweder die Nullfunktion oder eine strikt positive,  fallende Funktion in $\lambda$.   
% % \end{enumerate}
% % \end{lem}

% \begin{proof}

% \par 
% Zu Hinreichtung: Für $\lambda >0$ und $n\in \N$ betrachte erneut
% \begin{equation*}
% R^{(n)}(\lambda) = \sum_{k=0}^n R(\lambda, A)(BR(\lambda, A))^k.
% \end{equation*}
% Dann gilt 
% \begin{equation*}\label{eq2}
% %\sum_{k=0}^{n+1} (BR(\lambda, A))^k
%  I +BR^{(n)}(\lambda)=(\lambda I -A)R^{(n)}(\lambda) + (BR(\lambda, A))^{n+1}.
% \end{equation*} 
% Sei $x\in X_+$. Mit $-A\geq0$ sind alle beteiligten Terme positiv und da $X$ AL-Raum ist, erhalten wir
% \begin{align*}
% \|x\| + \|BR^{(n)}(\lambda)x\|= \lambda\| R^{(n)}(\lambda) x\| + \|AR^{(n)} (\lambda) x\|+\|(BR(\lambda, A))^{n+1}x\|.
% \end{align*}
% \hl{Mit $R^{(n)}(\lambda)x\in D(A)_+$ ist $\|BR^{(n)}(\lambda)x\|=\|AR^{(n)}(\lambda) x\|$} und damit
% \begin{equation*}
% \|x\|=\lambda \|R^{(n)}(\lambda)x\| + \|(BR(\lambda, A))^{n+1}x\|.
% \end{equation*}
% Mit $\lim_{n\to\infty}R^{(n)}(\lambda)x= R(\lambda, K)x$ wird dies zu
% \begin{align*}
% \lim_{n\to\infty}\|(BR(\lambda, A))^n x\|=\|x\|- \lambda\|R(\lambda, K)x\|
% \end{align*}
% Mithilfe der  \index{}\textbf{Integraldarstellung der Resolvente} können wir dies umformulieren zu
% \begin{align*}
% \lim_{n\to\infty}\|(BR(\lambda, A))^n x\| = \|x\|- \lambda\int_0^\infty \exp(-\lambda t)\|T_K(t)x\|\textnormal dt.
% \end{align*}
% Da $(T_K(t))_{t\geq0}$ normiert ist, wird dies zu
% \begin{equation*}
% \lim_{n\to\infty}\|(BR(\lambda, A))^n x\| =\lambda \int_0^\infty \exp(-\lambda t)(\|x\|-\|T_K(t)x\|)\textnormal dt=0
% \end{equation*}
% Folglich ist  $(BR(\lambda, A))^n$ stark konvergent mit $\lim_{n\to\infty}(BR(\lambda, A))^{n}x=0$.

% \par
% Zur  Rückrichtung: Sei $x\in X_+$. Betrachte erneut
% \begin{equation*}
% \|x\| - \lim_{n\to\infty}\|(BR(\lambda, A))^n x\| = \lambda\int_0^\infty \exp(-\lambda t)\|T_K(t)x\|\textnormal dt\geq0.
% \end{equation*} 
% Wegen $0\leq \lim_{n\to\infty}\|(BR(\lambda, A))^n x\|\leq \|x\|$ ist die Folge $(BR(\lambda, A))^n$  punktweise beschränkt. Da $BR(\lambda, A)$ für  $\lambda\to 0$ monoton fallend ist, existiert $\lim_{\lambda\downarrow 0} \lim_{n\to\infty}\|(BR(\lambda, A))^n\|$ punktweise \hl{und es gilt}
% \begin{equation*}
% \|x\| - \lim_{\lambda\downarrow0}\lim_{n\to\infty}\|(BR(\lambda, A))^n x\|=\lim_{t\to\infty} \|T_K(t)x\|.
% \end{equation*}
% Wegen $\lim_{n\to\infty}\|(BR(\lambda, A))^n x\| =0$ erhalten wir 
% \begin{equation*}
% \lim_{t\to\infty} \|T_K(t)x\| = \|x\|.
% \end{equation*}
% Wegen $\|T_K(t+s)x\|=\|T_K(s)T_K(t)x\|\leq \|T_K(t)x\|$ ist $(T_K(t))_{t\geq0}$ für $t\to\infty$ monoton fallend. Dann liefert $\lim_{t\to\infty}\|T_K(t)\|\geq \|x\|$, dass $\|T_K(t)x\|\geq\|x\|$ für alle $t\geq0$ gilt. 


\par
% Zu (3): Sei $0\neq x\in X_+$. Dann ist die Funktion $\lambda^{-1}(\|x\|-\lim_{n\to\infty}\|(BR(\lambda, A))^n x\|)$ strikt positiv. Wegen $k(\lambda, x)$ fallend ist diese Funktion ebenfalls fallend in $\lambda$. Die Aussage folgt mit Grenzübergang $\lambda \downarrow 0$.



% Damit ist $\lim_{n\to\infty}\|(BR(\lambda, A))^n x\|$ für $\lambda >0$ fest gewählt eine positive, lineare Funktion in $x$.



\par 
% Wir zeigen, dass $\lim_{n\to\infty}\|(BR(\lambda, A))^n x\|$ monoton fallend ist. Nach Bemerkung \Cref{BR(lambda, A) fallend} ist $0\leq BR(\mu, A)\leq BR(\lambda, A)$ für $\lambda >\mu$ und damit auch
% \begin{equation*}
% 0\leq (BR(\mu, A))^n x\leq (BR(\lambda, A))^n x, \quad \forall x\in X_+.
% \end{equation*}
% Demnach ist $\lim_{n\to\infty}\|(BR(\mu, A))^n x\|\leq \lim_{n\to\infty}\|(BR(\lambda, A))^n x\|$, d. h. $\lim_{n\to\infty}\|(BR(\lambda, A))^n x\|$ ist fallende Funktion in $\lambda$. 

% \end{proof}

% \begin{satz}\cite{kato_1954}\label{Vollständige Charakterisierung der Normiertheit von T_K(t)}
% Es sind äquivalent:
% \begin{enumerate}
% \item Für alle $x\in X_+$ sowie $t\geq 0$ gilt $\|T_K(t)x\| = \|x\|$.
% \item Für alle $\lambda >0$ ist $(BR(\lambda, A))^n\to 0$ stark konvergent.
% \item Für alle $\lambda >0$ besitzt $A^*_0 x^* = \lambda x^*$ mit $0\neq x^*\in X^*$ keine Lösung.
% \item Für alle  $\lambda >0$ liegt  $(\lambda I - Q_0)$ dicht in $X$.
% \end{enumerate}
% \end{satz}

% % \begin{bem}\cite{kato_1954}\label{bem:}
% % \textcolor{red}{Die Äquivalenz ist weiterhin gültig, falls die Forderung ''Für alle $\lambda >0$'' durch die Forderung ''Es gibt $\lambda >0$'' abgeschwächt wird .}
% % \end{bem}

% \begin{proof}
% \par 
% Zur Äquivalenz von (1), (2): ist klar.

% \par
% Zur Äquivalenz von (3), (4): [...]

% \par
% Zur  Äquivalenz von (2), (4): wegen
% \begin{equation}
% I +BR^{(n)}(\lambda)=(\lambda I - A)R^{(n)}(\lambda) + (BR(\lambda, A))^{n+1}
% \end{equation}
% gilt $(BR(\lambda, A))^n\to 0$ stark genau dann, wenn $\lim_{n\to\infty}(\lambda I -A-B)R^{(n)}(\lambda)= I$ stark konvergiert. 

% \par 
% Sei  $x\in X$. Wegen $R^{(n)}(\lambda)x\in D(A)$ ist $(\lambda I -A-B)D(A)\subseteq X$ dicht. Zeige, dass $(\lambda I -Q_0)D(Q_0) = (\lambda I -A-B)D(Q_0)$ dicht in $(\lambda I -A-B)D(A)$ liegt. [...]

% \par
% Sei umgekehrt $(\lambda I - Q_0)D(Q_0)$ dicht in $X$. Wegen 
% \begin{align}
% (I-BR(\lambda, A))X
% &=(I-BR(\lambda, A))(\lambda +A)D(A)\\
% &= (\lambda I - A - B)D(A)\supseteq (\lambda I -Q_0)D(Q_0)
% \end{align}
% ist $(I-BR(\lambda, A))X$ ebenfalls dicht definiert. Setze \begin{equation}
% K^{(n)}(\lambda):=(n+1)^{-1}(I + BR(\lambda, A)+ \dots + (BR(\lambda, A))^n).
% \end{equation}
% Wegen $\|BR(\lambda, A)\|\leq 1$ ist mit $n\to\infty$ für alle $x\in X$ 
% \begin{equation}
% K^{(n)}(\lambda)(I-BR(\lambda, A))x = (n+1)^{-1}(I-(BR(\lambda, A))^{n+1})x\to 0.
% \end{equation}
% Da $(I-BR(\lambda, A))X$ dicht in $X$ liegt sowie $K^{(n)}(\lambda)$ mit $\|K^{(n)}(\lambda)\|\leq 1$ gleichmäßig beschränkt ist, muss $K^{(n)}(\lambda)\to 0$ stark konvergieren. Zeige $\|K^{(n)}(\lambda)\|\geq \|(BR(\lambda, A))^n x\|$ für $x\in X_+$. Dann  folgt die Behauptung $(BR(\lambda, A))^nx\to 0$ mit der starken Konvergenz von $K^{(n)}(\lambda)\to 0$. Wegen $(BR(\lambda, A))^k\geq 0$ sowie  \begin{align}
% \|(BR(\lambda, A))^n x\| 
% &= \|B_\lambda^{n-m} (BR(\lambda, A))^m x\|\\
% &\leq \|(BR(\lambda, A))^m x\|
% \end{align} 
% ist $\|K^{(n)}(\lambda)\| = (n+1)^{-1}\sum_{k=0}^n\|(BR(\lambda, A))^k x\|\geq \|(BR(\lambda, A))^n x\|$ für alle $x\in X_+$. Damit ist $(BR(\lambda, A))^n x\to 0$ für alle $x\in X_+$, also auch $(BR(\lambda, A))^n\to 0$ stark.
% \end{proof}

% \begin{satz}\cite{kato_1954}\label{Hinreichende Bedingung für Eindeutigkeit}
% Sei $K$ Generator einer Kontraktionshalbgruppe $(T_K(t))_{t\geq 0}$ mit $Q_0\subseteq K$. Dann sind äquivalent:
% \begin{enumerate}
% \item  $(T_K(t))_{t\geq 0}$ ist normiert.
% \item  $(T_K(t))_{t\geq 0}$ ist eindeutig bestimmt.
% \end{enumerate}
% \end{satz}

% \begin{proof}
% Für hinreichend großes $\lambda >0$ ist  Resolvente $(\lambda I - K)^{-1}$ eine beschränkte Fortsetzung von $(\lambda I -Q_0)^{-1}$, d. h. es ist 
% \begin{equation}
% (\lambda I - Q_0)^{-1}\subseteq (\lambda I - K)^{-1}\subseteq X.
% \end{equation}
% Da aber nach \Cref{4} die Menge $(\lambda I - Q_0)$ dicht in $X$ liegt, muss $(\lambda I -K)^{-1}$ bereits eindeutig bestimmt sein. Damit ist aber auch der Generator $K$ eindeutig bestimmt, und die Behauptung folgt.
% \end{proof}



\chapter{Anwendungen}

\section{Geburts- und Todesprozesse (GTP)}

\par
Wir wollen mithilfe eines \textbf{Geburts- und Todesprozess} (GTP)
%, der Spezialfall eines \textbf{Kolmogorov'schen Differentialsystem}, 
die Entwicklung einer Population in Abhängigkeit der Zeit beschreiben. Dieses abzählbare System an Elementen ist gegeben durch
\begin{align*}
x_0' &= -a_0x_0 + d_1 x_1\\
&\;\;\vdots\\
x_n' &= -a_nx_n + d_{n+1}x_{n+1} + b_{n-1}x_{n-1}  \\
&\;\;\vdots
\end{align*}

\par 
In diesem Fall ist bezeichnet $x_n$ die \textbf{Wahrscheinlichkeit}, mit der die Population zu einem gewissen Zeitpunkt $t$ aus $n$ Individuen besteht. Mithilfe eines (nicht spezifizierten) Mechanismus kann sich der Zustand des Systems zu $k+1$, vermöge der 'Geburt' eines Individuums, oder analog zu $k-1$, vermöge des 'Todes' eines Individuums, ändern; bezeichnen die Wahrscheinlichkeit, mit der eine Zustandsänderung $k\mapsto k+1$ auftritt mit  $b_k$ und analog $d_k$ für $k\mapsto k-1$; mit $a_n=d_n+b_n$ bezeichnet man obiges System auch als klassischen GTP.

In \cite{kato_1954} zeigt Kato die Existenz einer $C_0$-Halbgruppe, welche obiges Kolmogorov'sche Differentialsystem für alle absolut summierbaren Folgen $\textbf{x}\in l_1$ löst. Das Ergebnis wurde  in \cite{} auf AL-Räume verallgemeinert, welche die Räume $l_1$ und $l_1$ umfassen. In diesem Text stellen wir das Ergebnis von Banasiak et al.  \cite{banasiak_lachowicz_2007} vor, welches die Existenz für Operatoren in KB-Räumen verallgemeinert. Diese umfassen neben den AL-Räumen auch reflexive Banachräume. 



% Gegeben sei ein klassischer Geburts- und Todesprozess
% \begin{align*}
% x_0' &= -a_0x_0 + d_1 x_1\\
% &\;\;\vdots\\
% x_n' &= -a_nx_n + d_{n+1}x_{n+1} + b_{n-1}x_{n-1}  \\
% &\;\;\vdots
% \end{align*}
% Wir bezeichnen mit fett gedruckten Lettern die zugehörigen Folgen an Koeffizienten, etwa $\textbf{x}=(x_0, x_1,\dots, x_n,\dots)$; ohne Einschränkung können wir $b_{-1}=d_0=0$ setzen (vgl. \cite{}). Weiter seien die Folgen $\textbf{d}, \textbf{b}$ und $\textbf{a}$ alle nicht-negativ. Wir das System für alle Elemente im Raum $X=l_p$ aller $p$-fach  summierbaren Folgen betrachten.

% \begin{konstr}
% Setzte $\mathcal K$ für die Koeffizientenmatrix aller Einträge der rechten Seite obigen Systems. Für alle Folgen $\textbf{x}\in l_p$ ist 
% \begin{equation*}
% \mathcal K\textnormal{\textbf{x}}:=\{b_{n-1}x_{n-1} - a_nx_n + d_{n+1}x_{n+1}\}_{n\in\N_0},
% \end{equation*}
% und analog definiere $\mathcal A$, $\mathcal B$ mit
% \begin{equation*}
%  \mathcal A\textnormal{\textbf{x}} := \{-a_nx_n\}_{n\in\N_0},\quad 
%  \mathcal B\textnormal{\textbf{x}} := \{b_{n-1}x_{n-1} + d_{n+1}x_{n+1}\}_{n\in\N_0}.
% \end{equation*} 
% \end{konstr}


% Die formalen Abbildungen $\mathcal A$ und $\mathcal B$ können, je nach Definitionsbereich,  verschiedene Operatoren auf $X$ definieren. Bezeichne etwa mit $(A, D(A))$ die Einschränkung von $\mathcal A$ auf den Definitionsbereich 
% \begin{equation}
% D(A)=\{\textbf{x}\in X;\mathcal A\textbf x\in X\}.
% \end{equation}
% Insbesondere ist damit $\mathcal B\textbf{x}\in X_+$ für alle  $\textbf x\in D(A)_+$.

% \par
% Damit können wir den positiven Operator $(B, D(B))$ als Einschränkung von $\mathcal B$ auf $D(A)$ definieren, also $D(B)=D(A)$. Für alle $x\in D(A)$ ist 
% \begin{equation}
% \|B\textbf{x}\|\leq\|A\textbf{x}\|.
% \end{equation}

% \par
% Damit kann der Geburts- und Todesprozess verstanden werden als
% \begin{equation}
% \textbf{x'}=\mathcal A\textbf{x} + \mathcal B\textbf{x}.
% \end{equation}

% Wir möchten die Existenz einer Familie von Operatoren zeigen, welche das Kolmogorov'sche Differentialsystem "löst". [...]

\begin{bem}
Im folgenden bezeichnen wir Folgen stets mit fett gedruckten Lettern, etwa $\textbf{x}=(x_0,x_1,\dots)$. Weiter nehmen wir an, dass die Folgen $\textbf{d},\textbf{b}$ und $\textbf{a}$ nicht-negativ sind und $b_{-1}=d_0=0$ gilt.
\end{bem}

\begin{konstr}
Gegeben sei der formale Operator  $\mathbb  K$ in $l$ eines GTP, d. h. für die nicht-negativen Koeffizientenfolgen $\textbf{b},\textbf{a}$ und $\textbf{d}$ setze:
\begin{equation*}
(\mathbb K\textnormal{\textbf{x}})_n:=b_{n-1}x_{n-1} -a_nx_n + d_{n+1}x_{n+1},\quad \forall n\in\N,\quad\forall \textbf{x}\in l.
\end{equation*}
Analog definiere die formalen Operatoren $\mathbb A$ und $\mathbb B$ in $l$ durch:
\begin{equation*}
(\mathbb A\textbf{x})_n:=-a_nx_n\quad\text{ und }\quad (\mathbb B\textbf{x})_n:=b_{n-1}x_{n-1} + d_{n+1}x_{n+1},\quad\forall n\in\N,\forall \textbf{x}\in l.
\end{equation*}
\end{konstr}

\begin{defi}Für $p\in[1,\infty)$ bezeichne $(\mathcal K_p, D(\mathcal K_p))$ die \textbf{maximale Realisierung} von $\mathbb K$ in $l_p$, wobei
\begin{equation*}
\mathcal K_p\textnormal{\textbf{x}}:= \mathbb K\textnormal{\textbf{x}},\quad \textbf{x}\in D(\mathcal K_p):=\{\textnormal{\textbf{x}}\in l_p; \mathbb K\textnormal{\textbf{x}} \in l_p\}.
\end{equation*}
\end{defi}

\begin{prop}\label{Abgeschlossenheit des maximalen Operators K_p}
Für alle $p\in[1,\infty)$ ist $(\mathcal K_p, D(\mathcal K_p))$ ein abgeschlossener Operator.
\end{prop}

\begin{proof}
Sei $(\textnormal{\textbf{x}}^{(n)})_{n\in\N}$ eine konvergente Folge in $l_p$ mit $\textnormal{\textbf{x}}^{(n)}\to \textnormal{\textbf{x}}$ und $\mathcal  K_p\textnormal{\textbf{x}}^{(n)}\to\textnormal{\index{}\textbf y}$. Dann ist $(\textnormal{\textbf{x}}^{(n)})_{n\in\N}$ punktweise konvergent, für alle $k\in\N$ ist also $x_k^{(n)}\to x_k$. \hl{Nach Definition von $\mathcal  K_p$ also $y_k = b_{k-1}x_{k-1} -a_k x_k + d_{k+1} x_{k+1}$, also auch $\mathcal  K_p\textnormal{\textbf{x}} = \textnormal{\index{}\textbf y}$.}
\end{proof}

\begin{defi}
Für  $p\in[1,\infty)$ bezeichne $(A_p, D(A_p))$  die Einschränkung von $\mathbb A$ auf $D(A_p)$, wobei
\begin{equation*}
A_p\textbf{x} := \mathbb A\textbf{x}, \quad \textbf{x}\in D(A_p):= \{\textnormal{\textbf{x}}\in l_p; \mathbb A\textnormal{\textbf{x}}\in l_p\}=\Big\{\textnormal{\textbf{x}}\in l_p; \sum_{n=0}^\infty a_n^p |x_n|^p < \infty \Big\}.
\end{equation*}
Analog bezeichne $(B_p, D(B_p))$ die Einschränkung $\mathbb B|_{D(A_p)}$.
\end{defi}

\begin{prop}\label{A_p Generator einer Kontraktionshalbgruppe}
Für alle $p\in[1,\infty)$ ist $(A_p, D(A_p))$ Generator einer positiven Kontraktionshalbgruppe in $l_p$.
\end{prop}

\begin{proof}
\par 
Nach Konstruktion ist $A_p$ dicht definiert. 

\par
Sei $\lambda>0$. Dann ist $\lambda\in \varrho(A_p)$ und $R(\lambda, A_p)$ ist für alle $\textbf{y}\in l_p$ gegeben durch
\begin{equation*}
(R(\lambda, A_p)\textnormal{\index{}\textbf y})_n = \frac{y_n}{\lambda + a_n},\quad \forall n\in\N_0.
\end{equation*}
Die Beschränktheit von $A_p$ erhalten wir mit
\begin{equation*}
\|A_pR(\lambda, A_p)\textnormal{\index{}\textbf y}\|_p^p = \sum_{n=0}^\infty \frac{a_n^p}{(\lambda +  a_n)^p}|y_n|^p\leq\sum_{n=0}^\infty |y_n|^p=\|\textbf{y}\|_p^p,
\end{equation*}
Schließlich ist
\begin{equation*}
\|R(\lambda, A_p)\textnormal{\index{}\textbf y}\|_p^p = \sum_{n=0}^\infty \frac{1}{(\lambda + a_n)^p}|y_n|^p \leq \frac{1}{\lambda^p}\|\textnormal{\index{}\textbf y}\|_p^p.
\end{equation*}
Mit \textbf{Hille-Yosida} ist somit $(A_p, D(A_p))$ Generator einer Kontraktionshalbgruppe $(T_{A_p}(t))_{t\geq0}$ in $l_p$. Mit $R(\lambda, A_p)\geq0$ ist $(T_{A_p}(t))_{t\geq0}$ positiv.
\end{proof}

\section{Störungstheorem für GTP in $l_p$}

\begin{fsatz}
Es seien die Operatoren  $(A_1, D(A_1))$ und  $(B_1, D(B_1))$ in $l_1$ gegeben. Angenommen, für die nicht-negativen Folgen  $\textbf{\textnormal{b}},\textbf{\textnormal{d}}\in l_1$ gilt: 
\begin{equation*}
a_n\geq (b_n + d_n),\quad \forall n\in\N_0.
\end{equation*}
Dann existiert eine Fortsetzung $(K_1, D(K_1))$ von $(A_1+B_1, D(A_1))$, welche Generator einer positiven Kontraktionshalbgruppe in $l_1$ ist.
\end{fsatz}

\begin{proof}
\par
Wir wollen \Cref{Hauptaussage in AL} anwenden.

\par
Zu (B1), (B2): Analgo zum Beweis von (1) und (2) in \Cref{Fortsetzung von K_p} sind die Aussagen (B1) und (B2)  erfüllt.

\par
Zu (B3): Sei $\textbf{x}\in D(A_1)_+$. Mit der Annahme ist $0\leq b_n\leq a_n$ und $0\leq d_n\leq a_n$ für alle $n\in\N$ und mit $b_{-1}=d_0=0$ erhalten wir:
\begin{align*}
\sum_{n=0}^\infty ((A_1+B_1)\textbf{x})_n
&= -\sum_{n=0}^\infty a_n x_n \sum_{n=0}^\infty b_{n-1}x_{n-1}+\sum_{n=0}^\infty d_{n+1}x_{n+1}\\
&=-\sum_{n=0}^\infty a_n x_n + \sum_{n=0}^\infty b_n x_n+\sum_{n=0}^\infty d_n x_n\leq 0.
\end{align*}
\end{proof}


% \begin{folg}
% Es gilt stets $K_1\subseteq \mathcal K_1$.
% \end{folg}

% \begin{proof}

% \end{proof}

% \begin{fsatz}
% Es gilt $K_1=\overline{A+B}$ genau dann, wenn 
% \begin{equation}
% \sum_{n=0}^\infty \frac{1}{b_n}\Bigg(\sum_{i=0}^\infty\prod_{j=1}^i\frac{d_{n+j}}{b_{n+j}}\Bigg) = +\infty.
% \end{equation}
% \end{fsatz}

% \begin{proof}

% \end{proof}

\begin{fsatz}\cite{banasiak_lachowicz_2007}\label{Fortsetzung von K_p}
Sei $p\in(1,\infty)$ und die  Operatoren $(A_p, D(A_p))$ und  $(B_p, D(B_p))$ in $l_p$ gegeben. Agenommen, die Koeffizientenfolgen $\textnormal{\textbf{d}}, \textnormal{\textbf{b}}\in l_p$ sind nicht-fallend und es gibt $\alpha\in[0,1]$ so, dass gilt:
% \begin{enumerate}
% \item $0\leq b_n\leq \alpha a_n$ für alle $n\in\N_0$.
% \item $0\leq d_{n+1}\leq (1-\alpha)a_n$ für alle $n\in\N_0$.
% \end{enumerate}
\begin{equation*}\label{Annahme für Abschätzung}
0\leq b_n\leq \alpha a_n,\quad 0\leq d_{n+1}\leq (1-\alpha)a_n, \quad \forall n\in \N_0.
\end{equation*}
Dann existiert eine Fortsetzung $(K_p, D(K_p))$ von  $(A_p + B_p, D(A_p))$, welche Generator einer positiven Kontraktionshalbgruppe in $l_p$ ist.
\end{fsatz}

\begin{proof}
\par
Zeige, dass die Voraussetzungen für \Cref{Hauptaussage} erfüllt sind.

\par
Zu (1): Dies entspricht der Behauptung in \Cref{A_p Generator einer Kontraktionshalbgruppe}.

\par
Zu (2): Sei $\textbf{x}\in D(A_p)_+$. Da $-A_p$ positiv ist, reicht es zu zeigen, dass  $\|B_p\textbf{x}\|_p\leq\|A_p\textbf{x}\|_p$ gilt (siehe  \Cref{stärkere Annahme (2)}). Mit $b_{-1}=d_0=0$ ist
\begin{align*}
\|B_p\textnormal{\textbf{x}}\|_p
&= \Big(\sum_{n=0}^\infty |b_{n-1}x_{n-1} + d_{n+1}x_{n+1}|^p\Big)^{1/p}\\
&\leq \Big(\sum_{n=0}^\infty b_{n-1}^p |x_{n-1}|^p\Big)^{1/p} + \Big(\sum_{n=0}^\infty d_{n+1}^p |x_{n+1}|^p\Big)^{1/p}\\
&\leq \Big(\sum_{n=0}^\infty b_n^p|x_n|^p\Big)^{1/p} + \Big(\sum_{n=0}^\infty d_n^p |x_n|^p\Big)^{1/p}.
\end{align*}
Mit der Monotonie von $\textbf{d}$ ist $d_{n}\leq d_{n+1}\leq (1-\alpha)a_n$ und damit
\begin{align*}
\|B_p\textnormal{\textbf{x}}\|_p
&\leq \Big(\sum_{n=0}^\infty \alpha a_n^p|x_n|^p\Big)^{1/p} + \Big(\sum_{n=0}^\infty a_n^p |x_n|^p\Big)^{1/p} - \Big(\sum_{n=0}^\infty \alpha a_n^p |x_n|^p\Big)^{1/p}\\
&= \Big(\sum_{n=0}^\infty a_n^p |x_n|^p\Big)^{1/p}= \|A_p \textnormal{\textbf{x}}\|_p.
\end{align*}
Da $B_p$ mit $\textbf{b},\textbf{d}\geq0$ positiver Operator ist, gilt dies bereits für alle $\textbf{x}\in D(A)$. Somit ist $r(B_p(R(\lambda, A_p))\leq 1$ für alle $\lambda>0$.

\par
Zu (3): Klar, da $(B_p\textbf{x})_n=b_{n-1}x_{n-1}+d_{n+1}x_{n+1}$ für alle $n\in\N_0$ und $\textbf{b},\textbf{d}\geq0$ gilt.

\par
Zu (4): Sei $\textnormal{\textbf{x}}\in D(A_p)_+$. Definiere $\tilde{\textnormal{\textbf{x}}} = (\tilde x_n)_{n\in\N}$ durch
\begin{equation*}
\tilde x_n := 
\begin{cases}
0\quad&\text{falls }x_n=0,\\
x_n^{p-1}&\text{sonst.}
\end{cases}
\end{equation*}
Sei $q\in(1,\infty)$ mit $1/p + 1/q=1$. Dann ist $\tilde{\textnormal{\textbf{x}}}\in l_q=l_p^*$. \hl{Damit gilt  Multiplizieren wir $\tilde{\textbf{x}}$ mit dem Faktor $\|\textbf{x}\|_p^{1-p}$, so erhalten wir  $\langle \tilde{\textbf{x}},\textbf{x}\rangle = \|\textbf{x}\|$. }

\par
Für alle $n\in\N$ sei ohne Einschränkung $x_n\neq 0$. Weiter können wir  $\tilde{\textbf{x}}$ ohne die Multiplikation des Faktors $\|\textbf{x}\|_p^{1-p}$ für den Beweis der Abschätzung $\langle(A_p+B_p)\textbf{x} ,\tilde{\textbf{x}} \rangle\leq0$ annehmen. Da $a_n\geq (b_n + d_{n+1})$ für alle $n\in\N_0$ ist, gilt
\begin{align*}
\langle &(A_p+B_p)\textnormal{\textbf{x}}, \tilde{\textnormal{\textbf{x}}}\rangle\\
&= \sum_{n=0}^\infty (A_p\textnormal{\textbf{x}}+B_p\textnormal{\textbf{x}})_n x_n^{p-1}\\
&= -\sum_{n=0}^\infty a_n x_n^{p} + \sum_{n=0}^\infty b_{n-1} x_{n-1} x_n^{p-1} + \sum_{n=0}^\infty d_{n+1} x_{n+1}x_n^{p-1}\\
&\leq -\sum_{n=0}^\infty b_n x_n^p - \sum_{n=0}^\infty d_{n+1}x_n^p +\sum_{n=0}^\infty b_{n-1} x_{n-1} x_n^{p-1} + \sum_{n=0}^\infty d_{n+1}x_{n+1} x_n^{p-1}.
\end{align*}
Mit der \index{}\textbf{Hölder-Ungleichung} erhalten wir
\begin{align*}
\langle (A_p+B_p)\textnormal{\textbf{x}}, \tilde{\textbf{x}}\rangle 
\leq &\Big(\sum_{n=0}^\infty b_n x_n^p\Big)^{1/p}\Big(\sum_{n=0}^\infty b_n x_{n+1}^p\Big)^{1/p} - \sum_{n=0}^\infty b_n x_n^p\\
& + \Big(\sum_{n=0}^\infty d_n x_n^p \Big)^{1/p}\Big(\sum_{n=0}^\infty d_{n+1}x_n^p\Big)^{1/q} - \sum_{n=0}^\infty d_{n+1} x_n^p.
\end{align*}
Wegen $b_n\leq b_{n+1}$ und  $d_n\leq d_{n+1}$ können wir somit  $\langle (A_p+B_p)\textnormal{\textbf{x}}, \tilde{\textnormal{\textbf{x}}}\rangle \leq 0$ folgern.
\end{proof}

% \section{Eigenschaften der Fortsetzung}

% \begin{satz}\label{toller satz}
% Sei $X$ Banachraum sowie  $(A, D(A))$ und $(B, D(B))$ Operatoren in $X$ mit $D(A)\subseteq D(B)$. Angenommen, es gilt:
% \begin{enumerate}
% \item $A$ ist Generator einer Kontraktionshalbgruppe  $(T_A(t))_{t\geq0}$.
% \item Für alle $t\in[0, 1]$ ist  $A+tB$ dissipativ.
% \item Es gibt $0\leq\alpha\leq 1 $ und $\beta\geq0$ derart, dass für alle $x\in D(A)$ gilt
% \begin{equation*}
% \|Bx\|\leq \alpha\|Ax\| + \beta\|x\|.
% \end{equation*}
% \end{enumerate}
% Dann folgt:
% \begin{enumerate}
% \item Für $\alpha < 1$ ist $A+B$ Generator einer Kontraktionshalbgruppe.
% \item Ist zudem $B^*$  in $X^*$ dicht definiert, dann ist für $\alpha =1$ der Abschluss $\overline{A+B}$ Generator einer Kontraktionshalbgruppe.
% \end{enumerate}
% \end{satz}

% \begin{bem}
% \hl{Es seien die Voraussetzungen in \mbox{\Cref{toller satz}} gegeben. Ist $X$ reflexiv und $B$ abschließbar, dann liegt $D(B^*)$ dicht in $X^*$.}
% \end{bem}

% \begin{proof}[Beweis des Satzes]
% Siehe \cite{}.
% \end{proof}



% \begin{satz}
% Sei $p\in(1,\infty)$. Dann gilt $K_p=\overline{A_p + B_p}$.
% \end{satz}

% \begin{proof}
% Die Voraussetzungen in \Cref{toller satz} sind für die Operatoren $A_p$ und $B_p$ erfüllt. Mit  $p\in (1,\infty)$ ist $l_p$ reflexiv (vgl. \cite{}). Analog zu \Cref{Abgeschlossenheit des maximalen Operators K_p} können wir zeigen, dass $\mathcal B$ abgeschlossen ist. Dann ist  $B_p$ abschließbar und nach \Cref{toller satz} also  $\overline{A_p + B_p}$ Generator einer Kontraktionshalbgruppe. \hl{Da $K_p$ nach \mbox{\Cref{Minimalität von K}} minimal ist, erhalten wir $K_p=\overline{A_p + B_p}$.}
% \end{proof}



% \begin{bem}
% \hl{Für den Fall $p=1$ sehen wir mit \mbox{\Cite{banasiak_2004}}, dass $K_1\neq\overline{A_1+B_1}$ gilt.}
% \end{bem}


% \begin{satz}
% Sei $p\in[1,\infty)$. Dann gilt $K_p\subseteq \mathcal K_p$.
% \end{satz}


% \begin{proof}
% Sei $\textbf{x}^r\to\textbf{x}$ für $r\uparrow 1$ in $l_p$. Dann gilt für alle $n\in\N$
% \begin{align*}
% \lim_{r\to1}((I-\mathcal K_p)\textbf{x}^r)_n
% &=\lim_{r\to 1}x_n^r + a_nx_n^r - b_{n-1}x_{n-1}^r - d_{n+1}x_{n+1}^r\\
% &=x_n+a_nx_n - b_{n-1}x_{n-1} - d_{n+1}x_{n+1}\\
% &=((I-\mathcal K_p)\textbf{x})_n.
% \end{align*}
% Sei $\textbf{y}\in l_p$. Für  $\textbf{x}^r:=R(1, A+rB)\textbf{y}$ gilt mit \Cref{}
% \begin{equation*}
% \lim_{r\uparrow 1}\textbf{x}^r=\lim_{r\uparrow1}R(1, A+rB)\textbf{y}= R(1, K_p)\textbf{y}.
% \end{equation*}
% \hl{Da $R(1, A+rB)$ die Resolvente von $(A+rB, D(A))$ ist, welche eine Einschränkung der maximalen Realisierung von $\mathcal -\mathcal A+r\mathcal B$ ist, erhalten wir} für alle $n\in\N$
% \begin{align*}
% ((I-\mathcal K_p)\textbf{x}^r)_n
% &= x_n^r + a_nx_n^r - rb_{n-1}x_{n-1}^r - rd_{n+1}x_{n+1}^r\\
% &\quad -(1-r)(b_{n-1}x_{n-1}^r + d_{n+1}x_{n+1}^r)\\
% &=y_n - (1-r)(b_{n-1}^r + d_{n+1}x_{n+1}^r).
% \end{align*}
% Da $n$ fest gewählt ist, geht der letzte Term gegen Null und mit \Cref{} erhalten wir $((I-\mathcal K_p)\textbf{x})_n=y_n$, also $
% (I-\mathcal K_p)R(1, K_p)\textbf{y}=\textbf{y}$.
% \end{proof}



